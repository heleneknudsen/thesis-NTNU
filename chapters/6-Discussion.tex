\chapter{Discussion}
This chapter discusses the research questions by using the results found in the four test cases of this research. The questions will be answered sequentially. The chapter also discusses the work done in this thesis, both what was done good which lead to new knowledge and contributions, but also what could have been done differently. Challenges and mistakes which lead to decisions being made are explained here. Limitations to this research will also be included as this could have affected how strong the results are and can explain why certain decisions were made. 
\section{Answer to the research questions}
As presented in Chapter 1, the research questions to be answered in this thesis are:

\begin{enumerate}
    \item What kind of information can be gathered from air quality monitors when carrying out a passive network eavesdropping attack?
    \item What are the differences in level of inference from different air quality monitors from different vendors?
    \item How can the private information gathered be misused by an adversary?
\end{enumerate}

In this research, four different test cases have been used to test if it is possible see traffic pattern changes from when an event is triggered in the environment the device relies. By looking at calculations and graphical views over the traffic flows during the test, we were able to find results to answer the research questions in this thesis. The baseline capturings showed that all the devices communicates with a layer of security, as the packets sent on \gls{Wi-Fi} is encrypted and not readable while carrying out a passive network eavesdropping attack. Therefore it is not possible to gather private information sent in the payloads to and from the devices. 

For the routine behavioural events defined as cooking, showering and window open, only showering and window open gave notifications of threshold values exceeded for all executions. None of the devices showed a significant difference from when an event was triggered at a specific time compared to the traffic sent before and after the event or compared to the baseline capturings. While comparing the baseline traffic to the event traffic, the devices Nedis SmartLife and Mill Sense showed that there are visible differences from the baseline capturings to right before and after the events which should expectedly have been the same. This can be a result that the traffic can be affected by smaller indoor changes such as a season can change the indoor air quality. For example, during summer windows can be more open than during winter were ovens can be more used. 

However, for the weekend test, one of the devices shows a clear difference from when a user is home or gone during the time period of a weekend. When Netatmo Smart Indoor Air Quality Monitor was used, it shows that it is possible to distinguish whether a user is home or gone by looking at the total number of bytes sent and received to that device during the weekend and looking at size of the packets sent by the device which is never higher than 134 bytes when the user is gone for the weekend. This kind of information will expose private information about a user were it is possible to see whether the user is home or not for a longer period of time. These two factors proves that it is possible to infer private information form the device. For Mill Sense and Nedis SmartLife, no similar findings were discovered during the tests. 

In the weekend testing, it looked like there were differences for both Mill and Netatmo for the first 5 weekends at home, but only Netatmo had the same significantly different pattern for weekends at home and gone. For Mill, the pattern changed for the two last weekends at home where the packet size were the same as for the weekends gone. This shows the importance of including enough executions for each of the test cases. However, for over 70\% of the weekends at home, a clear difference is visible when a user is at home compared to gone. Due to time constraints, no more weekends were tested, but more tests could reveal if these two weekends were an exception or if it is not possible to distinguish a weekend at home or gone for this device. For Nedis, no significant differences were possible to see during the weekend testing and therefore private information about whether a user is at home or gone during the weekend was not possible to discover during these tests. 

The baseline capturings shows that the devices communicates differently. For Netatmo and Nedis, most of the packets are outbound bytes meaning that these devices send a lot more traffic than they receive. For Mill, inbound traffic was higher than outbound traffic. Also comparing the traffic from the three devices, as shown in Figure \ref{fig:ComparingBaselines}, shows that the traffic is different from the three devices. Netatmo has a more continuous line of the packets, while Mill sends traffic periodically and have more spikes and Nedis shows many spikes that are a lot higher than for the rest of the baseline period. These differences can be used to distinguish the traffic between the three air quality monitor from different vendors. 

The differences in level of inference from the three different air quality monitors are only distinguishable when it comes to Test Case 4: Weekends. For the routine behavioural test cases, neither of the devices expose private information regarding when the event is ongoing or what kind of event is triggered. However, the results from the weekend test shows that Nedis and Mill is a more secure choice than Netatmo when it comes to selecting air quality monitor to install in our home. Also when it comes to identifying changes in traffic, Nedis and Mill shows that the baseline traffic can change on other factors than a triggered event and it will therefore be harder for an attacker to understand normal traffic. It is good for users to know that one should consider more factors than just the appearance or functionality of the air quality monitor when selecting what device to install in their home. Knowing that the network traffic of an air quality monitor is not the same regardless of the manufacture, is important when selecting the right one. 

As only the weekend test for Netatmo exposed private information, only this specific test case can be used to evaluate how this information can be misused by an adversary. If an adversary has conducted a passive network eavesdropping attack against a user who has installed the Netatmo Smart Indoor Air Quality Monitor, it is possible to know if a user is gone for a longer period of time. Even though weekends were used in this test, the results will also be applicable for longer periods of time such as a week or several weeks. An adversary could use this private information to carry out malicious actions against the users home such as a burglary. Another aspect is that an attacker could learn about our habits for being home or not, such as every Easter or Christmas, the user is gone or every weekend during winter the user is not home. 

However, it is interesting to consider to what extent private information from the other test cases can be misused. Both cooking and showering are events where the users is awake and doing an active action in the home environment. If an adversary were to find out the routines of a user for these two cases, it may not be able to misuse it to the extent as the results from the weekend test can. Knowing that an unwanted party knows that every day at 7am you are taking a shower, may not feel that scary and invading, but if this is combined with attacks against other devices to find out your whole daily routine, it can feel like a bigger invasion of privacy. Window open at night can indicate a users sleeping patterns and therefore identifies a time period where the user is not awake in its home environment. This can feel like a more invasion of privacy as the user does not have as much control over the situation as when cooking or showering. On the other side, being able to know if a user is gone for a whole weekend is easier to misuse than just sleeping as the user is present at home.   

\section{Valuable contributions and challenges faced}
When starting this research, a decision to include several different air quality monitors from different manufactures were made. Looking at the results and evaluation, this was positive for the contribution as it shows that there are differences in level of private information inference from the different devices. This shows that analyzing the network traffic from one air quality monitor is not representative for all air quality monitors and is something to consider if setting up a test environment where only one air quality monitor is included.

Originally, only the routine test cases were selected, cooking, showering and window open. For the work of setting up the devices, sniffer and how to capture, traffic was captured over a longer period of time to also test the storage and processing power of the hardware used. During this initial setup, a discovery of the differences in traffic for Netatmo Smart Indoor Air Quality Monitor from when the user was gone or not were very visible. Therefore, the fourth test case was defined and tested on all the devices. 

When defining the scope and test cases, it was unclear how the devices communicated including packet sizes, amount of packets, how frequent and if traffic was encrypted or not. When starting the capturing and analyzing of data from the different devices, it became clear that the devices do not send the same amount of traffic. Netatmo Smart Indoor Air Quality Monitor sends the least amount of packets, while Mill Sense and Nedis SmartLife sends significantly more packets. For Nedis SmartLife, analyzing the amount of traffic was a challenge, both time-wise as creating graphs took a long time, but also looking more into the traffic. The processing power of a standard computer were just enough to process the data and figures shown in the evaluation and analysis result chapter. 

Many decisions and limitations had to be set for this research because of time constraints. Originally the plan was to test more air quality monitors from other manufactures, communicating on different communication protocols, such as Bluetooth or ZigBee. However, seeing the amount of work and results that needed to be analyzed to do this in a good way, only \gls{Wi-Fi} devices were selected. This part is therefore moved to future work.  

A challenge faced during the research happened for the test cases for cooking. The events were designed to change the indoor environment so much that the air quality monitors would sense values outside of the defined threshold values. However, when doing the cooking test, only a few of the executions actually triggered the notifications to be send. On the other side, for showering and window open, every execution lead to a significant change in threshold values and sent several notifications to the connected phone. 

Another challenge encountered during the test, was the differences in the baseline capturings compared to the event traffic for Nedis SmartLife and Mill Sense, before the event was triggered, as we would have expected these two traffic patterns to be similar. However, the decision were made to include the baseline capturings and compare them to the event traffic because it also shows methodologically the setup that was originally designed, and it is still possible to compare traffic from when the event was ongoing to traffic before and after the event. It is unclear why there are differences in standard traffic from the devices, but differences in season as the test cases have been conducted during January and the baseline during March can be one reason. Another reason can be that the baseline was not captured at the exact same spot as were the test cases were carried out. This was because of time constraints. It was not enough time to have a long baseline in each room and conduct the test cases during the research period even though this would have been preferred. Since the air quality monitors are \gls{IoT} sensor that are always on and sense the environment at the time, differences day to day and month to month can be significant. The indoor environment can be affected by many different factors and it is impossible to have the exact same values for the indoor air during longer period of times. 

The differences in standard traffic shows that in order to have the same traffic pattern to test with, the best way would be to have the air quality monitors in a closed environment. Then the environment could only be changed by the specific event which are tested. However, this takes away the realistic parts of the test. The chance of identifying the specific events may be bigger, but if it is not applicable in a live environment, there is no use to launch such an attack against a target. Another aspect of this is that if changes in the environment is significant within the same environment then it will also be hard for an attacker to find certain signatures in their own environment to be applicable on a target environment and device. 