\chapter*{Method}
This chapter describes the material used for testing, the method for testing, the environment setup and the test cases that has been conducted. 
\section*{Air Quality Monitor Survey}
\addcontentsline{toc}{section}{Airthings Quality Monitor Survey}
For selection of material, the problem description and research questions have been used as a reference. In order to answer the RQs the air quality monitors needs to be manufactured from different vendors and communicate over different communication protocols. To find the specific AQMs to use in this research, several online sources are used and compared. As this research is conducted on NTNU Gjøvik in Norway, it is also preferable that the devices are bought in Norwegian stores or online pages. The devices chosen should also be popular and easy accessible for any user. Considering these factors, the following points are made out to select the devices:
\begin{itemize}
    \item The devices should be manufactured from different vendors
    \item The devices should use different communication protocols
    \item The devices should preferably be available in Norwegian stores
    \item The devices should be popular and available for any user
\end{itemize}
Tibber \cite{Tibber} is a Norwegian power company that specializes in combining smart technology, as with IoT devices, and live app representaion of the power consumption and smart devices. \cite{Tibber} On their webpage it is possible to buy several IoT smart devices for your home. Here they recommend two different AQMs, Netatmo Healthy Home Coach and Aqara TVOC Air Quality Monitor. Netatmo communicates over WiFi, while Aqara communicates over Zigbee. 
\\\\
When searching for "Indoor Air Quality Monitor Tests" many of the online pages test several different air quality monitors, but many of them test different devices from the same manufacturer, Airthings. \cite{AQMTest1} \cite{AQMTest2} \cite{AQMTest3} Airthings \cite{Airthings} is a company that manufactures and sells smart radon and indoor air quality monitors. They offer a range of air quality monitors that can measure a variety of indoor air quality parameters such as radon, particulate matter, mold, CO2, humidity, temperature, VOCs, pressure and pollen. \cite{AirthingsProducts} To bring a wider variety of communication protocols to investigate in this research, the House Kit was chosen as traffic can be capture over Ethernet. It consists of the Wave Mini, the Wave Radon and a Hub that is connected with Ethernet. \cite{AirthingsProducts}
\\\\
Elkjøp \cite{Elkjøp} and Komplett \cite{Komplett} are two of Norways biggest electrical stores with a wide range of different smart home devices both in store and online. These are therefore a natural choice when users are looking for smart devices. When searching for "Air Quality Monitors" on their webpages, Elkjøps results are from 4 different vendors and Komplett from 8 different vendors. A part from the previous chosen devices, both Elkjøp and Komplett sells Mill Sense Air Quality monitors. Mill \cite{Mill} is a Norwegian company that manufactures and sells products for indoor climate and heating, with the goal of developing devices that fits the indoor interior environment. They offer one air quality monitor called Mill Sense. The Mill Sense air quality monitor communicates over WiFi and connects to the users app to provide insight on the indoor climate.
\\\\

\\\\\
\subsection*{Airthings Home Kit}
\addcontentsline{toc}{subsection}{Airthings Home Kit}

\subsection*{Netatmo Smart Indoor Air Quality Monitor}
\addcontentsline{toc}{subsection}{Netatmo Smart Indoor Air Quality Monitor}

\subsection*{Mill Sense Air Quality Monitor}
\addcontentsline{toc}{subsection}{Mill Sense Air Quality Monitor}

\subsection*{Nedis Smart Sensor}
\addcontentsline{toc}{subsection}{Nedis Smart Sensor}

\subsection*{Aqara TVOC Air Quality Monitor}
\addcontentsline{toc}{subsection}{Aqara TVOC Air Quality Monitor}

\section*{Environment Setup}
\addcontentsline{toc}{section}{Environment Setup}

\subsection*{How to setup the environment}
\addcontentsline{toc}{section}{How to setup the environment}

\section*{Test Cases}
\addcontentsline{toc}{section}{Test Cases}
\subsection*{How to design the test cases}
\addcontentsline{toc}{section}{How to design the test cases}
\subsection*{Introduction to each test case}
\addcontentsline{toc}{section}{Introduction to each test case}