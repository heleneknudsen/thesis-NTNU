\textbf{Title:}\hspace{2cm} Private Information inference in air quality monitors
\\\\
\textbf{Student:}\hspace{2cm} Helene Skjelsbæk Knudsen
\\\\\\
\textbf{Problem Description:}\\
Internet of Things are becoming more widespread and more available for any user. Devices that normally did not communicate with other devices, are being given functionality by adding sensors and technology. This functionality can be used to communicate with other devices, servers on the cloud or make their decisions. Producers of these devices are constantly adding functionality and new devices to keep up with the increasing demand from users wanting to create smart homes. Turning on the heat at your home from another location through an application on your mobile phone or having the doorlock send you notifications when it is opened or locked are examples of functionality that makes up the Internet of Things.
\\\\
Air Quality Monitors are a type of Internet of Things devices that monitors the indoor climate in any users home. The devices can communicate over different communication protocols to their apps or the vendors cloud storage. In addition, their functionality varies as they as equipped with different sensors giving the users different values of how good or bad their indoor climate is. The monitors can communicate with other devices and be integrated in a smart home environment. However, having devices monitoring your home 24/7 arises security issues. Private information about user behaviour can be attractive to different parties, from company's wanting to know how to target advertises to malicious actors that wants to misuse the knowledge of users habits. 
\\\\
This thesis will investigate several different Air Quality Monitors to see how and what private information can be gathered from carrying out a passive network eavesdropping attack. As the market contains a wide variety of air quality monitor devices, a few devices from different vendors should be chosen to carry out the attack and discovery what kind of private information in a user environment can be collected. 
\\
\ \\
\begin{flalign*}
     \\\textbf{Supervisor:}&& \text{Jia-Chun Lin, NTNU}
\end{flalign*}