\noindent
\textbf{Title:} \hfill Private Information Inference Based on Network Traffic Patterns of Air Quality Monitors
\\\\\\
\textbf{Student:} \hfill Helene Skjelsbæk Knudsen
\\\\\\\\
\\
\textbf{Problem Description:}\\
Internet of Things are becoming more widespread and available for any user. Home appliances that normally did not communicate with each other, are being given functionality by adding sensors and technology. This functionality can be used to communicate with other devices, servers in the cloud or carry out automated tasks based on own decisions. Manufactures of these devices are constantly adding functionality and new devices to keep up with the increasing demand from users wanting to create smart homes. Turning on the heat at your home from another location through an application on your mobile phone or having the doorlock send you notifications when it is opened or locked are examples of functionality that makes up the Internet of Things.

Air quality monitors are a type of Internet of Things devices that monitor the indoor climate in it's installed environment. The devices can communicate over different communication protocols to their application or the vendors cloud storage. In addition, their functionality varies as they as equipped with different sensors giving the users different values of how good or bad their indoor climate is. The monitors can communicate with other devices and be integrated in a smart home environment, such as fans or heaters. However, having devices monitoring our home 24/7 arises security issues. Private information about user behaviour can be attractive to different parties, from companies wanting to know how to target advertises to malicious actors that want to misuse the information. 

This thesis will investigate several different air quality monitors to see how and what private information can be gathered and inferred from carrying out a passive network eavesdropping attack. As the market contains a wide variety of air quality monitor devices, devices from different vendors should be chosen to carry out the attack and discover what kind of private information in a user environment can be collected. 

\ \\
\begin{flalign*}
     \\\textbf{Supervisor:}&& \text{Jia-Chun Lin, NTNU}
\end{flalign*}