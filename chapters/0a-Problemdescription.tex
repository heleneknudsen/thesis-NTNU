\textbf{Title:}\hspace{2cm} Private Information inference in air quality monitors
\\\\
\textbf{Student:}\hspace{2cm} Helene Skjelsbæk Knudsen
\\\\\\
\textbf{Problem Description:}\\
Internet of Things are increasing in popularity and types of smart devices. Both devices that we usually use everyday, but also devices with new functionality, that we did not know we needed, are being produced as smart devices. As a result of the increased demand of Internet of Things devices, several vendors have started producing these devices. The vaerity of devices from different vendors and functionality gives the users many options that can be hard to select from. 
\\\\
Air Quality Monitors are a type of Internet of Things devices that monitors the indoor climate in any users home. The devices communicates over different communication protocols to their apps or the vendors cloud storage. The look of these devices 

However, having a device monitoring us 24/7 arises some security issues regarding private information inference. 
\\
\\
\ \\
\begin{flalign*}
     \\\textbf{Date Approved:} && \text{07.10.2022} && \\\textbf{Responsible Professor:}&& \text{Lasse Øverlier, NTNU} &&
     \\\textbf{Supervisor:}&& \text{Jia-Chun Lin, NTNU}
\end{flalign*}