\textbf{Title:}\hspace{2cm} Private Information inference in air quality monitors
\\\\
\textbf{Student:}\hspace{2cm} Helene Skjelsbæk Knudsen
\\\\\\
\textbf{Problem Description:}\\
Internet of Things are becoming more widespread and more available for any user. Devices that normally did not communicate with other devices, are being given functionality by adding sensors and technology. This functionality can be used to communicate with other devices, servers on the cloud or make their decisions. Producers of these devices are constantly adding functionality and new devices to keep up with the increasing demand from users wanting to create smart homes. Turning on the heat at your home from another location through an application on your mobile phone or having the doorlock send you notifications when it is opened or locked are examples of functionality that makes up the Internet of Things.
\\\\
Air Quality Monitors are a type of Internet of Things devices that monitors the indoor climate in any users home. The devices communicates over different communication protocols to their apps or the vendors cloud storage. However, having a device monitoring us 24/7 arises some security issues regarding private information.
\\\\
This thesis will investigate several different Air Quality Monitors to see how and what private information can be gathered from conducting passive network eavesdropping attacks. The monitors should be communicating over different protocols and manufactured by different vendors. 
\\
\ \\
\begin{flalign*}
     \\\textbf{Supervisor:}&& \text{Jia-Chun Lin, NTNU}
\end{flalign*}