\chapter*{Introduction}
%Show that you are knowledgeable about previous research on the topic
%Introduce the readers to what they already know and what they do not know
%Present the problem or phenomenon you set out to study
%Present other research conducted within the same field
%Indicate gaps in information that you seek to fill out
%Present the research questions or hypotheses you intend to investigate
This chapter aims to give the reader a clear vision of the master thesis.  
\section*{Background and motivation}
\addcontentsline{toc}{section}{Background and motivation}
The Internet of Things (IoT) exists of a growing number of physical devices connected to the Internet to perform smart tasks. \cite{IoTSurveyAl-Fuqaha} Every-day devices can be equipped with smart functionality to improve our lives, but also to improve critical societal functions such as in health care or industrial technology. The devices range from a robot vacuum cleaner that users can control through their phone or cameras installed for elders that stream to a nurse. These smart devices can communicate and connect to each other and other services on the Internet and makes out an IoT system. The devices analyzes how users, machines or eco-systems behave and act accordingly. An emerging request for smart devices has resulted in an rapid growth in IoT devices worldwide. \cite{IoTAndPrivacy}
\\\\
We spend a lot of our lives inside, breathing in the air that is available in the indoor space. \cite{IndoorAirQualityMonitorIoT} The air affects our health and can potentially cause several chronic health problems, for example lung cancer or respiratory infections. \cite{IAQMonitorReview} Common air pollution's, such as smoke or car exhaust,  are easy to sense and avoid for people not trying to get effected by the dangerous particles they emit. However, with the evolution of new Internet of Things devices, indoor air quality monitors are increasing in popularity. \cite{SecurityAndDataIntInAQM} The devices are developing into becoming smaller, more affordable and appealing to include in your home environment while adopting several different sensors to report on the indoor air quality. 

\section*{Research Objectives and Research Questions}
\addcontentsline{toc}{section}{Research Objectives and Research Questions}

The following research questions (RQs) have been raised and will be answered through this master:
\begin{itemize}
    \item 
    \textbf{RQ1:} What kind of information can be gathered from air quality monitors when carrying out a network eavesdropping attack? And what kind of private information about the users and the environment can be inferred from the collected traffic?\\
    \item 
    \textbf{RQ2:} What are the differences in level of inference from different air quality monitors from different vendors?\\
    \item
    \textbf{RQ3:} What are the differences in level of inference from network eavesdropping air quality monitors communication on different protocols?\\
    \item 
    \textbf{RQ4:} How can the private information gathered be misused by an adversary?\\
\end{itemize}

\section*{Research Scope and Delimitation}
\addcontentsline{toc}{section}{Research Scope and Delimitation}
This research will present 5 different air quality monitors, all from different vendors. The air quality monitors communicates over the different protocols: Wi-Fi, Bluetooth, ZigBee and Ethernet and corresponding sniffers will be used to sniff traffic from the sensors. 

This thesis will not look into decrypting traffic if the air quality monitors encrypt the communication. The focus will be on conducting a passive network privacy inference attack and therefore only look at non-encrypted data whether it is the whole packet or only the header. 

\section*{Contribution}
\addcontentsline{toc}{section}{Contribution}

\section*{Thesis Structure}
\addcontentsline{toc}{section}{Thesis Structure}
This thesis is organized in the following structure:\\\\
\textbf{Background}
\\\\
\textbf{Related Work}
\\\\
\textbf{Method}
\\\\
\textbf{Evaluation and Analysis}
\\\\
\textbf{Discussion}
\\\\
\textbf{Conclusion}