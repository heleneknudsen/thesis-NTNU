\chapter{Introduction}
This chapter introduces the master thesis while presenting the background and motivation, followed by the research objectives and research questions that will be answered throughout this research. Scope and delimitation gives a clear understanding of what is included and not included in the thesis. The thesis structure outlines and gives a brief understanding of each of the following chapters in this thesis. The contribution that this thesis aim to give to research are presented in a separate subsection. 

\section{Background and motivation}
The Internet of Things (IoT) exists of a growing number of physical devices connected to the Internet to perform smart tasks \cite{IoTSurveyAl-Fuqaha}. Every-day devices can be equipped with smart functionality to improve our lives, but also to improve critical societal functions such as in health care or industrial technology. The devices can range from a robot vacuum cleaner that users can control through their phone or cameras installed for elders to stream to a nurse who resides centrally. These smart devices can communicate and connect to each other and other services using the Internet and makes out an IoT system. The devices analyzes how users, machines or eco-systems behave and act accordingly. An emerging request for smart devices has resulted in an rapid growth in IoT devices worldwide \cite{IoTAndPrivacy}. The devices are becoming more user friendly, smarter with added functionality and aesthetically more suitable to place or wear in any environment. 
\\\\
We spend a lot of our lives inside, breathing in the air that is available in the indoor space \cite{IndoorAirQualityMonitorIoT}. The air affects our health and can potentially cause several chronic health problems, for example lung cancer or respiratory infections \cite{IAQMonitorReview}. Common air pollution's, such as smoke or car exhaust, are easy to sense and avoid for people not trying to get effected by the dangerous particles they emit. It is also more wide-known that good outdoor air is beneficial for your health, not considering that the air indoor can also severely affect your health \cite{IndoorAirQuality}. Therefore, including the fact that Internet of Things devices are evolving, indoor air quality monitors are increasing in popularity and functionality \cite{SecurityAndDataIntInAQM}. The air quality monitors are also developing into becoming smaller, more affordable and appealing to include in your home environment while adopting several different sensors to report on the indoor air quality trying to become a more popular choice for users. 
\\\\
As users are installing these sensors inside their own homes and allow them to monitor their home environment every single day, they will be collecting data about the environment and changes or behaviour that affect the air quality monitor sensors. Therefore,it is interesting to look further into how easy it is to collect this data and infer what kind of user behaviour is happening in the environment. As harmless as a passive sensor, that is just collecting data about different indoor climate rates may seem, it is important to understand the risks one takes when installing these and connecting them to the Internet. Understanding what kind of private information is possible to infer from these devices and what makes the differences can be crucial when deciding which air quality monitor on the market to buy and install in your home.

\section{Research Objectives and Research Questions}
This thesis will conduct a network attack called passive network eavesdropping attack, that will be further explained in the background and method, and launch it against a group of individual air quality monitors residing in a home environment. In order to decided which devices to use and how to carry out an attack, a survey of the devices will be presented. A justification for which test cases to engage the sensors and devices is important when analyzing the results. When the network eavesdropping attack have been carried out and data  from the different test cases are collected, the results will be analyzed to see if and how much private information can be gathered from the different devices. The results will also look into if there are significant differences between the air quality monitors. Lastly, the research will investigate how the private information inferred can be used by malicious actors in a harmful way for users having the air quality monitors installed in their home.
\\\\
Based on the problem description, motivation and research objectives, the following research questions (RQs) have been raised and will be answered throughout the research of this master:
\begin{itemize}
    \item 
    \textbf{RQ1:} What kind of information can be gathered from air quality monitors when carrying out a network eavesdropping attack? And what kind of private information about the users and the environment can be inferred from the collected traffic?\\
    \item 
    \textbf{RQ2:} What are the differences in level of inference from different air quality monitors from different vendors?\\
    \item 
    \textbf{RQ3:} How can the private information gathered be misused by an adversary?\\
\end{itemize}

\section{Research Scope and Delimitation}
This research presents three different air quality monitors, all selected from different vendors. The air quality monitors communicates over the same protocol, Wi-Fi, but have different functionalities, applications and sensors. A Wi-Fi sniffer together with \textit{tshark} running on \textit{Kali Linux} will be used to sniff and store traffic from the sensors. To be able to answer the research questions, a baseline traffic pattern will be compared to traffic during triggered scheduled user events. The goal is to investigate what kind of private information, if any, it is possible to collect from conducting a passive network eavesdropping attack on the different devices and give an understanding of how this information can be misused if an adversary gains access. 
\\\\
This thesis will not look into decrypting traffic if the air quality monitors encrypt the communication. The focus will be on conducting a passive network privacy inference attack and therefore only look at non-encrypted data whether it is the whole packet or only the header. The research will also not look into different factors of how to do a successful attack, such as distance, materials of the building or power of the sniffer. The thesis will not cover all phases of a passive network eavesdropping attack, but a prerequisite of this is that the attacker has gained a strong enough wireless access to the users network through their Service Set Identifier (SSID) to be able to read traffic sent from and to the devices in the environment. It will not look into how to identify the IoT devices as they are already known with MAC address in this research. However, there are a numerous of researches that have looked into identification of several different IoT devices, such as \cite{IdentifyIoT1} and \cite{IdentifyingIoT2}. 

\section{Contribution}
This thesis contributes with research on different individual air quality monitors and what kind of private information that can be inferred from them. Even though there are other researches that have carried out a passive network eavesdropping attack trying to infer private information from IoT devices, a lot of these researches investigate on IoT devices that can clearly pose a threat to users privacy if inferred, such as cameras, watches or motion sensors. In a lot of researches found online, air quality monitors are often a part of an IoT environment and only one type of air quality monitor are included in the same environment, making it harder to compare the differences between different brands and sensors. Research on specifically air quality monitors are popular when it comes to their functionality and sensor and how good they can read the indoor environment, but looking into security and privacy of the devices, the research decreases significantly. Therefore, will this research contribute to not only looking into if a passive network eavesdropping attack can expose private information, but also if there are differences when multiple indoor air quality monitors are placed to observe the same environment and are exposed to the exact same test cases. 

\section{Thesis Structure}
This thesis is structured in the following chapters:\\\\
\textbf{Background}
\\
The background defines and describes important information for the reader to be able to understand the research and results. The background first presents general information about air quality monitors, how they work, how the sensors work and what factors play a part when selecting which air quality monitor to buy. The concepts of private information inference and passive network eavesdropping are explained as these are the method that will be used on the devices to answer the research questions. Lastly, Wi-Fi is explained and elaborated as this is the communication protocol that the devices communicate over and where the sniffing will take place. 
\\\\
\textbf{Related Work}
\\
The related work chapter aims to give the reader a selection of previous work and published research done by others on the topic. Former research on the security and privacy issues of specifically air quality monitors have the largest weight in this chapter. Both attacks, security frameworks and suggestions and privacy challenges are included in this chapter. As air quality monitors are a part of IoT devices, the other section of this chapter highlights research done on misusing of private information on IoT devices, where research done on IoT devices in general or a smart home or smart home care environment with several different IoT devices. 
\\\\
\textbf{Method}
\\
The method will describe in detail how this research has been carried out and give justification on the choices made. The survey for choosing the air quality monitors are presented and a more detailed description on the three devices chosen and how they work are presented. To ensure reputability, the environment setup are shown for both hardware and software components used to carry out the attack. Lastly, the test cases close up the chapter by giving justification on why they were specifically chosen and how they are designed and details on how they are carried out. 
\\\\
\textbf{Evaluation and Analysis}
\\
\\\\
\textbf{Discussion}
\\
%Fallgruver!! Hva kunne jeg gjort bedre?
\\\\
\textbf{Conclusion}
\\
The conclusion will conlude this research by answering the research question in short term and summing up this master thesis. 