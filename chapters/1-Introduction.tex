\chapter{Introduction}
This chapter introduces the master thesis while presenting the background and motivation, followed by the research objectives and research questions that will be answered throughout this research. Scope and delimitation gives a clear understanding of what is included and not in the thesis. The thesis structure outlines and gives a brief understanding of each of the following chapters in this thesis. The contributions are presented in a separate subsection. 

\section{Background and Motivation}
The \gls{IoT} exists of a growing number of physical and virtual devices connected to the Internet to perform smart tasks \cite{IoTSurveyAl-Fuqaha}. Every-day devices can be equipped with smart functionality to improve our lives, but also to improve critical societal functions such as in health care or industrial technology. The devices can range from a robot vacuum cleaner that users can control through their phone or cameras installed for elders to stream to a nurse who resides centrally. These smart devices can communicate and connect to each other and other services using the Internet and makes out an \gls{IoT} system. The devices analyzes how users, machines or eco-systems behave and act accordingly. An emerging request for smart devices has resulted in a rapid growth in \gls{IoT} devices worldwide \cite{IoTAndPrivacy}. The devices are becoming more user friendly, smarter with added functionality and aesthetically and more suitable to place or wear in any environment. 

We spend a lot of our lives inside, breathing in the air that is available in the indoor space \cite{IndoorAirQualityMonitorIoT}. The air affects our health and can potentially cause chronic health problems, for example lung cancer or respiratory infections \cite{IAQMonitorReview}. Common air pollution's, such as smoke or car exhaust, are easy to sense and avoid for people not trying to get effected by the dangerous particles they emit. It is also more wide-known that good outdoor air is beneficial for our health, not considering that the air indoor can also severely affect our health \cite{IndoorAirQuality}. Being more aware of our indoor environment and considering the fact that Internet of Things devices are evolving rapidly, indoor \gls{AQM}s are increasing in popularity and functionality \cite{SecurityAndDataIntInAQM}. The air quality monitors are also developing into becoming smaller, more affordable and appealing to include in our home environment while adopting several different sensors to report on the indoor air quality trying to become a more popular choice for users. 

As users are installing these sensors inside their own homes and allow them to monitor their home environments all day, the air quality monitors will be collecting data about the environment and behaviour that affect the air quality monitor sensors. Therefore, it is interesting to look further into how easy it is to collect this data and infer what kind of user behaviour is ongoing in the environment. As harmless as a passive sensor that is just collecting data about different indoor climate rates may seem, it is important to understand the risks one takes when installing these and connecting them to the Internet. Understanding what kind of private information is possible to infer from these devices and what makes the differences can be crucial when deciding which air quality monitor on the market to buy and install in our home.

\section{Research Objectives and Research Questions}
This thesis will conduct a network attack called \textit{passive network eavesdropping attack}, and launch it against a group of individual air quality monitors residing in a home environment. In order to decide which devices to use and how to carry out an attack, a survey of the devices will be presented. A justification for which test cases to trigger the sensors and devices is important when analyzing the results. When the passive network eavesdropping attack has been carried out and data  from the different test cases are collected, the results will be analyzed to see if and how much private information can be gathered from the different devices. The results will also look into if there are significant differences between the air quality monitors. Lastly, the research will investigate how the private information inferred can be used by malicious actors in a harmful way for users having the air quality monitors installed in their home.

Based on the problem description, motivation and research objectives, the following \gls{RQ} have been raised and will be answered throughout this master project:
\begin{itemize}
    \item 
    \textbf{\gls{RQ}1:} What kind of information can be gathered from air quality monitors when carrying out a passive network eavesdropping attack?\\
    \item 
    \textbf{\gls{RQ}2:} What are the differences in level of inference on different air quality monitors from different vendors?\\
    \item 
    \textbf{\gls{RQ}3:} How can the private information gathered be misused by an adversary?\\
\end{itemize}

\section{Research Scope and Delimitation}
This research presents three different air quality monitors, all selected from different vendors. The \gls{AQM}s communicates over the same protocol, \gls{Wi-Fi}, but have different functionalities, applications and sensors. A \gls{Wi-Fi} sniffer together with \textit{tshark} running on \textit{Kali Linux} will be used to capture and store wireless traffic. To answer the research questions, a baseline traffic pattern will be compared to traffic during triggered scheduled user events. The goal is to investigate what kind of private information it is possible to infer from conducting a passive network eavesdropping attack on the different devices. The analysis and evaluation will be based on network traffic patterns and give an understanding of how this information can be misused. 

This thesis will not look into decrypting traffic if the air quality monitors encrypt the communication. The focus will be on conducting a passive network privacy inference attack based on network traffic patterns and therefore only look at non-encrypted data whether it is the whole packet or only the header. The research will not look into different factors of how to do a successful attack, such as distance, materials of the building or signal strength of the sniffer. The thesis will not cover all phases of a passive network eavesdropping attack, but a prerequisite of this is that the attacker has gained a strong enough wireless access to the user's network and can read traffic sent from and to the devices in the environment. We will not look into how to identify the \gls{IoT} devices as they are already known with \gls{MAC} addresses in this research. But for readers information, there are other researches that have looked into identifying different \gls{IoT} devices, such as in \cite{IdentifyIoT1} and \cite{IdentifyingIoT2}. 

\section{Contributions}
This thesis contributes with research on different individual air quality monitors and what kind of private information that can be inferred from them. Even though there are other researches that have carried out a passive network eavesdropping attack trying to infer private information from \gls{IoT} devices, a lot of these researches investigate on \gls{IoT} devices that can clearly pose a threat to user's privacy if inferred, such as cameras, watches or motion sensors. In a lot of researches found online, air quality monitors are often a part of an \gls{IoT} environment and only one type of air quality monitor is included in the same environment, making it harder to compare the differences between different brands, manufactures, functionality and sensors. 

Research on specifically air quality monitors is popular when it comes to their functionality and sensor and how good they can read the indoor environment, but when looking into security and privacy of the devices, the research decreases significantly. Therefore, will this research contribute to not only looking into if a passive network eavesdropping attack can expose private information, but also if there are differences when multiple indoor air quality monitors are placed to observe the same environment and are exposed to the exact same test cases. 

\section{Thesis Structure}
This rest of this thesis is structured in the following chapters:\\\\
\textbf{Background}
\\
The background describes important information applicable for the research and results. General information about air quality monitors, how they work, how the sensors work and what factors to consider when selecting which air quality monitor is presented. The concepts of private information inference, passive network eavesdropping and \gls{Wi-Fi} are explained as these are relevant for the tests conducted. 
\\\\
\textbf{Related Work}
\\
The chapter presents a selection of previous research done by others on the problem topic. Security and privacy issues of specifically air quality monitors are highly weighted in this chapter. The chapter also presents research done on misusing of private information on other \gls{IoT} devices, including smart environments with several different \gls{IoT} devices. 
\\\\
\textbf{Method}
\\
The method describes how this research has been carried out. A survey for choosing specific air quality monitors are presented with a description of the devices. To ensure reproducibility, the environment setup is shown with all the components used. The test cases are presented by giving justification on why they were chosen, how they are designed and details on how they are carried out. A description on how the data captured are analyzed is also included in the method. 
\\\\
\textbf{Evaluation and Analysis Results}
\\
The chapter presents the results from the research to answers the research questions. A general procedure applicable for analyzing all tests is presented. Then a subsection with standard baseline traffic from the devices is presented. The results from the tests are presented in separate subsections.  
\\\\
\textbf{Discussion}
\\
This chapter answers the research questions defined in this thesis. It also discusses the work and results for the tests carried out. The challenges and mistakes and how this thesis could have been done differently will be brought to light, but also strengths and decision that were made to be able to discover the findings in the evaluation and analysis results. 
\\\\
\textbf{Conclusions and Future Work}
\\
This chapter concludes the research by answering the research question in short term and summing up this master thesis, while focusing on the contributions made. Included in this chapter is also future work suggesting how to further look into the research topic.