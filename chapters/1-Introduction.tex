\chapter*{Introduction}
%Show that you are knowledgeable about previous research on the topic
%Introduce the readers to what they already know and what they do not know
%Present the problem or phenomenon you set out to study
%Present other research conducted within the same field
%Indicate gaps in information that you seek to fill out
%Present the research questions or hypotheses you intend to investigate
This chapter sets out how the 
\section*{Background and motivation}
\addcontentsline{toc}{section}{Background and motivation}
%Introduction to IoT
The Internet of Things (IoT) exists of a growing number of physical devices connected to the Internet to perform smart tasks. \cite{IoTSurveyAl-Fuqaha} Every-day devices can be equipped with smart functionality to improve our lives, but also to improve critical societal functions such as in health care or industrial technology. The devices range from a robot vacuum cleaner that users can control through their phone or cameras installed for elders that stream to a nurse. These smart devices can communicate and connect to each other and other services on the Internet and makes out an IoT system. The devices analyzes how users, machines or eco-systems behave and act accordingly. An emerging request for smart devices has resulted in an rapid growth in IoT devices worldwide. %\cite{Statistics on this}
\\

\section*{Research Objectives and Research Questions}
\addcontentsline{toc}{section}{Research Objectives and Research Questions}

The following research questions (RQs) have been raised and will be answered through this master:
\begin{itemize}
    \item 
    \textbf{RQ1:} What kind of private information can be gathered from air quality monitors when carrying out a network eavesdropping attack?\\
    \item 
    \textbf{RQ2:} What are the differences in level of inference from different air quality monitors from different vendors?\\
    \item
    \textbf{RQ3:} What are the differences in level of inference from network eavesdropping air quality monitors communication on different protocols?\\
    \item 
    \textbf{RQ4:} How can the private information gathered be misused by an adversary?\\
\end{itemize}

\section*{Research Scope and Delimitation}
\addcontentsline{toc}{section}{Research Scope and Delimitation}
Will not look at decrypting information
If data is encrypted, I will only look at metadata

\section*{Contribution}
\addcontentsline{toc}{section}{Contribution}

\section*{Thesis Structure}
\addcontentsline{toc}{section}{Thesis Structure}
This thesis is organized in the following structure:\\
\textbf{Related Work}
\\
\textbf{Air Quality Monitor Survey}
\\
\textbf{Method}
\\
\textbf{Environment}
\\
\textbf{Test Cases}
\\
\textbf{Results}
\\
\textbf{Analysis of the result}
\\
\textbf{Conclusion}
\\
