\chapter*{Abstract}
The emerging use of \gls{IoT} devices in homes arises security concerns whether or not these devices expose private information about a user and its environment. Air quality monitors are a group of \gls{IoT} devices which are always on to sensor the indoor air quality of the environment. Even tough air quality monitors have been included in several tests within security, there are research gaps in comparing different devices to each other and conducting specific test cases on the air quality monitors. This master thesis carries out a passive network eavesdropping attack on three different air quality monitors to investigate if the network traffic pattern changes during an event triggered. The devices all communicates over \gls{Wi-Fi} and are manufactured from different vendors. The research begins with a baseline capture of the devices to learn how their traffic patterns are when events are not triggered. Then four different test cases were tested on the devices to see if it is possible to infer private information through the network patterns. The test cases are cooking, showering, window open during night and weekends at home or gone. Only one of the air quality monitors expose private information from one of the test cases. For this devices it is possible to infer whether a user is home or gone by looking at bytes sent and received and the differences in packet sizes for the two cases. This private information can be misused by an adversary to know when a user is home or not just by looking at the Wi-Fi traffic sent to and from that device. The main contribution of this master thesis is the collection and analysis of the different test cases for the three devices included and showing that one of the devices expose private information based on the network traffic patterns collected and analyzed during a passive network eavesdropping attack. This research also demonstrates that air quality monitors communicating over the same protocol, in this case \gls{Wi-Fi}, have different patterns and expose private information differently. It also motivates to continue the research on air quality monitors, as these are not as popular to test as other \gls{IoT} devices. 