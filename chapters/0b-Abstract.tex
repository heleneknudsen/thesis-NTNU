\chapter*{Abstract}
The emerging use of IoT devices in homes arises security concerns whether or not these devices expose private information about users and their environment. Air quality monitors are a type of IoT devices which are always on to monitor the indoor air quality of the environment. Even though air quality monitors have been included in several tests within security, there are research gaps in comparing different devices to each other and conducting test cases specifically designed to see if there are differences in the level of inference on the air quality monitors. This master thesis carries out a passive network eavesdropping attack on three different air quality monitors to investigate if the network traffic pattern changes during a triggered event and if these patterns can be used to infer user activities. The devices all communicate over Wi-Fi and are manufactured from different vendors. The research first presents with a baseline capture of the devices to learn their traffic patterns when events are not triggered. Then four different test cases were tested on the devices to see if it is possible to infer private information through looking at the corresponding network patterns. The test cases designed in this thesis are cooking, showering, window open during night, and weekends at home or gone. The results showed that only one of the air quality monitors expose private information from one of the test cases. For this device it is possible to infer whether a user is home or gone by looking at bytes sent and received and the differences in packet sizes. This private information can be misused by an adversary to know when a user is home or not just by looking at the Wi-Fi traffic sent to and from that device. This research demonstrates that air quality monitors communicating over the same protocol, in this case Wi-Fi, have different traffic patterns and expose private information differently. It also motivates to continue the research on air quality monitors, as these have not been as popular to test as other IoT devices. 