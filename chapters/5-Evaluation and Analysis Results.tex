\chapter{Evaluation and Analysis Results}
The evaluation and analysis result chapter presents all gathered and analyzed data from the four different tests carried out in this thesis. First, the baseline for each device are shown both separately and to compare to each other. For each event; cooking, showering, window open and weekend, several graphs and calculations are presented side by side to analyze. This is divided into categories with the three different devices. Then the actual events are compared to the baseline for each device to look further into differentiate the events from standard traffic. The results are commented and evaluated in the same sections that presents the results. 

\section{Baseline}
The capturing of baseline traffic was conducted over the course of 10 days in the hallway of the environment. During the baseline, the devices were not directly affected by the specific events, such as cooking, showering or window open in the same room as the devices resides. The baseline traffic will be used to look at standard traffic from the devices and to compare this to the events in both graphs and calculations in Sections 5.2-5.4. The traffic from the capture files were encrypted on layer 2 (\gls{Wi-Fi}) and therefore it is not possible to extract any values from the payload of the packets. This applies to all the devices. Since decryption of traffic is out of scope for this thesis, the results will only analyze patterns and no payload information. The filter used for all the baseline files is:

\begin{itemize}
\item frame.time >= "Mar 06, 2023 "00:00:00"" \&\& frame.time <= "Mar 15, 2023 "24:00:00""
\end{itemize}

\subsection{Netatmo Baseline}
Figures \ref{fig:NetatmoBaselineTotalPackets} and \ref{fig:NetatmoBaselineTotalBytes} show the graphs for Netatmo from the baseline capturing from 6th of March 2023 to 15th of March 2023. For the baseline graphs, it is possible to see that packets are sent continually at a rate of around 250 bytes per 2 seconds and 2 packets per 2 seconds. In addition to the continuously stream of packets, the traffic flow also has spikes that looks random in both time and size. As these graphs shows the total packets and bytes sent and received, it can also be beneficial to look at what the graphs would look like if filtered on packets and bytes sent and packets and bytes received separately. 
\begin{figure} [H]
    \centering
    \includegraphics[scale=0.25]{figures/Netatmo_Baseline_TotalBytes.png}
    \caption{Netatmo baseline capture with total number of bytes as the y-axis}
    \label{fig:NetatmoBaselineTotalBytes}
\end{figure}

\begin{figure} [H]    
    \centering
    \includegraphics[scale=0.25]{figures/Netatmo_Baseline_TotalPackets.png}
    \caption{Netatmo baseline capture with total number of packets as the y-axis}
    \label{fig:NetatmoBaselineTotalPackets}
 \end{figure}

 \begin{figure}[H]
    \centering
    \begin{subfigure}[b]{0.4\textwidth}
        \includegraphics[width=\textwidth]{figures/Netatmo_Baseline_InboundBytes.png}
        \caption{Netatmo inbound bytes}
    \end{subfigure}
    \begin{subfigure}[b]{0.4\textwidth}
        \includegraphics[width=\textwidth]{figures/Netatmo_Baseline_OutboundBytes.png}
        \caption{Netatmo outbound bytes}
    \end{subfigure}
    \begin{subfigure}[b]{0.4\textwidth}
        \includegraphics[width=\textwidth]{figures/Netatmo_Baseline_InboundPackets.png}
        \caption{Netatmo inbound packets}
    \end{subfigure}
    \begin{subfigure}[b]{0.4\textwidth}
        \includegraphics[width=\textwidth]{figures/Netatmo_Baseline_OutboundPackets.png}
        \caption{Netatmo outbound packets}
    \end{subfigure}
    \caption{Netatmo baseline inbound and outbound bytes}
    \label{Fig:NetatmoBaselineOutandInboundTraffic}
 \end{figure}

For the inbound and outbound bytes and packets for Netatmo it is clear to see from Figure \ref{Fig:NetatmoBaselineOutandInboundTraffic} that the device sends a lot more than it receives. Calculations made on the baseline traffic are presented in Table \ref{tab:NetatmoBaselineCalculations}. 

\begin{table}[H]
    \caption{Calculations for Netatmo baseline capture}
    \centering
    \begin{tabular}{ll|l|}
        \cline{3-3}                                               &                               &             \textbf{Numbers} \\ \hline
        \multicolumn{1}{|c|}{\multirow{4}{*}{\textbf{Total}}}    & Packets              & 110,735         \\ \cline{2-3} 
        \multicolumn{1}{|c|}{}                                   & Bytes                & 14,959,396       \\ \cline{2-3} 
        \multicolumn{1}{|c|}{}                                   & Average bytes/second & 17               \\ \cline{2-3} 
        \multicolumn{1}{|c|}{}                                   & Average packet size  & 135 bytes        \\ \hline
        \multicolumn{1}{|l|}{\multirow{5}{*}{\textbf{Inbound}}}  & Packets              & 1,042            \\ \cline{2-3} 
        \multicolumn{1}{|l|}{}                                   & Bytes                & 83,446           \\ \cline{2-3} 
        \multicolumn{1}{|l|}{}                                   & Average bytes/second & 0                \\ \cline{2-3} 
        \multicolumn{1}{|l|}{}                                   & Average packet size  & 80 bytes          \\ \cline{2-3} 
        \multicolumn{1}{|l|}{}                                   & Biggest packet       & 377 bytes        \\ \hline
        \multicolumn{1}{|l|}{\multirow{5}{*}{\textbf{Outbound}}} & Packets              & 109,693          \\ \cline{2-3} 
        \multicolumn{1}{|l|}{}                                   & Bytes                & 14,875,950       \\ \cline{2-3} 
        \multicolumn{1}{|l|}{}                                   & Average bytes/second & 17               \\ \cline{2-3} 
        \multicolumn{1}{|l|}{}                                   & Average packet size  & 136 bytes         \\ \cline{2-3} 
        \multicolumn{1}{|l|}{}                                   & Biggest packet       & 1,150 bytes       \\ \hline
    \end{tabular}
    \label{tab:NetatmoBaselineCalculations}
\end{table}
Comparing the inbound and outbound graphs in Figure \ref{Fig:NetatmoBaselineOutandInboundTraffic} with the total graphs, in Figure \ref{fig:NetatmoBaselineTotalBytes} and Figure \ref{fig:NetatmoBaselineTotalPackets}, shows that the outbound graphs stands for the majority of the packets and are very similar to the total graphs. The same is numerically shown in Table \ref{tab:NetatmoBaselineCalculations}, where over 99\% of the total packets are outbound traffic. Therefore, it will be better to only display the events with graphs for total traffic to evaluate and analyze for the rest of the thesis. 

\subsection{Mill Baseline}
Figures \ref{fig:MillBaselineTotalPackets} and \ref{fig:MillBaselineTotalBytes} show the graphs for Mill from the baseline capturing from 6th of March 2023 to 15th of March 2023. The baseline traffic for Mill shows that the traffic varies a lot. As this device does not send live updates, but every minute, more spikes are included as it does not always send packets. It is also possible to see the spikes more clearly if the time range is smaller, this will be visible further when looking at the event and baseline comparison graphs for each event. 

Graphs for inbound and outbound traffic have also been made for Mill to see the differences for the packets sent. Figure \ref{Fig:MillBaselineOutandInboundTraffic} displays the different graphs for each of the traffic directions. Numerical calculations for the baseline traffic are presented in Table \ref{tab:MillBaselineCalculations}. 
\begin{figure} [H]
    \centering
    \includegraphics[scale=0.25]{figures/Mill_Baseline_TotalBytes.png}
    \caption{Mill baseline capture with total number of bytes as the y-axis}
    \label{fig:MillBaselineTotalBytes}
\end{figure}

\begin{figure} [H]
    \centering
    \includegraphics[scale=0.25]{figures/Mill_Baseline_TotalPackets.png}
    \caption{Mill baseline capture with total number of packets as the y-axis}
    \label{fig:MillBaselineTotalPackets}
 \end{figure}

\begin{figure}[H]
    \centering
    \begin{subfigure}[b]{0.4\textwidth}
        \includegraphics[width=\textwidth]{figures/Mill_Baseline_InboundBytes.png}
        \caption{Mill inbound bytes}
    \end{subfigure}
    \begin{subfigure}[b]{0.4\textwidth}
        \includegraphics[width=\textwidth]{figures/Mill_Baseline_OutboundBytes.png}
        \caption{Mill outbound bytes}
    \end{subfigure}
    \begin{subfigure}[b]{0.4\textwidth}
        \includegraphics[width=\textwidth]{figures/Mill_Baseline_InboundPackets.png}
        \caption{Mill inbound packets}
    \end{subfigure}
    \begin{subfigure}[b]{0.4\textwidth}
        \includegraphics[width=\textwidth]{figures/Mill_Baseline_OutboundPackets.png}
        \caption{Mill outbound packets}
    \end{subfigure}
    \caption{Mill baseline inbound and outbound bytes}
    \label{Fig:MillBaselineOutandInboundTraffic}
 \end{figure}

\begin{table}[H]
    \caption{Calculations for Mill baseline capture}
    \centering
    \begin{tabular}{ll|l|}
        \cline{3-3}                                              &                      & \textbf{Numbers} \\ \hline
        \multicolumn{1}{|c|}{\multirow{4}{*}{\textbf{Total}}}    & Packets              & 1,236,753       \\ \cline{2-3} 
        \multicolumn{1}{|c|}{}                                   & Bytes                & 129,253,290     \\ \cline{2-3} 
        \multicolumn{1}{|c|}{}                                   & Average bytes/second & 149             \\ \cline{2-3} 
        \multicolumn{1}{|c|}{}                                   & Average packet size  & 105 bytes       \\ \hline
        \multicolumn{1}{|l|}{\multirow{5}{*}{\textbf{Inbound}}}  & Packets              & 942,112         \\ \cline{2-3} 
        \multicolumn{1}{|l|}{}                                   & Bytes                & 95,458,773      \\ \cline{2-3} 
        \multicolumn{1}{|l|}{}                                   & Average bytes/second & 110             \\ \cline{2-3} 
        \multicolumn{1}{|l|}{}                                   & Average packet size  & 101 bytes       \\ \cline{2-3} 
        \multicolumn{1}{|l|}{}                                   & Biggest packet       & 1593 bytes      \\ \hline
        \multicolumn{1}{|l|}{\multirow{5}{*}{\textbf{Outbound}}} & Packets              & 294,640         \\ \cline{2-3} 
        \multicolumn{1}{|l|}{}                                   & Bytes                & 33,794,517      \\ \cline{2-3} 
        \multicolumn{1}{|l|}{}                                   & Average bytes/second & 39              \\ \cline{2-3} 
        \multicolumn{1}{|l|}{}                                   & Average packet size  & 115 bytes       \\ \cline{2-3} 
        \multicolumn{1}{|l|}{}                                   & Biggest packet       & 456 bytes       \\ \hline
    \end{tabular}
    \label{tab:MillBaselineCalculations}
\end{table}

As Figure \ref{Fig:MillBaselineOutandInboundTraffic} shows, the device receives a lot more packets and bytes than it sends. As the inbound graphs do not differ much from the total graphs, it will be best to proceed with the analysis in a total traffic aspect where both inbound and outbound traffic are included. This is also reflected in Table \ref{tab:MillBaselineCalculations}, which shows that 76\% of packets and 74\% of bytes are inbound traffic.

\subsection{Nedis Baseline}
Figures \ref{fig:NedisBaselineTotalPackets} and \ref{fig:NedisBaselineTotalBytes} show the graphs for Nedis from the baseline capturing from 6th of March 2023 to 15th of March 2023. The traffic sent and received by Nedis is characterized by varying a lot with high spikes especially for packets. The larger spikes, which are more visible in Figure \ref{fig:NedisBaselineTotalPackets}, showing the packets, occurs almost everyday. Since traffic is encrypted, it is not possible to see what these spikes are, but for further analysis it is important to understand that normal traffic for the device, can be large spikes occurring around the same time each night around 3am. 
\begin{figure} [H]
    \centering
    \includegraphics[scale=0.25]{figures/Nedis_Baseline_TotalBytes.png}
    \caption{Nedis baseline capture with total number of bytes as the y-axis}
    \label{fig:NedisBaselineTotalBytes}
\end{figure}

\begin{figure} [H]
    \centering
    \includegraphics[scale=0.25]{figures/Nedis_Baseline_TotalPackets.png}
    \caption{Nedis baseline capture with total number of packets as the y-axis}
    \label{fig:NedisBaselineTotalPackets}
 \end{figure}

\begin{figure}[H]
    \centering
    \begin{subfigure}[b]{0.4\textwidth}
        \includegraphics[width=\textwidth]{figures/Nedis_Baseline_InboundBytes.png}
        \caption{Nedis inbound bytes}
    \end{subfigure}
    \begin{subfigure}[b]{0.4\textwidth}
        \includegraphics[width=\textwidth]{figures/Nedis_Baseline_OutboundBytes.png}
        \caption{Nedis outbound bytes}
    \end{subfigure}
    \begin{subfigure}[b]{0.4\textwidth}
        \includegraphics[width=\textwidth]{figures/Nedis_Baseline_InboundPackets.png}
        \caption{Nedis inbound packets}
    \end{subfigure}
    \begin{subfigure}[b]{0.4\textwidth}
        \includegraphics[width=\textwidth]{figures/Nedis_Baseline_OutboundPackets.png}
        \caption{Nedis outbound packets}
    \end{subfigure}
    \caption{Nedis baseline inbound and outbound bytes}
    \label{Fig:NedisBaselineOutandInboundTraffic}
 \end{figure}
 
\begin{table}[H]
    \caption{Calculations for Nedis baseline capture}
    \centering
    \begin{tabular}{ll|l|}
        \cline{3-3}                                               &                               &             \textbf{Numbers} \\ \hline
        \multicolumn{1}{|c|}{\multirow{4}{*}{\textbf{Total}}}    & Packets              & 2,428,701         \\ \cline{2-3} 
        \multicolumn{1}{|c|}{}                                   & Bytes                & 295,022,494       \\ \cline{2-3} 
        \multicolumn{1}{|c|}{}                                   & Average bytes/second & 341               \\ \cline{2-3} 
        \multicolumn{1}{|c|}{}                                   & Average packet size  & 121 bytes        \\ \hline
        \multicolumn{1}{|l|}{\multirow{5}{*}{\textbf{Inbound}}}  & Packets              & 451,495           \\ \cline{2-3} 
        \multicolumn{1}{|l|}{}                                   & Bytes                & 88,595,049        \\ \cline{2-3} 
        \multicolumn{1}{|l|}{}                                   & Average bytes/second & 102                \\ \cline{2-3} 
        \multicolumn{1}{|l|}{}                                   & Average packet size  & 196 bytes         \\ \cline{2-3} 
        \multicolumn{1}{|l|}{}                                   & Biggest packet       & 522 bytes        \\ \hline
        \multicolumn{1}{|l|}{\multirow{5}{*}{\textbf{Outbound}}} & Packets              & 1,977,206         \\ \cline{2-3} 
        \multicolumn{1}{|l|}{}                                   & Bytes                & 206,427,445      \\ \cline{2-3} 
        \multicolumn{1}{|l|}{}                                   & Average bytes/second & 238               \\ \cline{2-3} 
        \multicolumn{1}{|l|}{}                                   & Average packet size  & 104 bytes         \\ \cline{2-3} 
        \multicolumn{1}{|l|}{}                                   & Biggest packet       & 485 bytes       \\ \hline
    \end{tabular}
    \label{tab:NedisBaselineCalculations}
\end{table} 

Figure \ref{Fig:NedisBaselineOutandInboundTraffic} shows differences in inbound and outbound traffic for Nedis. Figure \ref{Fig:NedisBaselineOutandInboundTraffic} shows that the spikes are packets which the device receives. Even though the graphs for inbound and outbound traffic from Nedis can look similar, the calculations presented in Table \ref{tab:NedisBaselineCalculations} shows that 81\% of the packets and 70\% of the bytes in the baseline are traffic sent from the device. Therefore, looking at the graphs in total will therefore give the most to analyze. Another aspect of looking at the graphs in total, compared to inbound and outbound separately is that since the traffic is encrypted on the layer 2, it is not possible to know what makes the possible changes. 

\subsection{Baseline Summary and Comparison Between the Devices}
This subsection summarizes and compares the baseline traffic for all the three devices. Table \ref{tab:ComparingBaselineCalculations} shows the differences in packets and bytes sent and received to each devices during the baseline capture. As the table shows, the three devices send different amounts of packets and bytes. Nedis sends the most while Netatmo sends the least amount in standby. For inbound and outbound traffic, Netatmo and Nedis sends more packets than it receives, while Mill receives more packets than it sends. The same is shown in Figure \ref{fig:ComparingBaselines} for the total number of bytes and packets, in Figure \ref{Fig:CompareBaselineOutandInboundBytes} for inbound traffic and in Figure \ref{Fig:CompareBaselineOutandInboundPackets} for outbound traffic.

\begin{table}[H]
    \centering
    \caption{Baseline capture summary for all the devices}
    \begin{tabular}{ll|l|l|l|}
    \cline{3-5}
                                                         &         & \textbf{Netatmo} & \textbf{Mill} & \textbf{Nedis} \\ \hline
        \multicolumn{1}{|l|}{\multirow{2}{*}{\textbf{Total}}}    & Packets & 110,735          & 1,236,753     & 2,428,701      \\ \cline{2-5} 
        \multicolumn{1}{|l|}{}                                   & Bytes   & 14,959,396       & 129,253,290   & 295,022,494    \\ \hline
        \multicolumn{1}{|l|}{\multirow{2}{*}{\textbf{Inbound}}}  & Packets & 1,042            & 942,112       & 451,495        \\ \cline{2-5} 
        \multicolumn{1}{|l|}{}                                   & Bytes   & 83,446           & 95,458,773    & 88,595,049     \\ \hline
        \multicolumn{1}{|l|}{\multirow{2}{*}{\textbf{Outbound}}} & Packets & 109,693          & 294,640       & 1,977,206      \\ \cline{2-5} 
        \multicolumn{1}{|l|}{}                                   & Bytes   & 14,875,950       & 33,794,517    & 206,427,445    \\ \hline
    \end{tabular}
    \label{tab:ComparingBaselineCalculations}
\end{table}
 
\begin{figure}[H]
    \centering
    \begin{subfigure}[b]{0.4\textwidth}
        \centering
        \includegraphics[width=1\hsize]{figures/Netatmo_Baseline_TotalPackets.png}
        \caption{Netatmo packets}
    \end{subfigure}
    \begin{subfigure}[b]{0.4\textwidth}
        \centering
        \includegraphics[width=1\hsize]{figures/Netatmo_Baseline_TotalBytes.png}
        \caption{Netatmo bytes}
    \end{subfigure}
    \begin{subfigure}[b]{0.4\textwidth}
        \centering
        \includegraphics[width=1\hsize]{figures/Mill_Baseline_TotalPackets.png}
        \caption{Mill packets}
    \end{subfigure}
    \begin{subfigure}[b]{0.4\textwidth}
        \centering
        \includegraphics[width=1\hsize]{figures/Mill_Baseline_TotalBytes.png}
        \caption{Mill bytes}
    \end{subfigure}
    \begin{subfigure}[b]{0.4\textwidth}
        \centering
        \includegraphics[width=1\hsize]{figures/Nedis_Baseline_TotalPackets.png}
        \caption{Nedis packets}
    \end{subfigure}
    \begin{subfigure}[b]{0.4\textwidth}
        \centering
        \includegraphics[width=1\hsize]{figures/Nedis_Baseline_TotalBytes.png}
        \caption{Nedis bytes}
    \end{subfigure}
    \caption{Traffic comparison between the baseline graphs with total number of packets and bytes for all devices}
    \label{fig:ComparingBaselines}
\end{figure}

\begin{figure}[H]
    \centering
    \begin{subfigure}[b]{0.4\textwidth}
        \includegraphics[width=\textwidth]{figures/Netatmo_Baseline_InboundBytes.png}
        \caption{Netatmo inbound bytes}
    \end{subfigure}
    \begin{subfigure}[b]{0.4\textwidth}
        \includegraphics[width=\textwidth]{figures/Netatmo_Baseline_OutboundBytes.png}
        \caption{Netatmo outbound bytes}
    \end{subfigure}
    \begin{subfigure}[b]{0.4\textwidth}
        \includegraphics[width=\textwidth]{figures/Mill_Baseline_InboundBytes.png}
        \caption{Mill inbound bytes}
    \end{subfigure}
    \begin{subfigure}[b]{0.4\textwidth}
        \includegraphics[width=\textwidth]{figures/Mill_Baseline_OutboundBytes.png}
        \caption{Mill outbound bytes}
    \end{subfigure}
    \begin{subfigure}[b]{0.4\textwidth}
        \includegraphics[width=\textwidth]{figures/Nedis_Baseline_InboundBytes.png}
        \caption{Nedis inbound bytes}
    \end{subfigure}
    \begin{subfigure}[b]{0.4\textwidth}
        \includegraphics[width=\textwidth]{figures/Nedis_Baseline_OutboundBytes.png}
        \caption{Nedis outbound bytes}
    \end{subfigure}
    \caption{Inbound and outbound baseline bytes comparison for all devices}
    \label{Fig:CompareBaselineOutandInboundBytes}
 \end{figure}

\begin{figure}[H]
    \centering
    \begin{subfigure}[b]{0.4\textwidth}
        \includegraphics[width=\textwidth]{figures/Netatmo_Baseline_InboundPackets.png}
        \caption{Netatmo inbound packets}
    \end{subfigure}
    \begin{subfigure}[b]{0.4\textwidth}
        \includegraphics[width=\textwidth]{figures/Netatmo_Baseline_OutboundPackets.png}
        \caption{Netatmo outbound packets}
    \end{subfigure}
    \begin{subfigure}[b]{0.4\textwidth}
        \includegraphics[width=\textwidth]{figures/Mill_Baseline_InboundPackets.png}
        \caption{Mill inbound packets}
    \end{subfigure}
    \begin{subfigure}[b]{0.4\textwidth}
        \includegraphics[width=\textwidth]{figures/Mill_Baseline_outboundPackets.png}
        \caption{Mill outbound packets}
    \end{subfigure}
    \begin{subfigure}[b]{0.4\textwidth}
        \includegraphics[width=\textwidth]{figures/Nedis_Baseline_InboundPackets.png}
        \caption{Nedis inbound Packets}
    \end{subfigure}
    \begin{subfigure}[b]{0.4\textwidth}
        \includegraphics[width=\textwidth]{figures/Nedis_Baseline_OutboundPackets.png}
        \caption{Nedis outbound packets}
    \end{subfigure}
    \caption{Inbound and outbound baseline packets comparison for all devices}
    \label{Fig:CompareBaselineOutandInboundPackets}
 \end{figure}

\newpage
\section{Test Case 1: Cooking}
This chapter presents the results and analysis conducted on Test Case 1: Cooking. The first subsection will present general information applicable to all the devices, and the following subsections will present the result and analysis for each of the devices separately. 
\subsection{General}
The cooking events are 10 in total and presented in Table \ref{tab:CookingDates}. Every device have the same time and dates for this event. 
\begin{table}[H]
    \centering
    \caption{Date and time for Test Case 1: Cooking}
    \begin{adjustbox}{width=1\textwidth}
            \begin{tabular}{l|l|l|l|l|l|l|l|l|l|l|}
            \cline{2-11} & 08.01 & 09.01 & 11.01 & 16.01 & 18.01 & 19.01 & 25.01 & 30.01 & 31.01 & 01.02 \\
            \hline
            \multicolumn{1}{|l|}{Started cooking}  & 15:58 & 15:59 & 16:05 & 16:02 & 16:04 & 16:01 & 16:02 & 16:01 & 16:01 & 16:02 \\ 
            \hline
            \multicolumn{1}{|l|}{Finished cooking} & 16:22 & 16:21 & 16:37 & 16:25 & 16:25 & 16:18 & 16:13 & 16:19 & 16:21 & 16:22 \\ \hline
            \end{tabular}
    \end{adjustbox}
    \label{tab:CookingDates}
\end{table}
\FloatBarrier

To be able to look even further into if it is a similar traffic pattern to each event that can be used to identify it, the same start and finish time have been used for every graph. To have the same amount of time on each event, the earliest start time and the latest finish time are used as filtering values for each of the \gls{pcaps} for the cooking event. 30 minutes before and after these times were used as the start and finish time for the capture files, while the actual event time given by Table \ref{tab:CookingDates} is marked red on the graphs. This gives the following values to use for further analysis:

\begin{itemize}
    \item Earliest cooking start: 15:58
    \item Latest cooking finished: 16:37
    \item Packet capture files start: 15:28
    \item Packet capture files end: 17:07
\end{itemize}

These timings give the following filter added to create the pcaps for each event:

\begin{itemize}
    \item frame.time >= "Month Date, Year 15:28:00" \&\& frame.time <= "Month Date, Year 17:07:00"
\end{itemize}

\newpage
\subsection{Netatmo}
Table \ref{tab:NetatmoCookingCalculations} presents calculations from all the cooking events and Table \ref{tab:NetatmoBaselineCookingCalculations} presents the calculations from all the corresponding baseline pcaps. Table \ref{tab:NetatmoComparingBaselineAndCookingCalculations} compares the average and standard deviation values from the events to the baseline. Figure \ref{fig:NetatmoCookingCalculations} presents a graphical overview of the packets and bytes from Table \ref{tab:NetatmoCookingCalculations} including average values.

\begin{table}[H]
    \centering
    \caption{Calculations on cooking events for Netatmo}
    \begin{tabular}{|l|l|l|l|l|l|}
    \hline
        \textbf{Dates} & \textbf{Packets} & \textbf{Bytes} & \textbf{Biggest packet} \\ \hline
        08.jan & 901 & 123,630 & 407 bytes\\ \hline
        09.jan & 703 & 97,019 & 407 bytes \\ \hline
        11.jan & 847 & 114,473 & 407 bytes\\ \hline
        16.jan & 1,082 & 146,001 & 407 bytes\\ \hline
        18.jan & 948 & 129,907 & 407 bytes\\ \hline
        19.jan & 828 & 111,629 & 407 bytes \\ \hline
        25.jan & 430 & 58,926 & 407 bytes \\ \hline
        30.jan & 815 & 110,838 & 407 bytes \\ \hline
        31.jan & 864 & 115,302 & 136 bytes \\ \hline
        01.feb & 805 & 109,050 & 407 bytes \\ \hline
    \end{tabular}
    \label{tab:NetatmoCookingCalculations}
\end{table}

\begin{table}[H]
    \centering
    \caption{Calculations on comparing baseline files for the cooking event for Netatmo}
    \begin{tabular}{|l|l|l|l|}
    \hline
        \textbf{Baseline} & \textbf{Packets} & \textbf{Bytes} & \textbf{Biggest packet} \\ \hline
        06.mar & 584 & 77,780 & 134 bytes\\ \hline
        07.mar & 972 & 132,406 & 407 bytes\\ \hline
        08.mar & 697 & 94,730 & 407 bytes \\ \hline
        09.mar & 868 & 117,020 & 407 bytes \\ \hline
        10.mar & 825 & 111,863 & 407 bytes \\ \hline
        11.mar & 745 & 101,534 & 407 bytes \\ \hline
        12.mar & 626 & 84,987 & 407 bytes \\ \hline
        13.mar & 764 & 101,772 & 136 bytes \\ \hline
        14.mar & 750 & 102,396 & 407 bytes \\ \hline
        15.mar & 812 & 108,703 & 407 bytes \\ \hline
    \end{tabular}
    \label{tab:NetatmoBaselineCookingCalculations}
\end{table}

\begin{table}[H]
    \centering
    \caption{Traffic comparison between the cooking event and baseline for Netatmo}
    \begin{tabular}{c|l|l|l|l|}
        \cline{2-5}
        \multicolumn{1}{l|}{}                                              & \textbf{Type} & \textbf{Packets} & \textbf{Bytes} & \textbf{Biggest packet} \\ \hline
        \multicolumn{1}{|c|}{\multirow{2}{*}{\textbf{Average}}}            & Cooking         & 822              & 111,678        & 380 bytes               \\ \cline{2-5} 
        \multicolumn{1}{|c|}{}                                             & Baseline      & 764              & 103,319        & 353 bytes                \\ \hline
        \multicolumn{1}{|c|}{\multirow{2}{*}{\textbf{Standard deviation}}} & Cooking         & 170              & 22,802         & 86 bytes                 \\ \cline{2-5} 
        \multicolumn{1}{|c|}{}                                             & Baseline      & 114              & 15,650         & 115 bytes               \\ \hline          
    \end{tabular}
    \label{tab:NetatmoComparingBaselineAndCookingCalculations}
\end{table}

\begin{figure}[H]
    \centering
    \begin{subfigure}{0.8\textwidth}
       \centering
       \includegraphics[width=1\hsize]{figures/Netatmo_Cooking_Calculations_Packets.png} 
    \end{subfigure}
    \begin{subfigure}{0.8\textwidth}
        \centering
        \includegraphics[width=1\hsize]{figures/Netatmo_Cooking_Calculations_Bytes.png} 
    \end{subfigure}
    \caption{Graphical presentation of event and baseline cooking calculations with packets and bytes, including average value extracted from Table \ref{tab:NetatmoComparingBaselineAndCookingCalculations} for Netatmo}
    \label{fig:NetatmoCookingCalculations}
\end{figure}

The graphs in Figures \ref{fig:NetatmoCookingPackets1}, \ref{fig:NetatmoCookingPackets2}, \ref{fig:NetatmoCookingBytes1} and \ref{fig:NetatmoCookingBytes2} display both bytes and packets for the cooking events in comparison with the baseline captures. The event graphs are placed on the left side of the figure and are framed in red, while the baseline graphs are placed on the right side of the figure and are framed in blue. The area marked red on the event graphs is when the event was ongoing, and not included in the baseline graphs as no event was ongoing and is only used for comparison. The x- and y-axis for all the graphs in the same figure have the same minimum and maximum values. 

\begin{figure}[H]
    \begin{subfigure}[b]{0.47\textwidth}
        \centering
        \tcbincludegraphics[size=fbox,width=1.1\hsize,colframe=red]{figures/Netatmo_Cooking_Packets_08.01.png}
    \end{subfigure}
    \begin{subfigure}[b]{0.47\textwidth}
        \centering
        \tcbincludegraphics[size=fbox,width=1.1\hsize, colframe=blue]{figures/Netatmo_Cooking_Baseline_Packets_06.03.png}
    \end{subfigure}
    \begin{subfigure}[b]{0.47\textwidth}
        \centering
        \tcbincludegraphics[size=fbox,width=1.1\hsize,colframe=red]{figures/Netatmo_Cooking_Packets_09.01.png}
    \end{subfigure}
    \begin{subfigure}[b]{0.47\textwidth}
        \centering
        \tcbincludegraphics[size=fbox,width=1.1\hsize,colframe=blue]{figures/Netatmo_Cooking_Baseline_Packets_07.03.png}
    \end{subfigure}
    \begin{subfigure}[b]{0.47\textwidth}
        \centering
        \tcbincludegraphics[size=fbox,width=1.1\hsize,colframe=red]{figures/Netatmo_Cooking_Packets_11.01.png}
    \end{subfigure}
    \begin{subfigure}[b]{0.47\textwidth}
        \centering
        \tcbincludegraphics[size=fbox,width=1.1\hsize,colframe=blue]{figures/Netatmo_Cooking_Baseline_Packets_08.03.png}
    \end{subfigure}
    \begin{subfigure}[b]{0.47\textwidth}
        \centering
        \tcbincludegraphics[size=fbox,width=1.1\hsize,colframe=red]{figures/Netatmo_Cooking_Packets_16.01.png}
    \end{subfigure}
    \begin{subfigure}[b]{0.47\textwidth}
        \centering
        \tcbincludegraphics[size=fbox,width=1.1\hsize,colframe=blue]{figures/Netatmo_Cooking_Baseline_Packets_09.03.png}
    \end{subfigure}
    \begin{subfigure}[b]{0.47\textwidth}
        \centering
        \tcbincludegraphics[size=fbox,width=1.1\hsize,colframe=red]{figures/Netatmo_Cooking_Packets_18.01.png}
    \end{subfigure}
    \begin{subfigure}[b]{0.47\textwidth}
        \centering
        \tcbincludegraphics[size=fbox,width=1.1\hsize,colframe=blue]{figures/Netatmo_Cooking_Baseline_Packets_10.03.png}
    \end{subfigure}
        \begin{subfigure}[b]{0.47\textwidth}
        \centering
        \tcbincludegraphics[size=fbox,width=1.1\hsize,colframe=red]{figures/Netatmo_Cooking_Packets_19.01.png}
    \end{subfigure}
    \begin{subfigure}[b]{0.47\textwidth}
        \centering
        \tcbincludegraphics[size=fbox,width=1.1\hsize,colframe=blue]{figures/Netatmo_Cooking_Baseline_Packets_11.03.png}
    \end{subfigure}
    \begin{subfigure}[b]{0.47\textwidth}
        \centering
        \tcbincludegraphics[size=fbox,width=1.1\hsize,colframe=red]{figures/Netatmo_Cooking_Packets_25.01.png}
    \end{subfigure}
    \hspace{0.6cm}
    \begin{subfigure}[b]{0.47\textwidth}
    \centering
        \tcbincludegraphics[size=fbox,width=1.1\hsize,colframe=blue]{figures/Netatmo_Cooking_Baseline_Packets_12.03.png}
        \end{subfigure}
    \caption{Graphs of traffic flows from the cooking events measured in packets with event graphs framed in red and baseline graphs framed in blue for Netatmo. Event times are marked in red on the event graphs.}
    \label{fig:NetatmoCookingPackets1}
\end{figure}

\begin{figure}[H]
    \begin{subfigure}[b]{0.45\textwidth}
        \centering
        \tcbincludegraphics[size=fbox,width=1.1\hsize,colframe=red]{figures/Netatmo_Cooking_Packets_30.01.png}
    \end{subfigure}
    \begin{subfigure}[b]{0.45\textwidth}
        \centering
        \tcbincludegraphics[size=fbox,width=1.1\hsize, colframe=blue]{figures/Netatmo_Cooking_Baseline_Packets_13.03.png}
    \end{subfigure}
    \begin{subfigure}[b]{0.45\textwidth}
        \centering
        \tcbincludegraphics[size=fbox,width=1.1\hsize,colframe=red]{figures/Netatmo_Cooking_Packets_31.01.png}
    \end{subfigure}
    \begin{subfigure}[b]{0.45\textwidth}
        \centering
        \tcbincludegraphics[size=fbox,width=1.1\hsize,colframe=blue]{figures/Netatmo_Cooking_Baseline_Packets_14.03.png}
    \end{subfigure}
    \begin{subfigure}[b]{0.45\textwidth}
        \centering
        \tcbincludegraphics[size=fbox,width=1.1\hsize,colframe=red]{figures/Netatmo_Cooking_Packets_01.02.png}
    \end{subfigure}
    \hspace{1.1cm}
    \begin{subfigure}[b]{0.45\textwidth}
        \centering
        \tcbincludegraphics[size=fbox,width=1.1\hsize,colframe=blue]{figures/Netatmo_Cooking_Baseline_Packets_15.03.png}
    \end{subfigure}
    \caption{Continuing from Figure \ref{fig:NetatmoCookingPackets1}}
    \label{fig:NetatmoCookingPackets2}
\end{figure}

\begin{figure}[H]
    \begin{subfigure}[b]{0.45\textwidth}
        \centering
        \tcbincludegraphics[size=fbox,width=1.1\hsize,colframe=red]{figures/Netatmo_Cooking_Bytes_30.01.png}
    \end{subfigure}
    \begin{subfigure}[b]{0.45\textwidth}
        \centering
        \tcbincludegraphics[size=fbox,width=1.1\hsize, colframe=blue]{figures/Netatmo_Cooking_Baseline_Bytes_13.03.png}
    \end{subfigure}
    \begin{subfigure}[b]{0.45\textwidth}
        \centering
        \tcbincludegraphics[size=fbox,width=1.1\hsize,colframe=red]{figures/Netatmo_Cooking_Bytes_31.01.png}
    \end{subfigure}
    \begin{subfigure}[b]{0.45\textwidth}
        \centering
        \tcbincludegraphics[size=fbox,width=1.1\hsize,colframe=blue]{figures/Netatmo_Cooking_Baseline_Bytes_14.03.png}
    \end{subfigure}
    \begin{subfigure}[b]{0.45\textwidth}
        \centering
        \tcbincludegraphics[size=fbox,width=1.1\hsize,colframe=red]{figures/Netatmo_Cooking_Bytes_01.02.png}
    \end{subfigure}
    \hspace{1.1cm}
    \begin{subfigure}[b]{0.45\textwidth}
        \centering
        \tcbincludegraphics[size=fbox,width=1.1\hsize,colframe=blue]{figures/Netatmo_Cooking_Baseline_Bytes_15.03.png}
    \end{subfigure}
    \caption{Remaining graphs from Figure \ref{fig:NetatmoCookingBytes2}}
    \label{fig:NetatmoCookingBytes1}
\end{figure}

\begin{figure}[H]
    \begin{subfigure}[b]{0.47\textwidth}
        \centering
        \tcbincludegraphics[size=fbox,width=1.1\hsize,colframe=red]{figures/Netatmo_Cooking_Bytes_08.01.png}
    \end{subfigure}
    \begin{subfigure}[b]{0.47\textwidth}
        \centering
        \tcbincludegraphics[size=fbox,width=1.1\hsize,colframe=blue]{figures/Netatmo_Cooking_Baseline_Bytes_06.03.png}
    \end{subfigure}
    \begin{subfigure}[b]{0.47\textwidth}
        \centering
        \tcbincludegraphics[size=fbox,width=1.1\hsize,colframe=red]{figures/Netatmo_Cooking_Bytes_09.01.png}
    \end{subfigure}
    \begin{subfigure}[b]{0.47\textwidth}
        \centering
        \tcbincludegraphics[size=fbox,width=1.1\hsize,colframe=blue]{figures/Netatmo_Cooking_Baseline_Bytes_07.03.png}
    \end{subfigure}
    \begin{subfigure}[b]{0.47\textwidth}
        \centering
        \tcbincludegraphics[size=fbox,width=1.1\hsize,colframe=red]{figures/Netatmo_Cooking_Bytes_11.01.png}
    \end{subfigure}
    \begin{subfigure}[b]{0.47\textwidth}
        \centering
        \tcbincludegraphics[size=fbox,width=1.1\hsize,colframe=blue]{figures/Netatmo_Cooking_Baseline_Bytes_08.03.png}
    \end{subfigure}
    \begin{subfigure}[b]{0.47\textwidth}
        \centering
        \tcbincludegraphics[size=fbox,width=1.1\hsize,colframe=red]{figures/Netatmo_Cooking_Bytes_16.01.png}
    \end{subfigure}
    \begin{subfigure}[b]{0.47\textwidth}
        \centering
        \tcbincludegraphics[size=fbox,width=1.1\hsize,colframe=blue]{figures/Netatmo_Cooking_Baseline_Bytes_09.03.png}
    \end{subfigure}
    \begin{subfigure}[b]{0.47\textwidth}
        \centering
        \tcbincludegraphics[size=fbox,width=1.1\hsize,colframe=red]{figures/Netatmo_Cooking_Bytes_18.01.png}
    \end{subfigure}
    \begin{subfigure}[b]{0.47\textwidth}
        \centering
        \tcbincludegraphics[size=fbox,width=1.1\hsize,colframe=blue]{figures/Netatmo_Cooking_Baseline_Bytes_10.03.png}
    \end{subfigure}
        \begin{subfigure}[b]{0.47\textwidth}
        \centering
        \tcbincludegraphics[size=fbox,width=1.1\hsize,colframe=red]{figures/Netatmo_Cooking_Bytes_19.01.png}
    \end{subfigure}
    \begin{subfigure}[b]{0.47\textwidth}
        \centering
        \tcbincludegraphics[size=fbox,width=1.1\hsize,colframe=blue]{figures/Netatmo_Cooking_Baseline_Bytes_11.03.png}
    \end{subfigure}
    \begin{subfigure}[b]{0.47\textwidth}
        \centering
        \tcbincludegraphics[size=fbox,width=1.1\hsize,colframe=red]{figures/Netatmo_Cooking_Bytes_25.01.png}
    \end{subfigure}
    \hspace{0.6cm}
    \begin{subfigure}[b]{0.47\textwidth}
    \centering
        \tcbincludegraphics[size=fbox,width=1.1\hsize,colframe=blue]{figures/Netatmo_Cooking_Baseline_Bytes_12.03.png}
        \end{subfigure}
    \caption{Graphs of traffic flows from the cooking events measured in bytes with event graphs framed in red and baseline graphs framed in blue for Netatmo. Event times are marked in red on the event graphs.}    \label{fig:NetatmoCookingBytes2}
\end{figure}

Table \ref{tab:NetatmoCookingCalculations} shows that the calculations varies a lot for each event. Number of packets varies from 403 to 1,082 and bytes from 58,926 to 146,001. The same is found for the baseline calculations, which also varies with 584 packets as the smallest to 972 as the highest amount of packets and for bytes from 77,780 to 132,406 bytes. The biggest packet from both the events and the baseline is mainly 407 bytes, with a few exceptions for both. The average values for the biggest packet are similar to each other for the events and baseline. Therefore it is not possible to distinguish an event based on the biggest packet. 

Considering the values in Table \ref{tab:NetatmoComparingBaselineAndCookingCalculations}, the average value of the baseline is within the standard deviation of the cooking events for both packets and bytes. This results in that the calculations cannot be used to identify the cooking event for Netatmo.

The same result is further confirmed in the graphs in Figures \ref{fig:NetatmoCookingPackets1} and \ref{fig:NetatmoCookingPackets2} for packets and \ref{fig:NetatmoCookingBytes1} and \ref{fig:NetatmoCookingBytes2} for bytes. For both cooking event graphs, in red, and baseline graphs, in blue, spikes that differentiate from the continuous traffic are visible for both packets and bytes. However, these are not connected to the red area when the event is ongoing and is also present in the baseline graphs, where no event is triggered. The event graphs does not show a traffic flow change in the red marked area. This results in that it is not possible to distinguish that cooking is ongoing in the environment by looking at the graphs of traffic flow. The results from cooking event on Netatmo is that there is no changes or differences in calculations or traffic flows which can be used to identify the event.

\newpage
\subsection{Mill}
The results from the cooking events for Mill are presented with both numerical values in tables and graphs and in figures where traffic patterns are analyzed. Table \ref{tab:MillCookingCalculations} presents the calculations from the event traffic with packets, bytes and biggest packet sent and received during the packet capturing. The same calculations have been made on the baseline traffic, in Table \ref{tab:MillBaselineCookingCalculations}. Table \ref{tab:MillComparingBaselineAndCookingCalculations} compares the average and standard deviation values for the event and baseline traffic. The calculations from these three tables are graphically presented in Figure \ref{fig:MillCookingCalculations} where packets and bytes from the event and baseline traffic are included with the average value for each of them. 

\begin{table}[H]
    \centering
    \caption{Calculations on cooking events for Mill}
    \begin{tabular}{|l|l|l|l|l|l|}
    \hline
        \textbf{Dates} & \textbf{Packets} & \textbf{Bytes} & \textbf{Biggest packet} \\ \hline
        08.jan & 8,875  & 1,097,786 & 456 bytes   \\ \hline
        09.jan & 7,948  & 1,057,936 & 456 bytes   \\ \hline
        11.jan & 10,222 & 1,310,644 & 456 bytes   \\ \hline
        16.jan & 10,014 & 1,358,615 & 1,353 bytes \\ \hline
        18.jan & 9,185  & 1,137,586 & 456 bytes   \\ \hline
        19.jan & 8,306  & 1,062,743 & 1,593 bytes \\ \hline
        25.jan & 8,246  & 1,033,976 & 1,583 bytes \\ \hline
        30.jan & 11,826 & 1,357,595 & 1,343 bytes  \\ \hline
        31.jan & 10,464 & 1,205,006 & 456 bytes \\ \hline
        01.feb & 10,124 & 1,219,206 & 456 bytes \\ \hline
    \end{tabular}
    \label{tab:MillCookingCalculations}
\end{table}

\begin{table}[H]
    \centering
    \caption{Calculations on comparing baseline files for the cooking event for the
device Mill}
    \begin{tabular}{|l|l|l|l|l|l|}
    \hline
        \textbf{Baseline} & \textbf{Packets} & \textbf{Bytes} & \textbf{Biggest packet} \\ \hline
        06.mar & 8,808  & 835,070   & 426 bytes \\ \hline
        07.mar & 9,310  & 940,428   & 1,353 bytes \\ \hline
        08.mar & 8,930  & 904,598   & 1,343 bytes \\ \hline
        09.mar & 10,675 & 1,046,076 & 456 bytes \\ \hline
        10.mar & 6,989  & 774,986   & 1,573 bytes \\ \hline
        11.mar & 9,983  & 1,107,006 & 1,273 bytes \\ \hline
        12.mar & 10,740 & 1,033,467 & 456 bytes \\ \hline
        13.mar & 8,134  & 826,038   & 426 bytes \\ \hline
        14.mar & 9,090  & 969,576   & 1,573 bytes \\ \hline
        15.mar & 6,539  & 740,761   & 429 bytes \\ \hline
    \end{tabular}
    \label{tab:MillBaselineCookingCalculations}
\end{table}

\begin{table}[H]
    \centering
    \caption{Traffic comparison between the cooking event and baseline for Mill}
    \begin{tabular}{c|l|l|l|l|}
        \cline{2-5}
        \multicolumn{1}{l|}{}                                              & \textbf{Type} & \textbf{Packets} & \textbf{Bytes} & \textbf{Biggest packet} \\ \hline
        \multicolumn{1}{|c|}{\multirow{2}{*}{\textbf{Average}}}            & Cooking         & 9,521              & 1,184,109       & 861 bytes               \\ \cline{2-5} 
        \multicolumn{1}{|c|}{}                                             & Baseline      & 8,920              & 917,801         & 931 bytes                \\ \hline
        \multicolumn{1}{|c|}{\multirow{2}{*}{\textbf{Standard deviation}}} & Cooking         & 1,221              & 125,182         & 529 bytes                 \\ \cline{2-5} 
        \multicolumn{1}{|c|}{}                                             & Baseline      & 1,404               & 122,928       &  527 bytes               \\ \hline          
    \end{tabular}
    \label{tab:MillComparingBaselineAndCookingCalculations}
\end{table}

\begin{figure}[H]
    \centering
    \begin{subfigure}{0.8\textwidth}
        \centering
        \includegraphics[width=1\hsize]{figures/Mill_Cooking_Calculations_Bytes.png} 
    \end{subfigure}
    \begin{subfigure}{0.8\textwidth}
        \centering
        \includegraphics[width=1\hsize]{figures/Mill_Cooking_Calculations_Packets.png} 
    \end{subfigure}
    \caption{Graphical presentation of event and baseline cooking calculations
with packets and bytes, including average value extracted from Table \ref{tab:MillComparingBaselineAndCookingCalculations} for Mill}
    \label{fig:MillCookingCalculations}
\end{figure}

The traffic pattern for the events and baseline are presented in Figure \ref{fig:MillCookingPackets1} and \ref{fig:MillCookingPackets2} for packets and in Figure \ref{fig:MillCookingBytes1} and \ref{fig:MillCookingBytes2} for bytes. In these figures, all graphs have the same minimum and maximum values on the y- and x-axis to be comparable. The graphs created from the different events are placed on the left side of the figure and framed in red, while the baseline is placed on the right side of the figure and framed in blue. The event times for when the event was ongoing are marked in red on the event graphs.

\begin{figure}[H]
    \begin{subfigure}[b]{0.47\textwidth}
        \centering
        \tcbincludegraphics[size=fbox,width=1.1\hsize,colframe=red]{figures/Mill_Cooking_Packets_08.01.png}
    \end{subfigure}
    \begin{subfigure}[b]{0.47\textwidth}
        \centering
        \tcbincludegraphics[size=fbox,width=1.1\hsize, colframe=blue]{figures/Mill_Cooking_Baseline_Packets_06.03.png}
    \end{subfigure}
    \begin{subfigure}[b]{0.47\textwidth}
        \centering
        \tcbincludegraphics[size=fbox,width=1.1\hsize,colframe=red]{figures/Mill_Cooking_Packets_09.01.png}
    \end{subfigure}
    \begin{subfigure}[b]{0.47\textwidth}
        \centering
        \tcbincludegraphics[size=fbox,width=1.1\hsize,colframe=blue]{figures/Mill_Cooking_Baseline_Packets_07.03.png}
    \end{subfigure}
    \begin{subfigure}[b]{0.47\textwidth}
        \centering
        \tcbincludegraphics[size=fbox,width=1.1\hsize,colframe=red]{figures/Mill_Cooking_Packets_11.01.png}
    \end{subfigure}
    \begin{subfigure}[b]{0.47\textwidth}
        \centering
        \tcbincludegraphics[size=fbox,width=1.1\hsize,colframe=blue]{figures/Mill_Cooking_Baseline_Packets_08.03.png}
    \end{subfigure}
    \begin{subfigure}[b]{0.47\textwidth}
        \centering
        \tcbincludegraphics[size=fbox,width=1.1\hsize,colframe=red]{figures/Mill_Cooking_Packets_16.01.png}
    \end{subfigure}
    \begin{subfigure}[b]{0.47\textwidth}
        \centering
        \tcbincludegraphics[size=fbox,width=1.1\hsize,colframe=blue]{figures/Mill_Cooking_Baseline_Packets_09.03.png}
    \end{subfigure}
    \begin{subfigure}[b]{0.47\textwidth}
        \centering
        \tcbincludegraphics[size=fbox,width=1.1\hsize,colframe=red]{figures/Mill_Cooking_Packets_18.01.png}
    \end{subfigure}
    \begin{subfigure}[b]{0.47\textwidth}
        \centering
        \tcbincludegraphics[size=fbox,width=1.1\hsize,colframe=blue]{figures/Mill_Cooking_Baseline_Packets_10.03.png}
    \end{subfigure}
        \begin{subfigure}[b]{0.47\textwidth}
        \centering
        \tcbincludegraphics[size=fbox,width=1.1\hsize,colframe=red]{figures/Mill_Cooking_Packets_19.01.png}
    \end{subfigure}
    \begin{subfigure}[b]{0.47\textwidth}
        \centering
        \tcbincludegraphics[size=fbox,width=1.1\hsize,colframe=blue]{figures/Mill_Cooking_Baseline_Packets_11.03.png}
    \end{subfigure}
    \begin{subfigure}[b]{0.47\textwidth}
        \centering
        \tcbincludegraphics[size=fbox,width=1.1\hsize,colframe=red]{figures/Mill_Cooking_Packets_25.01.png}
    \end{subfigure}
    \hspace{0.6cm}
    \begin{subfigure}[b]{0.47\textwidth}
    \centering
        \tcbincludegraphics[size=fbox,width=1.1\hsize,colframe=blue]{figures/Mill_Cooking_Baseline_Packets_12.03.png}
        \end{subfigure}
    \caption{Graphs of traffic flows from the cooking events measured in packets with event graphs framed in red and baseline graphs framed in blue for Mill. Event times are marked in red on the event graphs.}
    \label{fig:MillCookingPackets1}
\end{figure}

\begin{figure}[H]
    \begin{subfigure}[b]{0.45\textwidth}
        \centering
        \tcbincludegraphics[size=fbox,width=1.1\hsize,colframe=red]{figures/Mill_Cooking_Packets_30.01.png}
    \end{subfigure}
    \begin{subfigure}[b]{0.45\textwidth}
        \centering
        \tcbincludegraphics[size=fbox,width=1.1\hsize, colframe=blue]{figures/Mill_Cooking_Baseline_Packets_13.03.png}
    \end{subfigure}
    \begin{subfigure}[b]{0.45\textwidth}
        \centering
        \tcbincludegraphics[size=fbox,width=1.1\hsize,colframe=red]{figures/Mill_Cooking_Packets_31.01.png}
    \end{subfigure}
    \begin{subfigure}[b]{0.45\textwidth}
        \centering
        \tcbincludegraphics[size=fbox,width=1.1\hsize,colframe=blue]{figures/Mill_Cooking_Baseline_Packets_14.03.png}
    \end{subfigure}
    \begin{subfigure}[b]{0.45\textwidth}
        \centering
        \tcbincludegraphics[size=fbox,width=1.1\hsize,colframe=red]{figures/Mill_Cooking_Packets_01.02.png}
    \end{subfigure}
    \hspace{1.1cm}
    \begin{subfigure}[b]{0.45\textwidth}
        \centering
        \tcbincludegraphics[size=fbox,width=1.1\hsize,colframe=blue]{figures/Mill_Cooking_Baseline_Packets_15.03.png}
    \end{subfigure}
    \caption{Continuing from Figure \ref{fig:MillCookingPackets1}}
    \label{fig:MillCookingPackets2}
\end{figure}

\begin{figure}[H]
    \begin{subfigure}[b]{0.45\textwidth}
        \centering
        \tcbincludegraphics[size=fbox,width=1.1\hsize,colframe=red]{figures/Mill_Cooking_Bytes_30.01.png}
    \end{subfigure}
    \begin{subfigure}[b]{0.45\textwidth}
        \centering
        \tcbincludegraphics[size=fbox,width=1.1\hsize, colframe=blue]{figures/Mill_Cooking_Baseline_Bytes_13.03.png}
    \end{subfigure}
    \begin{subfigure}[b]{0.45\textwidth}
        \centering
        \tcbincludegraphics[size=fbox,width=1.1\hsize,colframe=red]{figures/Mill_Cooking_Bytes_31.01.png}
    \end{subfigure}
    \begin{subfigure}[b]{0.45\textwidth}
        \centering
        \tcbincludegraphics[size=fbox,width=1.1\hsize,colframe=blue]{figures/Mill_Cooking_Baseline_Bytes_14.03.png}
    \end{subfigure}
    \begin{subfigure}[b]{0.45\textwidth}
        \centering
        \tcbincludegraphics[size=fbox,width=1.1\hsize,colframe=red]{figures/Mill_Cooking_Bytes_01.02.png}
    \end{subfigure}
    \hspace{1.1cm}
    \begin{subfigure}[b]{0.45\textwidth}
        \centering
        \tcbincludegraphics[size=fbox,width=1.1\hsize,colframe=blue]{figures/Mill_Cooking_Baseline_Bytes_15.03.png}
    \end{subfigure}
    \caption{Remaining graphs from Figure \ref{fig:MillCookingBytes2}}
    \label{fig:MillCookingBytes1}
\end{figure}

\begin{figure}[H]
    \begin{subfigure}[b]{0.47\textwidth}
        \centering
        \tcbincludegraphics[size=fbox,width=1.1\hsize,colframe=red]{figures/Mill_Cooking_Bytes_08.01.png}
    \end{subfigure}
    \begin{subfigure}[b]{0.47\textwidth}
        \centering
        \tcbincludegraphics[size=fbox,width=1.1\hsize,colframe=blue]{figures/Mill_Cooking_Baseline_Bytes_06.03.png}
    \end{subfigure}
    \begin{subfigure}[b]{0.47\textwidth}
        \centering
        \tcbincludegraphics[size=fbox,width=1.1\hsize,colframe=red]{figures/Mill_Cooking_Bytes_09.01.png}
    \end{subfigure}
    \begin{subfigure}[b]{0.47\textwidth}
        \centering
        \tcbincludegraphics[size=fbox,width=1.1\hsize,colframe=blue]{figures/Mill_Cooking_Baseline_Bytes_07.03.png}
    \end{subfigure}
    \begin{subfigure}[b]{0.47\textwidth}
        \centering
        \tcbincludegraphics[size=fbox,width=1.1\hsize,colframe=red]{figures/Mill_Cooking_Bytes_11.01.png}
    \end{subfigure}
    \begin{subfigure}[b]{0.47\textwidth}
        \centering
        \tcbincludegraphics[size=fbox,width=1.1\hsize,colframe=blue]{figures/Mill_Cooking_Baseline_Bytes_08.03.png}
    \end{subfigure}
    \begin{subfigure}[b]{0.47\textwidth}
        \centering
        \tcbincludegraphics[size=fbox,width=1.1\hsize,colframe=red]{figures/Mill_Cooking_Bytes_16.01.png}
    \end{subfigure}
    \begin{subfigure}[b]{0.47\textwidth}
        \centering
        \tcbincludegraphics[size=fbox,width=1.1\hsize,colframe=blue]{figures/Mill_Cooking_Baseline_Bytes_09.03.png}
    \end{subfigure}
    \begin{subfigure}[b]{0.47\textwidth}
        \centering
        \tcbincludegraphics[size=fbox,width=1.1\hsize,colframe=red]{figures/Mill_Cooking_Bytes_18.01.png}
    \end{subfigure}
    \begin{subfigure}[b]{0.47\textwidth}
        \centering
        \tcbincludegraphics[size=fbox,width=1.1\hsize,colframe=blue]{figures/Mill_Cooking_Baseline_Bytes_10.03.png}
    \end{subfigure}
        \begin{subfigure}[b]{0.47\textwidth}
        \centering
        \tcbincludegraphics[size=fbox,width=1.1\hsize,colframe=red]{figures/Mill_Cooking_Bytes_19.01.png}
    \end{subfigure}
    \begin{subfigure}[b]{0.47\textwidth}
        \centering
        \tcbincludegraphics[size=fbox,width=1.1\hsize,colframe=blue]{figures/Mill_Cooking_Baseline_Bytes_11.03.png}
    \end{subfigure}
    \begin{subfigure}[b]{0.47\textwidth}
        \centering
        \tcbincludegraphics[size=fbox,width=1.1\hsize,colframe=red]{figures/Mill_Cooking_Bytes_25.01.png}
    \end{subfigure}
    \hspace{0.6cm}
    \begin{subfigure}[b]{0.47\textwidth}
    \centering
        \tcbincludegraphics[size=fbox,width=1.1\hsize,colframe=blue]{figures/Mill_Cooking_Baseline_Bytes_12.03.png}
        \end{subfigure}
    \caption{Graphs of traffic flows from the cooking events measured in bytes with event graphs framed in red and baseline graphs framed in blue for Mill. Event times are marked in red on the event graphs.} 
    \label{fig:MillCookingBytes2}
\end{figure}

Comparing Table \ref{tab:MillCookingCalculations} and \ref{tab:MillBaselineCookingCalculations} shows that the number of packets sent to and from the device is around the same value and not a significant difference between the two. The average values for packets in Table \ref{tab:MillComparingBaselineAndCookingCalculations} also shows that the values are similar to each other. For bytes, the calculations shows that when an event is ongoing, the overall values are higher than for standard baseline traffic. The comparison in Table \ref{tab:MillComparingBaselineAndCookingCalculations} shows that the average value for events are 266,308 bytes higher. The same is presented in Figure \ref{fig:MillCookingCalculations} where there are a bigger difference in bytes than packets for events and baseline traffic. For the biggest packet sent, the packet sizes varies from around 450 bytes to 1,550 bytes for events and baseline traffic over the same time, shown in Table \ref{tab:MillCookingCalculations} and \ref{tab:MillBaselineCookingCalculations}. This results in no difference in the bigger packets sent and received during an event. 

Table \ref{tab:MillComparingBaselineAndCookingCalculations} shows that the average value of the baseline packets are within the range of the standard deviation of the cooking events. However, for bytes, the average value is lower than the standard deviation. This is also visible in Figure \ref{fig:MillCookingCalculations} where the difference in bytes are much clearer than for packets. When looking at Figure \ref{fig:MillCookingBytes1} and \ref{fig:MillCookingBytes2}, it is a difference between the traffic before the event and the baseline traffic. The graphs marked in blue shows that when the baseline capturing was ongoing, the number of bytes sent are much lower than for the days when the events were triggered. Therefore, we can conclude that the difference in calculations for bytes are not linked directly to the event triggered, but are affected by other changes in the environment.

The graphs in Figures \ref{fig:MillCookingPackets1} and \ref{fig:MillCookingPackets2} for packets and Figures \ref{fig:MillCookingBytes1} and \ref{fig:MillCookingBytes2} for bytes does not show a significant change in traffic pattern from when an event is ongoing, in the graphs framed in red, and the standard traffic pattern from the baseline, in the graphs framed in blue. The variations the calculations gave in bytes from the events and baseline is not very visible in the graphs as they both vary a lot. However, knowing that more bytes are sent during the events, it is possible to see that more of the event graphs have higher spikes than the baseline, but this is not applicable for all events. Even though there are some differences visible, these are not applicable for all executions of cooking for Mill. Therefore, it is not possible to show that there is a specific change in traffic patterns for cooking and not possible to identify when cooking is ongoing in the environment. 

\newpage
\subsection{Nedis}
In Test Case 1: Cooking for Nedis, tables with numerical calculations are presented together with graphs of the traffic flow. In Table \ref{tab:NedisCookingCalculations} the total amount of packets and bytes, including the biggest packet during the event are presented for each of the 10 cooking events. Table \ref{tab:NedisBaselineCookingCalculations} presents the same values, but in regards of the baseline pcaps. Table \ref{tab:NedisComparingBaselineAndCookingCalculations} compares the average values and standard deviation from the events and baseline for packets, bytes and biggest packet. Figure \ref{fig:NedisCookingCalculations} presents the calculations from Tables \ref{tab:NedisCookingCalculations}, \ref{tab:NedisBaselineCookingCalculations} and \ref{tab:NedisComparingBaselineAndCookingCalculations} with packets and bytes together with the average value for both the event and baseline traffic to easier compare it.  

\begin{table}[H]
\centering
    \caption{Calculations on cooking events for Nedis}
\label{tab:NedisCookingCalculations}
    \begin{tabular}{|l|l|l|l|}
        \hline
        \textbf{Dates}    & \textbf{Packets} & \textbf{Bytes}     & \textbf{Biggest packet} \\ \hline
        08.jan             & 24,494           & 3,318,519          & 485 bytes               \\ \hline
        09.jan             & 26,787           & 3,870,361          & 424 bytes               \\ \hline
        11.jan             & 24,381           & 3,550,947          & 424 bytes               \\ \hline
        16.jan             & 26,035           & 3,895,071          & 485 bytes               \\ \hline
        18.jan             & 25,398           & 3,233,845          & 424 bytes               \\ \hline
        19.jan             & 20,857           & 3,242,594          & 424 bytes               \\ \hline
        25.jan             & 24,233           & 3,095,693          & 424 bytes               \\ \hline
        30.jan             & 24,867           & 3,257,812          & 424 bytes               \\ \hline
        31.jan             & 23,882           & 3,179,117          & 424 bytes               \\ \hline
        01.feb             & 25,397           & 3,477,014          & 424 bytes               \\ \hline
    \end{tabular}
\end{table}

\begin{table}[H]
    \centering
    \caption{Calculations on comparing baseline files for the cooking event for Nedis}
    \begin{tabular}{|l|l|l|l|l|l|}
    \hline
        \textbf{Baseline} & \textbf{Packets} & \textbf{Bytes} & \textbf{Biggest packet} \\ \hline
        06.mar & 13,014 & 1,413,457 & 421 bytes \\ \hline
        07.mar & 18,634 & 2,001,978 & 421 bytes \\ \hline
        08.mar & 18,643 & 2,019,872 & 421 bytes \\ \hline
        09.mar & 21,885 & 2,606,363 & 424 bytes \\ \hline
        10.mar & 17,956 & 2,326,898 & 424 bytes \\ \hline
        11.mar & 18,380 & 2,126,816 & 485 bytes \\ \hline
        12.mar & 10,728 & 1,312,915 & 421 bytes \\ \hline
        13.mar & 15,815 & 1,968,872 & 424 bytes \\ \hline
        14.mar & 15,926 & 1,878,449 & 341 bytes \\ \hline
        15.mar & 19,260 & 2,524,793 & 458 bytes \\ \hline
    \end{tabular}
    \label{tab:NedisBaselineCookingCalculations}
\end{table}

\begin{table}[H]
    \centering
    \caption{Traffic comparison between the cooking event and baseline for Nedis}
    \begin{tabular}{c|l|l|l|l|}
        \cline{2-5}
        \multicolumn{1}{l|}{}                                              & \textbf{Type} & \textbf{Packets} & \textbf{Bytes} & \textbf{Biggest packet} \\ \hline
        \multicolumn{1}{|c|}{\multirow{2}{*}{\textbf{Average}}}            & Cooking         & 24,633             & 3,412,097       & 436 bytes               \\ \cline{2-5} 
        \multicolumn{1}{|c|}{}                                             & Baseline      & 17,024             & 2,018,041       & 424 bytes                \\ \hline
        \multicolumn{1}{|c|}{\multirow{2}{*}{\textbf{Standard deviation}}} & Cooking         & 1,595              & 281,705         & 26 bytes                 \\ \cline{2-5} 
        \multicolumn{1}{|c|}{}                                             & Baseline      & 3,248              & 420,982         & 36 bytes               \\ \hline          
    \end{tabular}
    \label{tab:NedisComparingBaselineAndCookingCalculations}
\end{table}

\begin{figure}[H]
    \centering
    \begin{subfigure}{0.8\textwidth}
        \centering
        \includegraphics[width=1\hsize]{figures/Nedis_Cooking_Calculations_Bytes.png} 
    \end{subfigure}
    \begin{subfigure}{0.8\textwidth}
        \centering
        \includegraphics[width=1\hsize]{figures/Nedis_Cooking_Calculations_Packets.png} 
    \end{subfigure}
    \caption{Graphical presentation of event and baseline cooking calculations with packets and bytes, including average value extracted from Table \ref{tab:NedisComparingBaselineAndCookingCalculations} for Nedis}
    \label{fig:NedisCookingCalculations}
\end{figure}

The graphical presentation of the traffic patterns during events are presented in Figures \ref{fig:NedisCookingPackets1} and \ref{fig:NedisCookingPackets2} for packets and in Figures \ref{fig:NedisCookingBytes1} and \ref{fig:NedisCookingBytes2} for bytes. In these figures, the corresponding graphs from the baseline traffic is also included to look for traffic changes during the events. In these figures, the graphs for events are placed on the left side of the figure and framed in red, while the baseline graphs are placed on the right side and framed in blue. The event graphs are marked with red when the event was ongoing. The minimum and maximum values on the x- and y-axis are equal for all graphs within the same figure. 

\begin{figure}[H]
    \begin{subfigure}[b]{0.47\textwidth}
        \centering
        \tcbincludegraphics[size=fbox,width=1.1\hsize,colframe=red]{figures/Nedis_Cooking_Packets_08.01.png}
    \end{subfigure}
    \begin{subfigure}[b]{0.47\textwidth}
        \centering
        \tcbincludegraphics[size=fbox,width=1.1\hsize, colframe=blue]{figures/Nedis_Cooking_Baseline_Packets_06.03.png}
    \end{subfigure}
    \begin{subfigure}[b]{0.47\textwidth}
        \centering
        \tcbincludegraphics[size=fbox,width=1.1\hsize,colframe=red]{figures/Nedis_Cooking_Packets_09.01.png}
    \end{subfigure}
    \begin{subfigure}[b]{0.47\textwidth}
        \centering
        \tcbincludegraphics[size=fbox,width=1.1\hsize,colframe=blue]{figures/Nedis_Cooking_Baseline_Packets_07.03.png}
    \end{subfigure}
    \begin{subfigure}[b]{0.47\textwidth}
        \centering
        \tcbincludegraphics[size=fbox,width=1.1\hsize,colframe=red]{figures/Nedis_Cooking_Packets_11.01.png}
    \end{subfigure}
    \begin{subfigure}[b]{0.47\textwidth}
        \centering
        \tcbincludegraphics[size=fbox,width=1.1\hsize,colframe=blue]{figures/Nedis_Cooking_Baseline_Packets_08.03.png}
    \end{subfigure}
    \begin{subfigure}[b]{0.47\textwidth}
        \centering
        \tcbincludegraphics[size=fbox,width=1.1\hsize,colframe=red]{figures/Nedis_Cooking_Packets_16.01.png}
    \end{subfigure}
    \begin{subfigure}[b]{0.47\textwidth}
        \centering
        \tcbincludegraphics[size=fbox,width=1.1\hsize,colframe=blue]{figures/Nedis_Cooking_Baseline_Packets_09.03.png}
    \end{subfigure}
    \begin{subfigure}[b]{0.47\textwidth}
        \centering
        \tcbincludegraphics[size=fbox,width=1.1\hsize,colframe=red]{figures/Nedis_Cooking_Packets_18.01.png}
    \end{subfigure}
    \begin{subfigure}[b]{0.47\textwidth}
        \centering
        \tcbincludegraphics[size=fbox,width=1.1\hsize,colframe=blue]{figures/Nedis_Cooking_Baseline_Packets_10.03.png}
    \end{subfigure}
        \begin{subfigure}[b]{0.47\textwidth}
        \centering
        \tcbincludegraphics[size=fbox,width=1.1\hsize,colframe=red]{figures/Nedis_Cooking_Packets_19.01.png}
    \end{subfigure}
    \begin{subfigure}[b]{0.47\textwidth}
        \centering
        \tcbincludegraphics[size=fbox,width=1.1\hsize,colframe=blue]{figures/Nedis_Cooking_Baseline_Packets_11.03.png}
    \end{subfigure}
    \begin{subfigure}[b]{0.47\textwidth}
        \centering
        \tcbincludegraphics[size=fbox,width=1.1\hsize,colframe=red]{figures/Nedis_Cooking_Packets_25.01.png}
    \end{subfigure}
    \hspace{0.6cm}
    \begin{subfigure}[b]{0.47\textwidth}
    \centering
        \tcbincludegraphics[size=fbox,width=1.1\hsize,colframe=blue]{figures/Nedis_Cooking_Baseline_Packets_12.03.png}
        \end{subfigure}
    \caption{Graphs of traffic flows from the cooking events measured in packets with event graphs framed in red and baseline graphs framed in blue for Nedis. Event times are marked in red on the event graphs.}
    \label{fig:NedisCookingPackets1}
\end{figure}

\begin{figure}[H]
    \begin{subfigure}[b]{0.45\textwidth}
        \centering
        \tcbincludegraphics[size=fbox,width=1.1\hsize,colframe=red]{figures/Nedis_Cooking_Packets_30.01.png}
    \end{subfigure}
    \begin{subfigure}[b]{0.45\textwidth}
        \centering
        \tcbincludegraphics[size=fbox,width=1.1\hsize, colframe=blue]{figures/Nedis_Cooking_Baseline_Packets_13.03.png}
    \end{subfigure}
    \begin{subfigure}[b]{0.45\textwidth}
        \centering
        \tcbincludegraphics[size=fbox,width=1.1\hsize,colframe=red]{figures/Nedis_Cooking_Packets_31.01.png}
    \end{subfigure}
    \begin{subfigure}[b]{0.45\textwidth}
        \centering
        \tcbincludegraphics[size=fbox,width=1.1\hsize,colframe=blue]{figures/Nedis_Cooking_Baseline_Packets_14.03.png}
    \end{subfigure}
    \begin{subfigure}[b]{0.45\textwidth}
        \centering
        \tcbincludegraphics[size=fbox,width=1.1\hsize,colframe=red]{figures/Nedis_Cooking_Packets_01.02.png}
    \end{subfigure}
    \hspace{1.1cm}
    \begin{subfigure}[b]{0.45\textwidth}
        \centering
        \tcbincludegraphics[size=fbox,width=1.1\hsize,colframe=blue]{figures/Nedis_Cooking_Baseline_Packets_15.03.png}
    \end{subfigure}
    \caption{Continuing from Figure \ref{fig:NedisCookingPackets1}}
    \label{fig:NedisCookingPackets2}
\end{figure}

\begin{figure}[H]
    \begin{subfigure}[b]{0.45\textwidth}
        \centering
        \tcbincludegraphics[size=fbox,width=1.1\hsize,colframe=red]{figures/Nedis_Cooking_Bytes_30.01.png}
    \end{subfigure}
    \begin{subfigure}[b]{0.45\textwidth}
        \centering
        \tcbincludegraphics[size=fbox,width=1.1\hsize, colframe=blue]{figures/Nedis_Cooking_Baseline_Bytes_13.03.png}
    \end{subfigure}
    \begin{subfigure}[b]{0.45\textwidth}
        \centering
        \tcbincludegraphics[size=fbox,width=1.1\hsize,colframe=red]{figures/Nedis_Cooking_Bytes_31.01.png}
    \end{subfigure}
    \begin{subfigure}[b]{0.45\textwidth}
        \centering
        \tcbincludegraphics[size=fbox,width=1.1\hsize,colframe=blue]{figures/Nedis_Cooking_Baseline_Bytes_14.03.png}
    \end{subfigure}
    \begin{subfigure}[b]{0.45\textwidth}
        \centering
        \tcbincludegraphics[size=fbox,width=1.1\hsize,colframe=red]{figures/Nedis_Cooking_Bytes_01.02.png}
    \end{subfigure}
    \hspace{1.1cm}
    \begin{subfigure}[b]{0.45\textwidth}
        \centering
        \tcbincludegraphics[size=fbox,width=1.1\hsize,colframe=blue]{figures/Nedis_Cooking_Baseline_Bytes_15.03.png}
    \end{subfigure}
    \caption{Remaining graphs from Figure \ref{fig:NedisCookingBytes2}}
    \label{fig:NedisCookingBytes1}
\end{figure}

\begin{figure}[H]
    \begin{subfigure}[b]{0.47\textwidth}
        \centering
        \tcbincludegraphics[size=fbox,width=1.1\hsize,colframe=red]{figures/Nedis_Cooking_Bytes_08.01.png}
    \end{subfigure}
    \begin{subfigure}[b]{0.47\textwidth}
        \centering
        \tcbincludegraphics[size=fbox,width=1.1\hsize,colframe=blue]{figures/Nedis_Cooking_Baseline_Bytes_06.03.png}
    \end{subfigure}
    \begin{subfigure}[b]{0.47\textwidth}
        \centering
        \tcbincludegraphics[size=fbox,width=1.1\hsize,colframe=red]{figures/Nedis_Cooking_Bytes_09.01.png}
    \end{subfigure}
    \begin{subfigure}[b]{0.47\textwidth}
        \centering
        \tcbincludegraphics[size=fbox,width=1.1\hsize,colframe=blue]{figures/Nedis_Cooking_Baseline_Bytes_07.03.png}
    \end{subfigure}
    \begin{subfigure}[b]{0.47\textwidth}
        \centering
        \tcbincludegraphics[size=fbox,width=1.1\hsize,colframe=red]{figures/Nedis_Cooking_Bytes_11.01.png}
    \end{subfigure}
    \begin{subfigure}[b]{0.47\textwidth}
        \centering
        \tcbincludegraphics[size=fbox,width=1.1\hsize,colframe=blue]{figures/Nedis_Cooking_Baseline_Bytes_08.03.png}
    \end{subfigure}
    \begin{subfigure}[b]{0.47\textwidth}
        \centering
        \tcbincludegraphics[size=fbox,width=1.1\hsize,colframe=red]{figures/Nedis_Cooking_Bytes_16.01.png}
    \end{subfigure}
    \begin{subfigure}[b]{0.47\textwidth}
        \centering
        \tcbincludegraphics[size=fbox,width=1.1\hsize,colframe=blue]{figures/Nedis_Cooking_Baseline_Bytes_09.03.png}
    \end{subfigure}
    \begin{subfigure}[b]{0.47\textwidth}
        \centering
        \tcbincludegraphics[size=fbox,width=1.1\hsize,colframe=red]{figures/Nedis_Cooking_Bytes_18.01.png}
    \end{subfigure}
    \begin{subfigure}[b]{0.47\textwidth}
        \centering
        \tcbincludegraphics[size=fbox,width=1.1\hsize,colframe=blue]{figures/Nedis_Cooking_Baseline_Bytes_10.03.png}
    \end{subfigure}
        \begin{subfigure}[b]{0.47\textwidth}
        \centering
        \tcbincludegraphics[size=fbox,width=1.1\hsize,colframe=red]{figures/Nedis_Cooking_Bytes_19.01.png}
    \end{subfigure}
    \begin{subfigure}[b]{0.47\textwidth}
        \centering
        \tcbincludegraphics[size=fbox,width=1.1\hsize,colframe=blue]{figures/Nedis_Cooking_Baseline_Bytes_11.03.png}
    \end{subfigure}
    \begin{subfigure}[b]{0.47\textwidth}
        \centering
        \tcbincludegraphics[size=fbox,width=1.1\hsize,colframe=red]{figures/Nedis_Cooking_Bytes_25.01.png}
    \end{subfigure}
    \hspace{0.6cm}
    \begin{subfigure}[b]{0.47\textwidth}
    \centering
        \tcbincludegraphics[size=fbox,width=1.1\hsize,colframe=blue]{figures/Nedis_Cooking_Baseline_Bytes_12.03.png}
        \end{subfigure}
    \caption{Graphs of traffic flows from the cooking events measured in bytes with event graphs framed in red and baseline graphs framed in blue for Nedis. Event times are marked in red on the event graphs.} 
    \label{fig:NedisCookingBytes2}
\end{figure}

Comparing Table \ref{tab:NedisCookingCalculations} and \ref{tab:NedisBaselineCookingCalculations} shows that overall during the events, more packets and bytes are sent than during standard baseline traffic. While for the biggest packet during the cooking-time, there are not a significant difference to when an event is ongoing. For the biggest packet sent, the values are around 400 bytes and do not differ from cooking in Table \ref{tab:NedisCookingCalculations} to baseline in \ref{tab:NedisBaselineCookingCalculations}. The same is shown in Table \ref{tab:NedisComparingBaselineAndCookingCalculations} where the average for both packets and bytes are higher during events. This is also visible in Figure \ref{fig:NedisCookingCalculations} where all the blue marked bars are lower than the red marked graphs for events with one exception in packets. 

Table \ref{tab:NedisComparingBaselineAndCookingCalculations} do shows that the average value for the baseline is lower than the standard deviation for the cooking events for both packets and bytes. However, the traffic flows in Figure \ref{fig:NedisCookingPackets1}, \ref{fig:NedisCookingPackets2}, \ref{fig:NedisCookingBytes1} and \ref{fig:NedisCookingBytes2} shows that the traffic pattern before the event is not comparable to the baseline traffic as it shows a higher number of packets and bytes. This shows that the difference in calculations are not necessary linked to specific event triggering, but rather other changes in the environment from the days the events and the baseline were captured. 

Figure \ref{fig:NedisCookingPackets1}, \ref{fig:NedisCookingPackets2}, \ref{fig:NedisCookingBytes1} and \ref{fig:NedisCookingBytes2} shows that there are no significant traffic changes during the time of the event triggering that is visible for all executions of the event. It is therefore not possible to identify cooking through looking at either the calculations or the traffic flows for Nedis.

\newpage
\section{Test Case 2: Showering}
This chapter presents the results and analysis for Test Case 2: Showering. The first subsection describes the general evaluation which is applicable for all the devices and the next three subsections presents the analysis and results for each of the devices separately. 
\subsection{General}
The showering event has been conducted 10 times and Table \ref{tab:ShoweringDates} presents the 10 different dates and exact times for when the event was ongoing.
\begin{table}[H]
    \centering
    \caption{Date and time for Test Case 2: Showering}
    \begin{adjustbox}{width=1\textwidth} 
        \begin{tabular}{l|l|l|l|l|l|l|l|l|l|l|}
            \cline{2-11}
                & 08.01 & 09.01 & 11.01 & 16.01 & 18.01 & 19.01 & 25.01 & 30.01 & 31.01 & 01.02 \\ \hline
            \multicolumn{1}{|l|}{Started showering}  & 19:59 & 20:14 & 20:01 & 20:12 & 20:02 & 20:00 & 20:03 & 20:00 & 20:01 & 20:00 \\ \hline
            \multicolumn{1}{|l|}{Finished showering} & 20:14 & 20:34 & 20:17 & 20:31 & 20:19 & 20:16 & 20:19 & 20:18 & 20:17 & 20:16 \\ \hline
        \end{tabular}
    \end{adjustbox}
    \label{tab:ShoweringDates}
\end{table}

To be able to compare the events to each other and the standard baseline traffic, all the pcaps created for this event will have the same start and finish time. Graphically, each event are marked at what time the actual event was ongoing to separate from before and after the event. When deciding which start and end time for these events, the earliest start and latest finish are used. These times are then added 30 minutes before and after to cover at least 30 minutes before and after each event. This gives the following values to use for further analysis:

\begin{itemize}
    \item Earliest showering start: 19:59
    \item Latest showering finished: 20:34
    \item Packet capture files start: 19:29
    \item Packet capture files end: 21:04
\end{itemize}

These times results in the following filters added to create the pcaps for each event:
\begin{itemize}
    \item frame.time >= "Month Date, Year 19:29:00" \&\& frame.time <= "Month Date, Year 21:04:00"
\end{itemize}

\newpage
\subsection{Netatmo}
For Netatmo, the 10 events for showering are presented both graphically in figures and numerically in tables. Table \ref{tab:NetatmoShowerCalculations} shows packets, bytes and the biggest packet in each of the packet captures for the events. Table \ref{tab:NetatmoBaselineShowerCalculations} presents the same calculations, but for the corresponding baseline pcaps. In Table \ref{tab:NetatmoComparingBaselineAndShowerCalculations}, the average and standard deviation values from Table \ref{tab:NetatmoShowerCalculations} and \ref{tab:NetatmoBaselineShowerCalculations} are compared to each other. In context of these three tables, two graphical presentations of total bytes and packets including the average values for events and baseline are presented to compare events to the baseline traffic in Figure \ref{fig:NetatmoShowerCalculations}. 

\begin{table}[H]
    \centering
    \caption{Calculations on showering events for Netatmo}
    \begin{tabular}{|l|l|l|l|l|l|}
    \hline
        \textbf{Dates} & \textbf{Packets} & \textbf{Bytes} & \textbf{Biggest packet} \\ \hline
        08.jan & 870   & 118,119 & 407 bytes\\ \hline
        09.jan & 816   & 112,821 & 407 bytes \\ \hline
        11.jan & 1,075 & 145,838 & 407 bytes\\ \hline
        16.jan & 667   & 90,370  & 407 bytes\\ \hline
        18.jan & 802   & 109,285 & 407 bytes\\ \hline
        19.jan & 636   & 85,860  & 407 bytes \\ \hline
        25.jan & 897   & 120,580 & 134 bytes \\ \hline
        30.jan & 862   & 116,726 & 134 bytes \\ \hline
        31.jan & 704   & 94,140  & 136 bytes \\ \hline
        01.feb & 694   & 94,053  & 407 bytes \\ \hline
    \end{tabular}
    \label{tab:NetatmoShowerCalculations}
\end{table}

\begin{table}[H]
    \centering
    \caption{Calculations on comparing baseline files for the showering event for Netatmo}
    \begin{tabular}{|l|l|l|l|}
    \hline
        \textbf{Baseline} & \textbf{Packets} & \textbf{Bytes} & \textbf{Biggest packet} \\ \hline
        06.mar & 614 & 81,600  & 136 bytes\\ \hline
        07.mar & 710 & 94,869  & 407 bytes\\ \hline
        08.mar & 780 & 94,892  & 160 bytes \\ \hline
        09.mar & 884 & 118,116 & 134 bytes \\ \hline
        10.mar & 594 & 79,258  & 136 bytes \\ \hline
        11.mar & 615 & 82,664  & 407 bytes \\ \hline
        12.mar & 857 & 116,703 & 407 bytes \\ \hline
        13.mar & 514 & 68,470  & 136 bytes \\ \hline
        14.mar & 808 & 108,120 & 407 bytes \\ \hline
        15.mar & 731 & 97,682  & 134 bytes \\ \hline
    \end{tabular}
    \label{tab:NetatmoBaselineShowerCalculations}
\end{table}

\begin{table}[H]
    \centering
    \caption{Traffic comparison between the showering event and baseline for Netatmo}
    \begin{tabular}{c|l|l|l|l|}
        \cline{2-5}
        \multicolumn{1}{l|}{}                                              & \textbf{Type} & \textbf{Packets} & \textbf{Bytes} & \textbf{Biggest packet} \\ \hline
        \multicolumn{1}{|c|}{\multirow{2}{*}{\textbf{Average}}}            & Showering         & 802              & 108,779        & 325 bytes               \\ \cline{2-5} 
        \multicolumn{1}{|c|}{}                                             & Baseline      & 711              & 94,237         & 246 bytes                \\ \hline
        \multicolumn{1}{|c|}{\multirow{2}{*}{\textbf{Standard deviation}}} & Showering         & 133              & 18,181         & 132 bytes                 \\ \cline{2-5} 
        \multicolumn{1}{|c|}{}                                             & Baseline      & 123              & 16,541         & 138 bytes               \\ \hline          
    \end{tabular}
    \label{tab:NetatmoComparingBaselineAndShowerCalculations}
\end{table}

\begin{figure}[H]
    \centering
    \begin{subfigure}{0.8\textwidth}
       \centering
       \includegraphics[width=1\hsize]{figures/Netatmo_Shower_Calculations_Packets.png} 
    \end{subfigure}
    \begin{subfigure}{0.8\textwidth}
        \centering
        \includegraphics[width=1\hsize]{figures/Netatmo_Shower_Calculations_Bytes.png} 
    \end{subfigure}
    \caption{Graphical presentation of event and baseline shower calculations with packets and bytes, including average value extracted from Table \ref{tab:NetatmoComparingBaselineAndShowerCalculations} for Netatmo}
    \label{fig:NetatmoShowerCalculations}
\end{figure}

Figures \ref{fig:NetatmoShowerPackets1} and \ref{fig:NetatmoShowerPackets2} gives the graphical presentation of the traffic flow during the showering times with the total number of packets sent and received on the y-axis. Figure \ref{fig:NetatmoShowerBytes1} and \ref{fig:NetatmoShowerBytes2} shows the same, but measured in bytes on the y-axis. The figures presenting the traffic flow all have the same minimum and maximum values on the y- and x-axis and event graphs are placed on the left side of the figure with a red frame, while baseline graphs are placed on the right side of the figure with a blue frame. The timings for when the event was ongoing are marked with a red area on the event graphs to look for changes in traffic pattern at that specific time. 

\begin{figure}[H]
    \begin{subfigure}[b]{0.47\textwidth}
        \centering
        \tcbincludegraphics[size=fbox,width=1.1\hsize,colframe=red]{figures/Netatmo_Shower_Packets_08.01.png}
    \end{subfigure}
    \begin{subfigure}[b]{0.47\textwidth}
        \centering
        \tcbincludegraphics[size=fbox,width=1.1\hsize, colframe=blue]{figures/Netatmo_Shower_Baseline_Packets_06.03.png}
    \end{subfigure}
    \begin{subfigure}[b]{0.47\textwidth}
        \centering
        \tcbincludegraphics[size=fbox,width=1.1\hsize,colframe=red]{figures/Netatmo_Shower_Packets_09.01.png}
    \end{subfigure}
    \begin{subfigure}[b]{0.47\textwidth}
        \centering
        \tcbincludegraphics[size=fbox,width=1.1\hsize,colframe=blue]{figures/Netatmo_Shower_Baseline_Packets_07.03.png}
    \end{subfigure}
    \begin{subfigure}[b]{0.47\textwidth}
        \centering
        \tcbincludegraphics[size=fbox,width=1.1\hsize,colframe=red]{figures/Netatmo_Shower_Packets_11.01.png}
    \end{subfigure}
    \begin{subfigure}[b]{0.47\textwidth}
        \centering
        \tcbincludegraphics[size=fbox,width=1.1\hsize,colframe=blue]{figures/Netatmo_Shower_Baseline_Packets_08.03.png}
    \end{subfigure}
    \begin{subfigure}[b]{0.47\textwidth}
        \centering
        \tcbincludegraphics[size=fbox,width=1.1\hsize,colframe=red]{figures/Netatmo_Shower_Packets_16.01.png}
    \end{subfigure}
    \begin{subfigure}[b]{0.47\textwidth}
        \centering
        \tcbincludegraphics[size=fbox,width=1.1\hsize,colframe=blue]{figures/Netatmo_Shower_Baseline_Packets_09.03.png}
    \end{subfigure}
    \begin{subfigure}[b]{0.47\textwidth}
        \centering
        \tcbincludegraphics[size=fbox,width=1.1\hsize,colframe=red]{figures/Netatmo_Shower_Packets_18.01.png}
    \end{subfigure}
    \begin{subfigure}[b]{0.47\textwidth}
        \centering
        \tcbincludegraphics[size=fbox,width=1.1\hsize,colframe=blue]{figures/Netatmo_Shower_Baseline_Packets_10.03.png}
    \end{subfigure}
        \begin{subfigure}[b]{0.47\textwidth}
        \centering
        \tcbincludegraphics[size=fbox,width=1.1\hsize,colframe=red]{figures/Netatmo_Shower_Packets_19.01.png}
    \end{subfigure}
    \begin{subfigure}[b]{0.47\textwidth}
        \centering
        \tcbincludegraphics[size=fbox,width=1.1\hsize,colframe=blue]{figures/Netatmo_Shower_Baseline_Packets_11.03.png}
    \end{subfigure}
    \begin{subfigure}[b]{0.47\textwidth}
        \centering
        \tcbincludegraphics[size=fbox,width=1.1\hsize,colframe=red]{figures/Netatmo_Shower_Packets_25.01.png}
    \end{subfigure}
    \hspace{0.6cm}
    \begin{subfigure}[b]{0.47\textwidth}
    \centering
        \tcbincludegraphics[size=fbox,width=1.1\hsize,colframe=blue]{figures/Netatmo_Shower_Baseline_Packets_12.03.png}
        \end{subfigure}
    \caption{Graphs of traffic flows from the showering events measured in packets with event graphs framed in red and baseline graphs framed in blue for Netatmo. Event times are marked in red on the event graphs.}
    \label{fig:NetatmoShowerPackets1}
\end{figure}

\begin{figure}[H]
    \begin{subfigure}[b]{0.45\textwidth}
        \centering
        \tcbincludegraphics[size=fbox,width=1.1\hsize,colframe=red]{figures/Netatmo_Shower_Packets_30.01.png}
    \end{subfigure}
    \begin{subfigure}[b]{0.45\textwidth}
        \centering
        \tcbincludegraphics[size=fbox,width=1.1\hsize, colframe=blue]{figures/Netatmo_Shower_Baseline_Packets_13.03.png}
    \end{subfigure}
    \begin{subfigure}[b]{0.45\textwidth}
        \centering
        \tcbincludegraphics[size=fbox,width=1.1\hsize,colframe=red]{figures/Netatmo_Shower_Packets_31.01.png}
    \end{subfigure}
    \begin{subfigure}[b]{0.45\textwidth}
        \centering
        \tcbincludegraphics[size=fbox,width=1.1\hsize,colframe=blue]{figures/Netatmo_Shower_Baseline_Packets_14.03.png}
    \end{subfigure}
    \begin{subfigure}[b]{0.45\textwidth}
        \centering
        \tcbincludegraphics[size=fbox,width=1.1\hsize,colframe=red]{figures/Netatmo_Shower_Packets_01.02.png}
    \end{subfigure}
    \hspace{1.1cm}
    \begin{subfigure}[b]{0.45\textwidth}
        \centering
        \tcbincludegraphics[size=fbox,width=1.1\hsize,colframe=blue]{figures/Netatmo_Shower_Baseline_Packets_15.03.png}
    \end{subfigure}
    \caption{Continuing from Figure \ref{fig:NetatmoShowerPackets1}}
    \label{fig:NetatmoShowerPackets2}
\end{figure}

\begin{figure}[H]
    \begin{subfigure}[b]{0.45\textwidth}
        \centering
        \tcbincludegraphics[size=fbox,width=1.1\hsize,colframe=red]{figures/Netatmo_Shower_Bytes_30.01.png}
    \end{subfigure}
    \begin{subfigure}[b]{0.45\textwidth}
        \centering
        \tcbincludegraphics[size=fbox,width=1.1\hsize, colframe=blue]{figures/Netatmo_Shower_Baseline_Bytes_13.03.png}
    \end{subfigure}
    \begin{subfigure}[b]{0.45\textwidth}
        \centering
        \tcbincludegraphics[size=fbox,width=1.1\hsize,colframe=red]{figures/Netatmo_Shower_Bytes_31.01.png}
    \end{subfigure}
    \begin{subfigure}[b]{0.45\textwidth}
        \centering
        \tcbincludegraphics[size=fbox,width=1.1\hsize,colframe=blue]{figures/Netatmo_Shower_Baseline_Bytes_14.03.png}
    \end{subfigure}
    \begin{subfigure}[b]{0.45\textwidth}
        \centering
        \tcbincludegraphics[size=fbox,width=1.1\hsize,colframe=red]{figures/Netatmo_Shower_Bytes_01.02.png}
    \end{subfigure}
    \hspace{1.1cm}
    \begin{subfigure}[b]{0.45\textwidth}
        \centering
        \tcbincludegraphics[size=fbox,width=1.1\hsize,colframe=blue]{figures/Netatmo_Shower_Baseline_Bytes_15.03.png}
    \end{subfigure}
    \caption{Remaining graphs from Figure \ref{fig:NetatmoShowerBytes2}}
    \label{fig:NetatmoShowerBytes1}
\end{figure}

\begin{figure}[H]
    \begin{subfigure}[b]{0.47\textwidth}
        \centering
        \tcbincludegraphics[size=fbox,width=1.1\hsize,colframe=red]{figures/Netatmo_Shower_Bytes_08.01.png}
    \end{subfigure}
    \begin{subfigure}[b]{0.47\textwidth}
        \centering
        \tcbincludegraphics[size=fbox,width=1.1\hsize,colframe=blue]{figures/Netatmo_Shower_Baseline_Bytes_06.03.png}
    \end{subfigure}
    \begin{subfigure}[b]{0.47\textwidth}
        \centering
        \tcbincludegraphics[size=fbox,width=1.1\hsize,colframe=red]{figures/Netatmo_Shower_Bytes_09.01.png}
    \end{subfigure}
    \begin{subfigure}[b]{0.47\textwidth}
        \centering
        \tcbincludegraphics[size=fbox,width=1.1\hsize,colframe=blue]{figures/Netatmo_Shower_Baseline_Bytes_07.03.png}
    \end{subfigure}
    \begin{subfigure}[b]{0.47\textwidth}
        \centering
        \tcbincludegraphics[size=fbox,width=1.1\hsize,colframe=red]{figures/Netatmo_Shower_Bytes_11.01.png}
    \end{subfigure}
    \begin{subfigure}[b]{0.47\textwidth}
        \centering
        \tcbincludegraphics[size=fbox,width=1.1\hsize,colframe=blue]{figures/Netatmo_Shower_Baseline_Bytes_08.03.png}
    \end{subfigure}
    \begin{subfigure}[b]{0.47\textwidth}
        \centering
        \tcbincludegraphics[size=fbox,width=1.1\hsize,colframe=red]{figures/Netatmo_Shower_Bytes_16.01.png}
    \end{subfigure}
    \begin{subfigure}[b]{0.47\textwidth}
        \centering
        \tcbincludegraphics[size=fbox,width=1.1\hsize,colframe=blue]{figures/Netatmo_Shower_Baseline_Bytes_09.03.png}
    \end{subfigure}
    \begin{subfigure}[b]{0.47\textwidth}
        \centering
        \tcbincludegraphics[size=fbox,width=1.1\hsize,colframe=red]{figures/Netatmo_Shower_Bytes_18.01.png}
    \end{subfigure}
    \begin{subfigure}[b]{0.47\textwidth}
        \centering
        \tcbincludegraphics[size=fbox,width=1.1\hsize,colframe=blue]{figures/Netatmo_Shower_Baseline_Bytes_10.03.png}
    \end{subfigure}
        \begin{subfigure}[b]{0.47\textwidth}
        \centering
        \tcbincludegraphics[size=fbox,width=1.1\hsize,colframe=red]{figures/Netatmo_Shower_Bytes_19.01.png}
    \end{subfigure}
    \begin{subfigure}[b]{0.47\textwidth}
        \centering
        \tcbincludegraphics[size=fbox,width=1.1\hsize,colframe=blue]{figures/Netatmo_Shower_Baseline_Bytes_11.03.png}
    \end{subfigure}
    \begin{subfigure}[b]{0.47\textwidth}
        \centering
        \tcbincludegraphics[size=fbox,width=1.1\hsize,colframe=red]{figures/Netatmo_Shower_Bytes_25.01.png}
    \end{subfigure}
    \hspace{0.6cm}
    \begin{subfigure}[b]{0.47\textwidth}
    \centering
        \tcbincludegraphics[size=fbox,width=1.1\hsize,colframe=blue]{figures/Netatmo_Shower_Baseline_Bytes_12.03.png}
        \end{subfigure}
    \caption{Graphs of traffic flows from the showering events measured in bytes with event graphs framed in red and baseline graphs framed in blue for Netatmo. Event times are marked in red on the event graphs.}    
    \label{fig:NetatmoShowerBytes2}
\end{figure}

The number of packets and bytes are relatively similar for both events and baseline traffic which is shown in Figure \ref{fig:NetatmoShowerCalculations}. For events packets varies from 636 to 1,075 and for the baseline it varies from 514 to 884. For bytes the events varies from 85,860 to 145,838 and for the baseline it varies from 68,470 to 118,116. This results in an average value a bit lower for the baseline traffic, but not enough to see a significant change in amount of packets or bytes during the event. Even though the average baseline traffic is less, several of the baseline days exceeds both packets and bytes for some of the event traffic captures. For biggest packet sent and received, both event and baseline traffic varies between around 150 bytes to 407 bytes. 

Table \ref{tab:NetatmoComparingBaselineAndShowerCalculations} shows that the average value for the baseline is within the range of standard deviation for both packets and bytes. It is therefore not possible to use these values to identify showering on Netatmo. 

The graphs of the traffic flow show no distinct difference for packets in Figure \ref{fig:NetatmoShowerPackets1} and \ref{fig:NetatmoShowerPackets2}. For bytes in Figure \ref{fig:NetatmoShowerBytes1} and \ref{fig:NetatmoShowerBytes2} several of the event graphs, marked in red, have higher spikes than the baseline graphs marked in blue. However, two things challenges the difference: the first is that a few of the baseline graphs also have spikes and a few of the event graphs are missing the spikes. The other one is that the spikes also occurs before the event has started and may not be associated to an event triggering. Therefore, it is not possible to see specific traffic changes when showering is ongoing for Netatmo. 

\newpage
\subsection{Mill}
The results from the showering events for Mill are presented in this subsection. Table \ref{tab:MillShowerCalculations} presents the total amount of packets and bytes sent and received and the biggest packet from each packet captures for the events. The same numbers are presented in Table \ref{tab:MillBaselineShowerCalculations} for the corresponding baseline captures. Table \ref{tab:MillComparingBaselineAndShowerCalculations} compares the average and standard deviation values for events and baseline. The total number of packets and bytes are also presented graphically in Figure \ref{fig:MillShowerCalculations}, including the average values. Events and baseline are included in the same figure to easier compare the values. 

\begin{table}[H]
    \centering
    \caption{Calculations on showering events for Mill}
    \begin{tabular}{|l|l|l|l|l|l|}
    \hline
        \textbf{Dates} & \textbf{Packets} & \textbf{Bytes} & \textbf{Biggest packet} \\ \hline
        08.jan & 8,400 & 1,112,718 & 1353 bytes   \\ \hline
        09.jan & 7,775 & 970,700   & 456 bytes   \\ \hline
        11.jan & 9,819 & 1,166,768 & 456 bytes   \\ \hline
        16.jan & 8,800 & 1,107,872 & 1,421 bytes \\ \hline
        18.jan & 9,339 & 1,162,327 & 834 bytes   \\ \hline
        19.jan & 8,938 & 1,110,851 & 1,353 bytes \\ \hline
        25.jan & 9,690 & 1,136,156 & 1,593 bytes \\ \hline
        30.jan & 7,838 & 909,992   & 456 bytes  \\ \hline
        31.jan & 7,930 & 995,619   & 456 bytes \\ \hline
        01.feb & 7,787 & 905,808   & 456 bytes \\ \hline
    \end{tabular}
    \label{tab:MillShowerCalculations}
\end{table}

\begin{table}[H]
    \centering
    \caption{Calculations on comparing baseline files for the showering event for Mill}
    \begin{tabular}{|l|l|l|l|l|l|}
    \hline
        \textbf{Baseline} & \textbf{Packets} & \textbf{Bytes} & \textbf{Biggest packet} \\ \hline
        06.mar & 6,308  & 623,993   & 456 bytes \\ \hline
        07.mar & 5,968  & 675,359   & 426 bytes \\ \hline
        08.mar & 8,920  & 899,908   & 426 bytes \\ \hline
        09.mar & 9,241  & 922,721   & 456 bytes \\ \hline
        10.mar & 6,789  & 670,474   & 456 bytes \\ \hline
        11.mar & 6,750  & 686,700   & 1,337 bytes \\ \hline
        12.mar & 10,151 & 1,006,385 & 456 bytes \\ \hline
        13.mar & 7,032  & 794,714   & 456 bytes \\ \hline
        14.mar & 7,290  & 839,659   & 456 bytes \\ \hline
        15.mar & 7,927  & 809,616   & 426 bytes \\ \hline
    \end{tabular}
    \label{tab:MillBaselineShowerCalculations}
\end{table}

\begin{table}[H]
    \centering
    \caption{Traffic comparison between the showering event and baseline for Mill}
    \begin{tabular}{c|l|l|l|l|}
        \cline{2-5}
        \multicolumn{1}{l|}{}                                              & \textbf{Type} & \textbf{Packets} & \textbf{Bytes} & \textbf{Biggest packet} \\ \hline
        \multicolumn{1}{|c|}{\multirow{2}{*}{\textbf{Average}}}            & Showering         & 8,632            & 1,057,862       & 883 bytes               \\ \cline{2-5} 
        \multicolumn{1}{|c|}{}                                             & Baseline      & 7,638            & 792,953         & 535 bytes                \\ \hline
        \multicolumn{1}{|c|}{\multirow{2}{*}{\textbf{Standard deviation}}} & Showering         & 801              & 102,055         & 489 bytes                 \\ \cline{2-5} 
        \multicolumn{1}{|c|}{}                                             & Baseline      & 1,381            & 126,913       
          &  282 bytes               \\ \hline          
    \end{tabular}
    \label{tab:MillComparingBaselineAndShowerCalculations}
\end{table}

\begin{figure}[H]
    \centering
    \begin{subfigure}{0.8\textwidth}
        \centering
        \includegraphics[width=1\hsize]{figures/Mill_Shower_Calculations_Bytes.png} 
    \end{subfigure}
    \begin{subfigure}{0.8\textwidth}
        \centering
        \includegraphics[width=1\hsize]{figures/Mill_Shower_Calculations_Packets.png} 
    \end{subfigure}
    \caption{Graphical presentation of event and baseline shower calculations with packets and bytes, including average value extracted from Table \ref{tab:MillComparingBaselineAndShowerCalculations} for Mill}
    \label{fig:MillShowerCalculations}
\end{figure}

Figures \ref{fig:MillShowerPackets1} and \ref{fig:MillShowerPackets2} show the traffic flow during the events, placed on the left side of the figure and framed in red, and the baseline placed on the right side of the figure and framed in blue, measured in number of packets. Figures \ref{fig:MillShowerBytes1} and \ref{fig:MillShowerBytes2} display the same, but measured in number of bytes. The timings for when the event was ongoing are marked red on the event graphs. All graphs within the same figure have the same minimum and maximum values for the x- and y-axis. 

\begin{figure}[H]
    \begin{subfigure}[b]{0.47\textwidth}
        \centering
        \tcbincludegraphics[size=fbox,width=1.1\hsize,colframe=red]{figures/Mill_Shower_Packets_08.01.png}
    \end{subfigure}
    \begin{subfigure}[b]{0.47\textwidth}
        \centering
        \tcbincludegraphics[size=fbox,width=1.1\hsize, colframe=blue]{figures/Mill_Shower_Baseline_Packets_06.03.png}
    \end{subfigure}
    \begin{subfigure}[b]{0.47\textwidth}
        \centering
        \tcbincludegraphics[size=fbox,width=1.1\hsize,colframe=red]{figures/Mill_Shower_Packets_09.01.png}
    \end{subfigure}
    \begin{subfigure}[b]{0.47\textwidth}
        \centering
        \tcbincludegraphics[size=fbox,width=1.1\hsize,colframe=blue]{figures/Mill_Shower_Baseline_Packets_07.03.png}
    \end{subfigure}
    \begin{subfigure}[b]{0.47\textwidth}
        \centering
        \tcbincludegraphics[size=fbox,width=1.1\hsize,colframe=red]{figures/Mill_Shower_Packets_11.01.png}
    \end{subfigure}
    \begin{subfigure}[b]{0.47\textwidth}
        \centering
        \tcbincludegraphics[size=fbox,width=1.1\hsize,colframe=blue]{figures/Mill_Shower_Baseline_Packets_08.03.png}
    \end{subfigure}
    \begin{subfigure}[b]{0.47\textwidth}
        \centering
        \tcbincludegraphics[size=fbox,width=1.1\hsize,colframe=red]{figures/Mill_Shower_Packets_16.01.png}
    \end{subfigure}
    \begin{subfigure}[b]{0.47\textwidth}
        \centering
        \tcbincludegraphics[size=fbox,width=1.1\hsize,colframe=blue]{figures/Mill_Shower_Baseline_Packets_09.03.png}
    \end{subfigure}
    \begin{subfigure}[b]{0.47\textwidth}
        \centering
        \tcbincludegraphics[size=fbox,width=1.1\hsize,colframe=red]{figures/Mill_Shower_Packets_18.01.png}
    \end{subfigure}
    \begin{subfigure}[b]{0.47\textwidth}
        \centering
        \tcbincludegraphics[size=fbox,width=1.1\hsize,colframe=blue]{figures/Mill_Shower_Baseline_Packets_10.03.png}
    \end{subfigure}
        \begin{subfigure}[b]{0.47\textwidth}
        \centering
        \tcbincludegraphics[size=fbox,width=1.1\hsize,colframe=red]{figures/Mill_Shower_Packets_19.01.png}
    \end{subfigure}
    \begin{subfigure}[b]{0.47\textwidth}
        \centering
        \tcbincludegraphics[size=fbox,width=1.1\hsize,colframe=blue]{figures/Mill_Shower_Baseline_Packets_11.03.png}
    \end{subfigure}
    \begin{subfigure}[b]{0.47\textwidth}
        \centering
        \tcbincludegraphics[size=fbox,width=1.1\hsize,colframe=red]{figures/Mill_Shower_Packets_25.01.png}
    \end{subfigure}
    \hspace{0.6cm}
    \begin{subfigure}[b]{0.47\textwidth}
    \centering
        \tcbincludegraphics[size=fbox,width=1.1\hsize,colframe=blue]{figures/Mill_Shower_Baseline_Packets_12.03.png}
        \end{subfigure}
    \caption{Graphs of traffic flows from the showering events measured in packets with event graphs framed in red and baseline graphs framed in blue for Mill. Event times are marked in red on the event graphs.}
    \label{fig:MillShowerPackets1}
\end{figure}

\begin{figure}[H]
    \begin{subfigure}[b]{0.45\textwidth}
        \centering
        \tcbincludegraphics[size=fbox,width=1.1\hsize,colframe=red]{figures/Mill_Shower_Packets_30.01.png}
    \end{subfigure}
    \begin{subfigure}[b]{0.45\textwidth}
        \centering
        \tcbincludegraphics[size=fbox,width=1.1\hsize, colframe=blue]{figures/Mill_Shower_Baseline_Packets_13.03.png}
    \end{subfigure}
    \begin{subfigure}[b]{0.45\textwidth}
        \centering
        \tcbincludegraphics[size=fbox,width=1.1\hsize,colframe=red]{figures/Mill_Shower_Packets_31.01.png}
    \end{subfigure}
    \begin{subfigure}[b]{0.45\textwidth}
        \centering
        \tcbincludegraphics[size=fbox,width=1.1\hsize,colframe=blue]{figures/Mill_Shower_Baseline_Packets_14.03.png}
    \end{subfigure}
    \begin{subfigure}[b]{0.45\textwidth}
        \centering
        \tcbincludegraphics[size=fbox,width=1.1\hsize,colframe=red]{figures/Mill_Shower_Packets_01.02.png}
    \end{subfigure}
    \hspace{1.1cm}
    \begin{subfigure}[b]{0.45\textwidth}
        \centering
        \tcbincludegraphics[size=fbox,width=1.1\hsize,colframe=blue]{figures/Mill_Shower_Baseline_Packets_15.03.png}
    \end{subfigure}
    \caption{Continuing from Figure \ref{fig:MillShowerPackets1}}
    \label{fig:MillShowerPackets2}
\end{figure}

\begin{figure}[H]
    \begin{subfigure}[b]{0.45\textwidth}
        \centering
        \tcbincludegraphics[size=fbox,width=1.1\hsize,colframe=red]{figures/Mill_Shower_Bytes_30.01.png}
    \end{subfigure}
    \begin{subfigure}[b]{0.45\textwidth}
        \centering
        \tcbincludegraphics[size=fbox,width=1.1\hsize, colframe=blue]{figures/Mill_Shower_Baseline_Bytes_13.03.png}
    \end{subfigure}
    \begin{subfigure}[b]{0.45\textwidth}
        \centering
        \tcbincludegraphics[size=fbox,width=1.1\hsize,colframe=red]{figures/Mill_Shower_Bytes_31.01.png}
    \end{subfigure}
    \begin{subfigure}[b]{0.45\textwidth}
        \centering
        \tcbincludegraphics[size=fbox,width=1.1\hsize,colframe=blue]{figures/Mill_Shower_Baseline_Bytes_14.03.png}
    \end{subfigure}
    \begin{subfigure}[b]{0.45\textwidth}
        \centering
        \tcbincludegraphics[size=fbox,width=1.1\hsize,colframe=red]{figures/Mill_Shower_Bytes_01.02.png}
    \end{subfigure}
    \hspace{1.1cm}
    \begin{subfigure}[b]{0.45\textwidth}
        \centering
        \tcbincludegraphics[size=fbox,width=1.1\hsize,colframe=blue]{figures/Mill_Shower_Baseline_Bytes_15.03.png}
    \end{subfigure}
    \caption{Remaining graphs from Figure \ref{fig:MillShowerBytes2}}
    \label{fig:MillShowerBytes1}
\end{figure}

\begin{figure}[H]
    \begin{subfigure}[b]{0.47\textwidth}
        \centering
        \tcbincludegraphics[size=fbox,width=1.1\hsize,colframe=red]{figures/Mill_Shower_Bytes_08.01.png}
    \end{subfigure}
    \begin{subfigure}[b]{0.47\textwidth}
        \centering
        \tcbincludegraphics[size=fbox,width=1.1\hsize,colframe=blue]{figures/Mill_Shower_Baseline_Bytes_06.03.png}
    \end{subfigure}
    \begin{subfigure}[b]{0.47\textwidth}
        \centering
        \tcbincludegraphics[size=fbox,width=1.1\hsize,colframe=red]{figures/Mill_Shower_Bytes_09.01.png}
    \end{subfigure}
    \begin{subfigure}[b]{0.47\textwidth}
        \centering
        \tcbincludegraphics[size=fbox,width=1.1\hsize,colframe=blue]{figures/Mill_Shower_Baseline_Bytes_07.03.png}
    \end{subfigure}
    \begin{subfigure}[b]{0.47\textwidth}
        \centering
        \tcbincludegraphics[size=fbox,width=1.1\hsize,colframe=red]{figures/Mill_Shower_Bytes_11.01.png}
    \end{subfigure}
    \begin{subfigure}[b]{0.47\textwidth}
        \centering
        \tcbincludegraphics[size=fbox,width=1.1\hsize,colframe=blue]{figures/Mill_Shower_Baseline_Bytes_08.03.png}
    \end{subfigure}
    \begin{subfigure}[b]{0.47\textwidth}
        \centering
        \tcbincludegraphics[size=fbox,width=1.1\hsize,colframe=red]{figures/Mill_Shower_Bytes_16.01.png}
    \end{subfigure}
    \begin{subfigure}[b]{0.47\textwidth}
        \centering
        \tcbincludegraphics[size=fbox,width=1.1\hsize,colframe=blue]{figures/Mill_Shower_Baseline_Bytes_09.03.png}
    \end{subfigure}
    \begin{subfigure}[b]{0.47\textwidth}
        \centering
        \tcbincludegraphics[size=fbox,width=1.1\hsize,colframe=red]{figures/Mill_Shower_Bytes_18.01.png}
    \end{subfigure}
    \begin{subfigure}[b]{0.47\textwidth}
        \centering
        \tcbincludegraphics[size=fbox,width=1.1\hsize,colframe=blue]{figures/Mill_Shower_Baseline_Bytes_10.03.png}
    \end{subfigure}
        \begin{subfigure}[b]{0.47\textwidth}
        \centering
        \tcbincludegraphics[size=fbox,width=1.1\hsize,colframe=red]{figures/Mill_Shower_Bytes_19.01.png}
    \end{subfigure}
    \begin{subfigure}[b]{0.47\textwidth}
        \centering
        \tcbincludegraphics[size=fbox,width=1.1\hsize,colframe=blue]{figures/Mill_Shower_Baseline_Bytes_11.03.png}
    \end{subfigure}
    \begin{subfigure}[b]{0.47\textwidth}
        \centering
        \tcbincludegraphics[size=fbox,width=1.1\hsize,colframe=red]{figures/Mill_Shower_Bytes_25.01.png}
    \end{subfigure}
    \hspace{0.6cm}
    \begin{subfigure}[b]{0.47\textwidth}
    \centering
        \tcbincludegraphics[size=fbox,width=1.1\hsize,colframe=blue]{figures/Mill_Shower_Baseline_Bytes_12.03.png}
        \end{subfigure}
    \caption{Graphs of traffic flows from the showering events measured in bytes with event graphs framed in red and baseline graphs framed in blue for Mill. Event times are marked in red on the event graphs.}  
    \label{fig:MillShowerBytes2}
\end{figure}

Table \ref{tab:MillShowerCalculations} shows that the number of packets varies from 7,775 to 9,819 packets during the events compared to the baseline which varies from 5,968 to 10,151 packets shown in Table \ref{tab:MillBaselineShowerCalculations}. The baseline traffic has more variation than the event traffic. This also results in a higher standard deviation for the baseline shown in Table \ref{tab:MillComparingBaselineAndShowerCalculations}. For bytes, the same is shown, but the average value for the baseline is a lot smaller than for the events. As Table \ref{tab:MillBaselineShowerCalculations} shows, the bytes varies from 623,993 to 1,006,385 bytes, and in Table \ref{tab:MillShowerCalculations} the bytes varies from 909,992 to 1,166,768, which means that the average value for events are higher than for the baseline. Although the average value is higher, some of the baseline event days also measure similar amounts of bytes sent and received even though an event is not ongoing. This is also confirmed in Figure \ref{fig:MillShowerCalculations} where bytes for baseline are much lower than for events, but there are also events that are lower than baseline days. 

The biggest packet is mainly 456 bytes both during the showering event and the corresponding baseline with some variations of bigger packets sent for both cases. This results in no significant difference during an event. 

Table \ref{tab:MillComparingBaselineAndCookingCalculations} shows that for both packets and bytes, the average value of the baseline is smaller than the standard deviation from the events. However, Figure \ref{fig:MillShowerPackets1}, \ref{fig:MillShowerPackets2}, \ref{fig:MillShowerBytes1} and \ref{fig:MillShowerBytes2} shows that that traffic patterns before the events are not comparable to the traffic patterns of the corresponding baseline graphs and therefore it is not the specific event triggering that is causing the differences in the average values for the bytes. This results in that the differences in calculations cannot be used to distinguish showering from standard traffic.

In the figures with packets, in Figure \ref{fig:MillShowerPackets1} and \ref{fig:MillShowerPackets2}, and bytes, in Figure \ref{fig:MillShowerBytes1} and \ref{fig:MillShowerBytes2} no distinct differences are visible in the graphical representation of the traffic flows. There are spikes with more traffic sent before the events and during the event that are similar and does not identify the event executions. Neither from the calculations, nor the graphs for showering event for Mill it is possible to see a clear difference in traffic patterns from when the user is showering or not. 

\newpage
\subsection{Nedis}
This subsection presents the results and analysis for Nedis during the showering event. The total amount of packets, bytes and the biggest packet during the packet captures are presented in Table \ref{tab:NedisShoweringCalculations} for the events and in Table \ref{tab:NedisBaselineShowerCalculations} for the corresponding baseline captures. Table \ref{tab:NedisComparingBaselineAndShowerCalculations} compares the average and standard deviation values calculated for the events and baseline. Figure \ref{fig:NedisShowerCalculations} shows a graphical presentation of the total number of bytes and packets during the events and baseline in the same figure with average values included.

\begin{table}[H]
\centering
    \caption{Calculations on showering events for Nedis}
\label{tab:NedisShoweringCalculations}
    \begin{tabular}{|l|l|l|l|}
        \hline
        \textbf{Dates} & \textbf{Packets} & \textbf{Bytes} & \textbf{Biggest packet} \\ \hline
        08.jan          & 28,069           & 3,720,194      & 485 bytes               \\ \hline
        09.jan          & 23,156           & 3,395,004      & 424 bytes               \\ \hline
        11.jan          & 23,368           & 3,025,569      & 424 bytes               \\ \hline
        16.jan          & 21,026           & 2,598,912      & 485 bytes               \\ \hline
        18.jan          & 18,385           & 2,091,914      & 458 bytes               \\ \hline
        19.jan          & 25,416           & 3,258,851      & 421 bytes               \\ \hline
        25.jan          & 25,092           & 2,459,011      & 458 bytes               \\ \hline
        30.jan          & 22,415           & 3,302,325      & 485 bytes               \\ \hline
        31.jan          & 18,393           & 2,364,979      & 424 bytes               \\ \hline
        01.feb          & 20,306           & 2,461,409      & 485 bytes               \\ \hline
    \end{tabular}
\end{table}

\begin{table}[H]
    \centering
    \caption{Calculations on comparing baseline files for the showering event for Nedis}
    \begin{tabular}{|l|l|l|l|l|l|}
    \hline
        \textbf{Baseline} & \textbf{Packets} & \textbf{Bytes} & \textbf{Biggest packet} \\ \hline
        06.mar & 12,495 & 1,534,872 & 522 bytes \\ \hline
        07.mar & 18,342 & 2,483,719 & 522 bytes \\ \hline
        08.mar & 18,318 & 2,116,146 & 424 bytes \\ \hline
        09.mar & 14,639 & 1,865,223 & 522 bytes \\ \hline
        10.mar & 9,893  & 1,434,033 & 522 bytes \\ \hline
        11.mar & 6,092  & 779,215   & 426 bytes \\ \hline
        12.mar & 10,736 & 1,273,551 & 426 bytes \\ \hline
        13.mar & 13,478 & 1,649,747 & 424 bytes \\ \hline
        14.mar & 14,239 & 1,668,084 & 522 bytes \\ \hline
        15.mar & 22,383 & 2,710,729 & 522 bytes \\ \hline
    \end{tabular}
    \label{tab:NedisBaselineShowerCalculations}
\end{table}

\begin{table}[H]
    \centering
    \caption{Traffic comparison between the showering event and baseline for Nedis}
    \begin{tabular}{c|l|l|l|l|}
        \cline{2-5}
        \multicolumn{1}{l|}{}                                              & \textbf{Type} & \textbf{Packets} & \textbf{Bytes} & \textbf{Biggest packet} \\ \hline
        \multicolumn{1}{|c|}{\multirow{2}{*}{\textbf{Average}}}            & Showering         & 22,563             & 2,867,817       & 455 bytes               \\ \cline{2-5} 
        \multicolumn{1}{|c|}{}                                             & Baseline      & 14,062             & 1,752,432       & 483 bytes                \\ \hline
        \multicolumn{1}{|c|}{\multirow{2}{*}{\textbf{Standard deviation}}} & Showering         & 3,130              & 540,631         & 29 bytes                 \\ \cline{2-5} 
        \multicolumn{1}{|c|}{}                                             & Baseline      & 4,723              & 571,269         & 50 bytes               \\ \hline          
    \end{tabular}
    \label{tab:NedisComparingBaselineAndShowerCalculations}
\end{table}

\begin{figure}[H]
    \centering
    \begin{subfigure}{0.8\textwidth}
        \centering
        \includegraphics[width=1\hsize]{figures/Nedis_Shower_Calculations_Bytes.png} 
    \end{subfigure}
    \begin{subfigure}{0.8\textwidth}
        \centering
        \includegraphics[width=1\hsize]{figures/Nedis_Shower_Calculations_Packets.png} 
    \end{subfigure}
    \caption{Graphical presentation of event and baseline shower calculations with packets and bytes, including average value extracted from Table \ref{tab:NedisComparingBaselineAndShowerCalculations} for Nedis}
    \label{fig:NedisShowerCalculations}
\end{figure}

In Figures \ref{fig:NedisShowerPackets1} and \ref{fig:NedisShowerPackets2} a graphical presentation of the traffic flow measured in number of packets are displayed. The same view is presented in Figures \ref{fig:NedisShowerBytes1} and \ref{fig:NedisShowerBytes2} with number of bytes as the y-axis. These figures have the event graphs placed on the left side of the figure framed in red, while the baseline graphs are placed on the right side of the figure framed in blue to compare to each other. The time of the events is marked red on all of the event graphs. The x- and y-axis have the same minimum and maximum values for all graphs within the same figure. 

\begin{figure}[H]
    \begin{subfigure}[b]{0.47\textwidth}
        \centering
        \tcbincludegraphics[size=fbox,width=1.1\hsize,colframe=red]{figures/Nedis_Shower_Packets_08.01.png}
    \end{subfigure}
    \begin{subfigure}[b]{0.47\textwidth}
        \centering
        \tcbincludegraphics[size=fbox,width=1.1\hsize, colframe=blue]{figures/Nedis_Shower_Baseline_Packets_06.03.png}
    \end{subfigure}
    \begin{subfigure}[b]{0.47\textwidth}
        \centering
        \tcbincludegraphics[size=fbox,width=1.1\hsize,colframe=red]{figures/Nedis_Shower_Packets_09.01.png}
    \end{subfigure}
    \begin{subfigure}[b]{0.47\textwidth}
        \centering
        \tcbincludegraphics[size=fbox,width=1.1\hsize,colframe=blue]{figures/Nedis_Shower_Baseline_Packets_07.03.png}
    \end{subfigure}
    \begin{subfigure}[b]{0.47\textwidth}
        \centering
        \tcbincludegraphics[size=fbox,width=1.1\hsize,colframe=red]{figures/Nedis_Shower_Packets_11.01.png}
    \end{subfigure}
    \begin{subfigure}[b]{0.47\textwidth}
        \centering
        \tcbincludegraphics[size=fbox,width=1.1\hsize,colframe=blue]{figures/Nedis_Shower_Baseline_Packets_08.03.png}
    \end{subfigure}
    \begin{subfigure}[b]{0.47\textwidth}
        \centering
        \tcbincludegraphics[size=fbox,width=1.1\hsize,colframe=red]{figures/Nedis_Shower_Packets_16.01.png}
    \end{subfigure}
    \begin{subfigure}[b]{0.47\textwidth}
        \centering
        \tcbincludegraphics[size=fbox,width=1.1\hsize,colframe=blue]{figures/Nedis_Shower_Baseline_Packets_09.03.png}
    \end{subfigure}
    \begin{subfigure}[b]{0.47\textwidth}
        \centering
        \tcbincludegraphics[size=fbox,width=1.1\hsize,colframe=red]{figures/Nedis_Shower_Packets_18.01.png}
    \end{subfigure}
    \begin{subfigure}[b]{0.47\textwidth}
        \centering
        \tcbincludegraphics[size=fbox,width=1.1\hsize,colframe=blue]{figures/Nedis_Shower_Baseline_Packets_10.03.png}
    \end{subfigure}
        \begin{subfigure}[b]{0.47\textwidth}
        \centering
        \tcbincludegraphics[size=fbox,width=1.1\hsize,colframe=red]{figures/Nedis_Shower_Packets_19.01.png}
    \end{subfigure}
    \begin{subfigure}[b]{0.47\textwidth}
        \centering
        \tcbincludegraphics[size=fbox,width=1.1\hsize,colframe=blue]{figures/Nedis_Shower_Baseline_Packets_11.03.png}
    \end{subfigure}
    \begin{subfigure}[b]{0.47\textwidth}
        \centering
        \tcbincludegraphics[size=fbox,width=1.1\hsize,colframe=red]{figures/Nedis_Shower_Packets_25.01.png}
    \end{subfigure}
    \hspace{0.6cm}
    \begin{subfigure}[b]{0.47\textwidth}
    \centering
        \tcbincludegraphics[size=fbox,width=1.1\hsize,colframe=blue]{figures/Nedis_Shower_Baseline_Packets_12.03.png}
        \end{subfigure}
    \caption{Graphs of traffic flows from the showering events measured in packets with event graphs framed in red and baseline graphs framed in blue for Nedis. Event times are marked in red on the event graphs.}
    \label{fig:NedisShowerPackets1}
\end{figure}

\begin{figure}[H]
    \begin{subfigure}[b]{0.45\textwidth}
        \centering
        \tcbincludegraphics[size=fbox,width=1.1\hsize,colframe=red]{figures/Nedis_Shower_Packets_30.01.png}
    \end{subfigure}
    \begin{subfigure}[b]{0.45\textwidth}
        \centering
        \tcbincludegraphics[size=fbox,width=1.1\hsize, colframe=blue]{figures/Nedis_Shower_Baseline_Packets_13.03.png}
    \end{subfigure}
    \begin{subfigure}[b]{0.45\textwidth}
        \centering
        \tcbincludegraphics[size=fbox,width=1.1\hsize,colframe=red]{figures/Nedis_Shower_Packets_31.01.png}
    \end{subfigure}
    \begin{subfigure}[b]{0.45\textwidth}
        \centering
        \tcbincludegraphics[size=fbox,width=1.1\hsize,colframe=blue]{figures/Nedis_Shower_Baseline_Packets_14.03.png}
    \end{subfigure}
    \begin{subfigure}[b]{0.45\textwidth}
        \centering
        \tcbincludegraphics[size=fbox,width=1.1\hsize,colframe=red]{figures/Nedis_Shower_Packets_01.02.png}
    \end{subfigure}
    \hspace{1.1cm}
    \begin{subfigure}[b]{0.45\textwidth}
        \centering
        \tcbincludegraphics[size=fbox,width=1.1\hsize,colframe=blue]{figures/Nedis_Shower_Baseline_Packets_15.03.png}
    \end{subfigure}
    \caption{Continuing from Figure \ref{fig:NedisShowerPackets1}}
    \label{fig:NedisShowerPackets2}
\end{figure}

\begin{figure}[H]
    \begin{subfigure}[b]{0.45\textwidth}
        \centering
        \tcbincludegraphics[size=fbox,width=1.1\hsize,colframe=red]{figures/Nedis_Shower_Bytes_30.01.png}
    \end{subfigure}
    \begin{subfigure}[b]{0.45\textwidth}
        \centering
        \tcbincludegraphics[size=fbox,width=1.1\hsize, colframe=blue]{figures/Nedis_Shower_Baseline_Bytes_13.03.png}
    \end{subfigure}
    \begin{subfigure}[b]{0.45\textwidth}
        \centering
        \tcbincludegraphics[size=fbox,width=1.1\hsize,colframe=red]{figures/Nedis_Shower_Bytes_31.01.png}
    \end{subfigure}
    \begin{subfigure}[b]{0.45\textwidth}
        \centering
        \tcbincludegraphics[size=fbox,width=1.1\hsize,colframe=blue]{figures/Nedis_Shower_Baseline_Bytes_14.03.png}
    \end{subfigure}
    \begin{subfigure}[b]{0.45\textwidth}
        \centering
        \tcbincludegraphics[size=fbox,width=1.1\hsize,colframe=red]{figures/Nedis_Shower_Bytes_01.02.png}
    \end{subfigure}
    \hspace{1.1cm}
    \begin{subfigure}[b]{0.45\textwidth}
        \centering
        \tcbincludegraphics[size=fbox,width=1.1\hsize,colframe=blue]{figures/Nedis_Shower_Baseline_Bytes_15.03.png}
    \end{subfigure}
    \caption{Remaining graphs from Figure \ref{fig:NedisShowerBytes2}}
    \label{fig:NedisShowerBytes1}
\end{figure}

\begin{figure}[H]
    \begin{subfigure}[b]{0.47\textwidth}
        \centering
        \tcbincludegraphics[size=fbox,width=1.1\hsize,colframe=red]{figures/Nedis_Shower_Bytes_08.01.png}
    \end{subfigure}
    \begin{subfigure}[b]{0.47\textwidth}
        \centering
        \tcbincludegraphics[size=fbox,width=1.1\hsize,colframe=blue]{figures/Nedis_Shower_Baseline_Bytes_06.03.png}
    \end{subfigure}
    \begin{subfigure}[b]{0.47\textwidth}
        \centering
        \tcbincludegraphics[size=fbox,width=1.1\hsize,colframe=red]{figures/Nedis_Shower_Bytes_09.01.png}
    \end{subfigure}
    \begin{subfigure}[b]{0.47\textwidth}
        \centering
        \tcbincludegraphics[size=fbox,width=1.1\hsize,colframe=blue]{figures/Nedis_Shower_Baseline_Bytes_07.03.png}
    \end{subfigure}
    \begin{subfigure}[b]{0.47\textwidth}
        \centering
        \tcbincludegraphics[size=fbox,width=1.1\hsize,colframe=red]{figures/Nedis_Shower_Bytes_11.01.png}
    \end{subfigure}
    \begin{subfigure}[b]{0.47\textwidth}
        \centering
        \tcbincludegraphics[size=fbox,width=1.1\hsize,colframe=blue]{figures/Nedis_Shower_Baseline_Bytes_08.03.png}
    \end{subfigure}
    \begin{subfigure}[b]{0.47\textwidth}
        \centering
        \tcbincludegraphics[size=fbox,width=1.1\hsize,colframe=red]{figures/Nedis_Shower_Bytes_16.01.png}
    \end{subfigure}
    \begin{subfigure}[b]{0.47\textwidth}
        \centering
        \tcbincludegraphics[size=fbox,width=1.1\hsize,colframe=blue]{figures/Nedis_Shower_Baseline_Bytes_09.03.png}
    \end{subfigure}
    \begin{subfigure}[b]{0.47\textwidth}
        \centering
        \tcbincludegraphics[size=fbox,width=1.1\hsize,colframe=red]{figures/Nedis_Shower_Bytes_18.01.png}
    \end{subfigure}
    \begin{subfigure}[b]{0.47\textwidth}
        \centering
        \tcbincludegraphics[size=fbox,width=1.1\hsize,colframe=blue]{figures/Nedis_Shower_Baseline_Bytes_10.03.png}
    \end{subfigure}
        \begin{subfigure}[b]{0.47\textwidth}
        \centering
        \tcbincludegraphics[size=fbox,width=1.1\hsize,colframe=red]{figures/Nedis_Shower_Bytes_19.01.png}
    \end{subfigure}
    \begin{subfigure}[b]{0.47\textwidth}
        \centering
        \tcbincludegraphics[size=fbox,width=1.1\hsize,colframe=blue]{figures/Nedis_Shower_Baseline_Bytes_11.03.png}
    \end{subfigure}
    \begin{subfigure}[b]{0.47\textwidth}
        \centering
        \tcbincludegraphics[size=fbox,width=1.1\hsize,colframe=red]{figures/Nedis_Shower_Bytes_25.01.png}
    \end{subfigure}
    \hspace{0.6cm}
    \begin{subfigure}[b]{0.47\textwidth}
    \centering
        \tcbincludegraphics[size=fbox,width=1.1\hsize,colframe=blue]{figures/Nedis_Shower_Baseline_Bytes_12.03.png}
        \end{subfigure}
    \caption{Graphs of traffic flows from the showering events measured in bytes with event graphs framed in red and baseline graphs framed in blue for Nedis. Event times are marked in red on the event graphs.}  
    \label{fig:NedisShowerBytes2}
\end{figure}

Comparing the values for both packets and bytes in Figures \ref{tab:NedisShoweringCalculations} and \ref{tab:NedisBaselineShowerCalculations} show that during the events, most of the values are a lot higher than for the baseline calculations. For event, the packets varies from 18,385 to 28,069 and bytes from 2,091,914 to 3,720,194 bytes. However, there are values from the baseline that matches the range of the event values, such as for 7th, 8th and 15th of March where the values for packets are over 18,000 and 2,000,000 for bytes. For the biggest packet sent, the baseline have a higher overall value compared to the event values, where baseline are mostly around 500 bytes and events are lower than 500 bytes for every day. The average values also shows that the event values are higher than the baseline values for both packets and bytes. The difference in both average values and overall for events and baseline are significantly shown in Figure \ref{fig:NedisShowerCalculations} where both packets and bytes are much lower for baseline than events. 

Both in Table \ref{tab:NedisComparingBaselineAndShowerCalculations} and in Figure \ref{fig:NedisShowerCalculations} a clear difference between the showering and the baseline is visible. The average value for the baseline is much smaller than the standard deviation for the events for both packets and bytes. To look into if this difference is related to the event, Figure \ref{fig:NedisShowerPackets1}, \ref{fig:NedisShowerPackets2}, \ref{fig:NedisShowerBytes1} and \ref{fig:NedisShowerBytes2} will be evaluated. In these figures it is very visible that the differences in calculations are not connected to the event triggered, but rather holds a higher number of packets and bytes before the event compared to the standard baseline traffic. Therefore, these calculations are not comparable and will not identify when showering is ongoing in the environment for Nedis.

Looking at the graphs in Figures \ref{fig:NedisShowerPackets1} and \ref{fig:NedisShowerPackets2} the difference in number of packets are visible. However, it does not look like the number of packets changes when the shower event starts, but are just higher during several of those days. In these figures, one can also see that some of the baseline days looks similar as the event calculations in the tables. For bytes in Figure \ref{fig:NedisShowerBytes1} and \ref{fig:NedisShowerBytes2}, the results gives the same result as for packets. The same result is found here for Nedis as for the event tests on cooking. The calculations under the event looks different than for standard traffic in the baseline, but the graphs over the traffic flows shows that the differences are not related to when the event is triggered, but rather higher before and after the event. Therefore, it is not possible to see traffic pattern changes when showering is ongoing for Nedis. 

\newpage
\section{Test Case 3: Window Open}
This chapter presents the results and analysis on Test Case 3: Window Open. 
\subsection{General}
Test case 3: Window Open has been conducted 10 times over the course of 10 different days. The days and timings for the events are presented in Table \ref{tab:WindowDates}.

\begin{table}[H]
    \centering
    \caption{Date and time for Test Case 3: Window Open}
    \begin{adjustbox}{width=1\textwidth} 
            \begin{tabular}{l|l|l|l|l|l|l|l|l|l|l|}
                \cline{2-11}
                & 08.01 & 09.01 & 11.01 & 16.01 & 18.01 & 19.01 & 25.01 & 30.01 & 31.01 & 01.02 \\ \hline
                \multicolumn{1}{|l|}{Started window open}  & 23:00 & 23:00 & 22:50 & 23:10 & 23:15 & 23:02 & 22:59 & 23:00 & 22:59 & 22:59 \\ \hline
                \multicolumn{1}{|l|}{Finished window open} & 07:00 & 07:00 & 07:00 & 06:56 & 07:09 & 06:59 & 06:55 & 06:56 & 07:00 & 06:59 \\ \hline
            \end{tabular}
    \end{adjustbox}
    \label{tab:WindowDates}
\end{table}

To be able to compare the events to standby traffic, the baseline capture has been used with the same timings as for the actual events. To easier compare the events with each other and against the baseline, all packet captures have been filtered with the same start and finish time. Due to time limitations, the baseline only includes 9 full nights and therefore only 9 corresponding baseline packet captures have been made and used for comparison in this section. To ensure that all events have at least 30 minutes before and after the event was ongoing, the earliest time for starting and the latest time for finishing the event has been used to calculate the start and finish times for the files. Then 30 minutes are added to these times to ensure that each event has at least 30 minutes before and after to see traffic changes. The timings are calculated from Table \ref{tab:WindowDates} and presented in the list beneath: 

\begin{itemize}
    \item Earliest window open start: 22:50
    \item Latest window open finished: 07:09
    \item Packet capture files start: 22:20
    \item Packet capture files end: 07:39
\end{itemize}

These times results in the following filters added to create the pcaps for each event:
\begin{itemize}
    \item frame.time >= "Month Date, Year 22:20:00" \&\& frame.time <= "Month Date, Year 07:39:00"
\end{itemize}

\newpage
\subsection{Netatmo}
For the window open event for Netatmo, calculations are presented in Table \ref{tab:NetatmoWindowCalculations} with number of packets and bytes and the biggest packet during the capture. The same calculations on the comparing baseline files to be used in this test are presented in Table \ref{tab:NetatmoBaselineWindowCalculations}. The average values and standard deviation are compared for the event and baseline in Table \ref{tab:NetatmoComparingBaselineAndWindowCalculations}. The calculations are also shown graphically in Figure \ref{fig:NetatmoWindowCalculations} with number of packets and bytes from Tables \ref{tab:NetatmoWindowCalculations} and \ref{tab:NetatmoBaselineWindowCalculations} combined with their average value. 

\begin{table}[H]
    \centering
    \caption{Calculations on window open events for Netatmo}
    \begin{tabular}{|l|l|l|l|l|l|}
    \hline
        \textbf{Dates} & \textbf{Packets} & \textbf{Bytes} & \textbf{Biggest packet} \\ \hline
        08.jan & 4,716 & 641,245 & 407 bytes\\ \hline
        09.jan & 4,026 & 544,344 & 407 bytes \\ \hline
        11.jan & 5,149 & 697,920 & 407 bytes\\ \hline
        16.jan & 4,493 & 608,029 & 407 bytes\\ \hline
        18.jan & 4,165 & 565,885 & 407 bytes\\ \hline
        19.jan & 3,877 & 527,113 & 407 bytes \\ \hline
        25.jan & 4,460 & 605,296 & 407 bytes \\ \hline
        30.jan & 3,745 & 509,093 & 407 bytes \\ \hline
        31.jan & 3,494 & 468,922 & 407 bytes \\ \hline
        01.feb & 3,858 & 522,632 & 407 bytes \\ \hline
    \end{tabular}
    \label{tab:NetatmoWindowCalculations}
\end{table}

\begin{table}[H]
    \centering
    \caption{Calculations on comparing baseline files for the window open event for Netatmo}
    \begin{tabular}{|l|l|l|l|}
    \hline
        \textbf{Baseline} & \textbf{Packets} & \textbf{Bytes} & \textbf{Biggest packet} \\ \hline
        06.mar & 3,447 & 460,531 & 387 bytes\\ \hline
        07.mar & 4,356 & 591,234 & 407 bytes\\ \hline
        08.mar & 4,373 & 587,975 & 407 bytes \\ \hline
        09.mar & 4,513 & 609,590 & 407 bytes \\ \hline
        10.mar & 4,637 & 631,232 & 407 bytes \\ \hline
        11.mar & 4,010 & 545,391 & 407 bytes \\ \hline
        12.mar & 3,815 & 520,289 & 407 bytes \\ \hline
        13.mar & 3,571 & 478,194 & 407 bytes \\ \hline
        14.mar & 3,716 & 503,515 & 407 bytes \\ \hline
    \end{tabular}
    \label{tab:NetatmoBaselineWindowCalculations}
\end{table}

\begin{table}[H]
    \centering
    \caption{Traffic comparison between the window open event and baseline for Netatmo}
    \begin{tabular}{c|l|l|l|l|}
        \cline{2-5}
        \multicolumn{1}{l|}{}                                              & \textbf{Type} & \textbf{Packets} & \textbf{Bytes} & \textbf{Biggest packet} \\ \hline
        \multicolumn{1}{|c|}{\multirow{2}{*}{\textbf{Average}}}            & Window open         & 4,198            & 569,048        & 407 bytes               \\ \cline{2-5} 
        \multicolumn{1}{|c|}{}                                             & Baseline      & 4,049            & 547,550        & 405 bytes                \\ \hline
        \multicolumn{1}{|c|}{\multirow{2}{*}{\textbf{Standard deviation}}} & Window open         & 503              & 68,966         & 0 bytes                 \\ \cline{2-5} 
        \multicolumn{1}{|c|}{}                                             & Baseline      & 436              & 60,687         & 7 bytes               \\ \hline          
    \end{tabular}
    \label{tab:NetatmoComparingBaselineAndWindowCalculations}
\end{table}

\begin{figure}[H]
    \centering
    \begin{subfigure}{0.8\textwidth}
       \centering
       \includegraphics[width=1\hsize]{figures/Netatmo_Window_Calculations_Packets.png} 
    \end{subfigure}
    \begin{subfigure}{0.8\textwidth}
        \centering
        \includegraphics[width=1\hsize]{figures/Netatmo_Window_Calculations_Bytes.png} 
    \end{subfigure}
    \caption{Graphical presentation of event and baseline window open calculations with packets and bytes, including average value extracted from Table \ref{tab:NetatmoComparingBaselineAndWindowCalculations} for Netatmo}
    \label{fig:NetatmoWindowCalculations}
\end{figure}

Graphs over traffic patterns are presented in four different figures, for packets in Figures \ref{fig:NetatmoWindowPackets1} and \ref{fig:NetatmoWindowPackets2} and for bytes in Figures \ref{fig:NetatmoWindowBytes1} and \ref{fig:NetatmoWindowBytes2}. The graphs within the same figure have the same minimum and maximum values on the x- and y-axis. For all the figures, the event graphs are placed on the left side framed in red and have a red area marked on the graph which shows when the event was ongoing. The corresponding baseline graphs are placed on the right side of the figure and framed in blue to compare. 

\begin{figure}[H]
    \begin{subfigure}[b]{0.47\textwidth}
        \centering
        \tcbincludegraphics[size=fbox,width=1.1\hsize,colframe=red]{figures/Netatmo_Window_Packets_08.01-09.01.png}
    \end{subfigure}
    \begin{subfigure}[b]{0.47\textwidth}
        \centering
        \tcbincludegraphics[size=fbox,width=1.1\hsize, colframe=blue]{figures/Netatmo_Window_Baseline_Packets_06.03-07.03.png}
    \end{subfigure}
    \begin{subfigure}[b]{0.47\textwidth}
        \centering
        \tcbincludegraphics[size=fbox,width=1.1\hsize,colframe=red]{figures/Netatmo_Window_Packets_09.01-10.01.png}
    \end{subfigure}
    \begin{subfigure}[b]{0.47\textwidth}
        \centering
        \tcbincludegraphics[size=fbox,width=1.1\hsize,colframe=blue]{figures/Netatmo_Window_Baseline_Packets_07.03-08.03.png}
    \end{subfigure}
    \begin{subfigure}[b]{0.47\textwidth}
        \centering
        \tcbincludegraphics[size=fbox,width=1.1\hsize,colframe=red]{figures/Netatmo_Window_Packets_11.01-12.01.png}
    \end{subfigure}
    \begin{subfigure}[b]{0.47\textwidth}
        \centering
        \tcbincludegraphics[size=fbox,width=1.1\hsize,colframe=blue]{figures/Netatmo_Window_Baseline_Packets_08.03-09.03.png}
    \end{subfigure}
    \begin{subfigure}[b]{0.47\textwidth}
        \centering
        \tcbincludegraphics[size=fbox,width=1.1\hsize,colframe=red]{figures/Netatmo_Window_Packets_16.01-17.01.png}
    \end{subfigure}
    \begin{subfigure}[b]{0.47\textwidth}
        \centering
        \tcbincludegraphics[size=fbox,width=1.1\hsize,colframe=blue]{figures/Netatmo_Window_Baseline_Packets_09.03-10.03.png}
    \end{subfigure}
    \begin{subfigure}[b]{0.47\textwidth}
        \centering
        \tcbincludegraphics[size=fbox,width=1.1\hsize,colframe=red]{figures/Netatmo_Window_Packets_18.01-19.01.png}
    \end{subfigure}
    \begin{subfigure}[b]{0.47\textwidth}
        \centering
        \tcbincludegraphics[size=fbox,width=1.1\hsize,colframe=blue]{figures/Netatmo_Window_Baseline_Packets_10.03-11.03.png}
    \end{subfigure}
        \begin{subfigure}[b]{0.47\textwidth}
        \centering
        \tcbincludegraphics[size=fbox,width=1.1\hsize,colframe=red]{figures/Netatmo_Window_Packets_19.01-20.01.png}
    \end{subfigure}
    \begin{subfigure}[b]{0.47\textwidth}
        \centering
        \tcbincludegraphics[size=fbox,width=1.1\hsize,colframe=blue]{figures/Netatmo_Window_Baseline_Packets_11.03-12.03.png}
    \end{subfigure}
    \begin{subfigure}[b]{0.47\textwidth}
        \centering
        \tcbincludegraphics[size=fbox,width=1.1\hsize,colframe=red]{figures/Netatmo_Window_Packets_25.01-26.01.png}
    \end{subfigure}
    \hspace{0.6cm}
    \begin{subfigure}[b]{0.47\textwidth}
    \centering
        \tcbincludegraphics[size=fbox,width=1.1\hsize,colframe=blue]{figures/Netatmo_Window_Baseline_Packets_12.03-13.03.png}
        \end{subfigure}
    \caption{Graphs of traffic flows from the window open events measured in packets with event graphs framed in red and baseline graphs framed in blue for Netatmo. Event times are marked in red on the event graphs.}
    \label{fig:NetatmoWindowPackets1}
\end{figure}

\begin{figure}[H]
    \begin{subfigure}[b]{0.45\textwidth}
        \centering
        \tcbincludegraphics[size=fbox,width=1.1\hsize,colframe=red]{figures/Netatmo_Window_Packets_30.01-31.01.png}
    \end{subfigure}
    \begin{subfigure}[b]{0.45\textwidth}
        \centering
        \tcbincludegraphics[size=fbox,width=1.1\hsize, colframe=blue]{figures/Netatmo_Window_Baseline_Packets_13.03-14.03.png}
    \end{subfigure}
    \begin{subfigure}[b]{0.45\textwidth}
        \centering
        \tcbincludegraphics[size=fbox,width=1.1\hsize,colframe=red]{figures/Netatmo_Window_Packets_31.01-01.02.png}
    \end{subfigure}
    \begin{subfigure}[b]{0.45\textwidth}
        \centering
        \tcbincludegraphics[size=fbox,width=1.1\hsize,colframe=blue]{figures/Netatmo_Window_Baseline_Packets_14.03-15.03.png}
    \end{subfigure}
    \begin{subfigure}[b]{0.45\textwidth}
        \centering
        \tcbincludegraphics[size=fbox,width=1.1\hsize,colframe=red]{figures/Netatmo_Window_Packets_01.02-02.02.png}
    \end{subfigure}
    \caption{Continuing from Figure \ref{fig:NetatmoWindowPackets1}}
    \label{fig:NetatmoWindowPackets2}
\end{figure}

\begin{figure}[H]
    \begin{subfigure}[b]{0.45\textwidth}
        \centering
        \tcbincludegraphics[size=fbox,width=1.1\hsize,colframe=red]{figures/Netatmo_Window_Bytes_30.01-31.01.png}
    \end{subfigure}
    \begin{subfigure}[b]{0.45\textwidth}
        \centering
        \tcbincludegraphics[size=fbox,width=1.1\hsize, colframe=blue]{figures/Netatmo_Window_Baseline_Bytes_13.03-14.03.png}
    \end{subfigure}
    \begin{subfigure}[b]{0.45\textwidth}
        \centering
        \tcbincludegraphics[size=fbox,width=1.1\hsize,colframe=red]{figures/Netatmo_Window_Bytes_31.01-01.02.png}
    \end{subfigure}
    \begin{subfigure}[b]{0.45\textwidth}
        \centering
        \tcbincludegraphics[size=fbox,width=1.1\hsize,colframe=blue]{figures/Netatmo_Window_Baseline_Bytes_14.03-15.03.png}
    \end{subfigure}
    \begin{subfigure}[b]{0.45\textwidth}
        \centering
        \tcbincludegraphics[size=fbox,width=1.1\hsize,colframe=red]{figures/Netatmo_Window_Bytes_01.02-02.02.png}
    \end{subfigure}
    \caption{Remaining graphs from Figure \ref{fig:NetatmoWindowBytes2}}
    \label{fig:NetatmoWindowBytes1}
\end{figure}

\begin{figure}[H]
    \begin{subfigure}[b]{0.47\textwidth}
        \centering
        \tcbincludegraphics[size=fbox,width=1.1\hsize,colframe=red]{figures/Netatmo_Window_Bytes_08.01-09.01.png}
    \end{subfigure}
    \begin{subfigure}[b]{0.47\textwidth}
        \centering
        \tcbincludegraphics[size=fbox,width=1.1\hsize,colframe=blue]{figures/Netatmo_Window_Baseline_Bytes_06.03-07.03.png}
    \end{subfigure}
    \begin{subfigure}[b]{0.47\textwidth}
        \centering
        \tcbincludegraphics[size=fbox,width=1.1\hsize,colframe=red]{figures/Netatmo_Window_Bytes_09.01-10.01.png}
    \end{subfigure}
    \begin{subfigure}[b]{0.47\textwidth}
        \centering
        \tcbincludegraphics[size=fbox,width=1.1\hsize,colframe=blue]{figures/Netatmo_Window_Baseline_Bytes_07.03-08.03.png}
    \end{subfigure}
    \begin{subfigure}[b]{0.47\textwidth}
        \centering
        \tcbincludegraphics[size=fbox,width=1.1\hsize,colframe=red]{figures/Netatmo_Window_Bytes_11.01-12.01.png}
    \end{subfigure}
    \begin{subfigure}[b]{0.47\textwidth}
        \centering
        \tcbincludegraphics[size=fbox,width=1.1\hsize,colframe=blue]{figures/Netatmo_Window_Baseline_Bytes_08.03-09.03.png}
    \end{subfigure}
    \begin{subfigure}[b]{0.47\textwidth}
        \centering
        \tcbincludegraphics[size=fbox,width=1.1\hsize,colframe=red]{figures/Netatmo_Window_Bytes_16.01-17.01.png}
    \end{subfigure}
    \begin{subfigure}[b]{0.47\textwidth}
        \centering
        \tcbincludegraphics[size=fbox,width=1.1\hsize,colframe=blue]{figures/Netatmo_Window_Baseline_Bytes_09.03-10.03.png}
    \end{subfigure}
    \begin{subfigure}[b]{0.47\textwidth}
        \centering
        \tcbincludegraphics[size=fbox,width=1.1\hsize,colframe=red]{figures/Netatmo_Window_Bytes_18.01-19.01.png}
    \end{subfigure}
    \begin{subfigure}[b]{0.47\textwidth}
        \centering
        \tcbincludegraphics[size=fbox,width=1.1\hsize,colframe=blue]{figures/Netatmo_Window_Baseline_Bytes_10.03-11.03.png}
    \end{subfigure}
        \begin{subfigure}[b]{0.47\textwidth}
        \centering
        \tcbincludegraphics[size=fbox,width=1.1\hsize,colframe=red]{figures/Netatmo_Window_Bytes_19.01-20.01.png}
    \end{subfigure}
    \begin{subfigure}[b]{0.47\textwidth}
        \centering
        \tcbincludegraphics[size=fbox,width=1.1\hsize,colframe=blue]{figures/Netatmo_Window_Baseline_Bytes_11.03-12.03.png}
    \end{subfigure}
    \begin{subfigure}[b]{0.47\textwidth}
        \centering
        \tcbincludegraphics[size=fbox,width=1.1\hsize,colframe=red]{figures/Netatmo_Window_Bytes_25.01-26.01.png}
    \end{subfigure}
    \hspace{0.6cm}
    \begin{subfigure}[b]{0.47\textwidth}
    \centering
        \tcbincludegraphics[size=fbox,width=1.1\hsize,colframe=blue]{figures/Netatmo_Window_Baseline_Bytes_12.03-13.03.png}
        \end{subfigure}
    \caption{Graphs of traffic flows from the showering events measured in bytes with event graphs framed in red and baseline graphs framed in blue for Netatmo. Event times are marked in red on the event graphs.}  
    \label{fig:NetatmoWindowBytes2}
\end{figure}

Comparing Table \ref{tab:NetatmoWindowCalculations} for events and Table \ref{tab:NetatmoBaselineWindowCalculations} for baseline files, shows that the different columns are similar. The total number of packets stays around 4,000 packets for both events and baseline. For bytes, the values ranges from around 500,000 to around 700,000. This is also easily visible in Figure \ref{fig:NetatmoWindowCalculations} where total number of packets and bytes are very similar. For both events and baseline, the number of packets are around 3,000-5,000 and for bytes around 400,000 to 600,000. 

Table \ref{tab:NetatmoComparingBaselineAndWindowCalculations} and Figure \ref{fig:NetatmoWindowCalculations} shows that the values for window open and baseline are very similar. The average value for baseline packets and bytes are not smaller than the standard deviation for the window open. This means that it is not possible to identify if a user is showering using these calculations. 

Looking at the graphs in Figure \ref{fig:NetatmoWindowPackets1} and \ref{fig:NetatmoWindowPackets2}, there are no significant differences in the graphs for the events compared to the baseline on the right side. For both events and baseline graphs, the same traffic pattern is found. For bytes in Figure \ref{fig:NetatmoWindowBytes1} and \ref{fig:NetatmoWindowBytes2}, the events varies with some nights having many and high spikes, and other nights with smaller and fewer spikes. The same pattern is found for the baseline graphs marked in blue. This shows that it is not possible to distinguish whether the window is open or not by looking at the traffic patterns in a numerical or graphical way from the device Netatmo. 

\newpage
\subsection{Mill}
Table \ref{tab:MillWindowCalculations} presents the number of packets and bytes and the biggest packet during the capturing of the window open event for Mill. The same calculations are included for the comparing baseline files in Table \ref{tab:MillBaselineWindowCalculations}. To compare average and standard deviation values from the event, Table \ref{tab:MillComparingBaselineAndWindowCalculations} are shown. The calculations are also presented graphically in Figure \ref{fig:MillWindowCalculations} where number of packets and bytes for both the event dates and comparing baseline dates are included together with their average value extracted from Table \ref{tab:MillComparingBaselineAndWindowCalculations}.

\begin{table}[H]
    \centering
    \caption{Calculations on window open events for Mill}
    \begin{tabular}{|l|l|l|l|l|l|}
    \hline
        \textbf{Dates} & \textbf{Packets} & \textbf{Bytes} & \textbf{Biggest packet} \\ \hline
        08.jan & 49,204 & 6,159,905 & 1,353 bytes   \\ \hline
        09.jan & 49,515 & 6,055,179 & 1,545 bytes   \\ \hline
        11.jan & 45,680 & 4,998,933 & 1,515 bytes   \\ \hline
        16.jan & 45,559 & 4,881,294 & 1,353 bytes \\ \hline
        18.jan & 56,021 & 7,122,722 & 1,593 bytes   \\ \hline
        19.jan & 40,090 & 4,736,715 & 1,573 bytes \\ \hline
        25.jan & 52,728 & 6,166,428 & 1,421 bytes \\ \hline
        30.jan & 44,015 & 5,089,165 & 1,589 bytes  \\ \hline
        31.jan & 40,616 & 4,501,888 & 1,353 bytes \\ \hline
        01.feb & 37,963 & 4,030,897 & 456 bytes \\ \hline
    \end{tabular}
    \label{tab:MillWindowCalculations}
\end{table}

\begin{table}[H]
    \centering
    \caption{Calculations on comparing baseline files for the window open event for Mill}
    \begin{tabular}{|l|l|l|l|l|l|}
    \hline
        \textbf{Baseline} & \textbf{Packets} & \textbf{Bytes} & \textbf{Biggest packet} \\ \hline
        06.mar & 42,541 & 4,463,880 & 456 bytes \\ \hline
        07.mar & 44,018 & 4,634,112 & 800 bytes \\ \hline
        08.mar & 51,730 & 5,307,249 & 1,583 bytes \\ \hline
        09.mar & 56,022 & 5,568,551 & 1,343 bytes \\ \hline
        10.mar & 57,262 & 6,237,897 & 1,593 bytes \\ \hline
        11.mar & 49,145 & 4,959,292 & 1,593 bytes \\ \hline
        12.mar & 44,181 & 4,523,140 & 1,583 bytes \\ \hline
        13.mar & 49,628 & 5,238,854 & 456 bytes \\ \hline
        14.mar & 42,318 & 4,449,260 & 1,589 bytes \\ \hline
    \end{tabular}
    \label{tab:MillBaselineWindowCalculations}
\end{table}

\begin{table}[H]
    \centering
    \caption{Traffic comparison between the window open event and baseline for Mill}
    \begin{tabular}{c|l|l|l|l|}
        \cline{2-5}
        \multicolumn{1}{l|}{}                                              & \textbf{Type} & \textbf{Packets} & \textbf{Bytes} & \textbf{Biggest packet} \\ \hline
        \multicolumn{1}{|c|}{\multirow{2}{*}{\textbf{Average}}}            & Window open         & 46,139           & 5,374,313      & 1,353 bytes             \\ \cline{2-5} 
        \multicolumn{1}{|c|}{}                                             & Baseline      & 48,538           & 5,042,471      & 1,222 bytes              \\ \hline
        \multicolumn{1}{|c|}{\multirow{2}{*}{\textbf{Standard deviation}}} & Window open         & 5,782            & 954,686        & 338 bytes              \\ \cline{2-5} 
        \multicolumn{1}{|c|}{}                                             & Baseline      & 5,678            & 606,684        & 505 bytes             \\ \hline          
    \end{tabular}
    \label{tab:MillComparingBaselineAndWindowCalculations}
\end{table}

\begin{figure}[H]
    \centering
    \begin{subfigure}{0.8\textwidth}
        \centering
        \includegraphics[width=1\hsize]{figures/Mill_Window_Calculations_Bytes.png} 
    \end{subfigure}
    \begin{subfigure}{0.8\textwidth}
        \centering
        \includegraphics[width=1\hsize]{figures/Mill_Window_Calculations_Packets.png} 
    \end{subfigure}
    \caption{Graphical presentation of event and baseline window open calculations with packets and bytes, including average value extracted from Table \ref{tab:MillComparingBaselineAndWindowCalculations} for Mill}
    \label{fig:MillWindowCalculations}
\end{figure}

The packet capturings from the window open event for Mill is presented with traffic flows in four different figures. Figures \ref{fig:MillWindowPackets1} and \ref{fig:MillWindowPackets2} shows the traffic patterns for packets and Figures \ref{fig:MillWindowBytes1} and \ref{fig:MillWindowBytes2} for bytes. The event graphs are marked in red and placed on the left side of the figures, with a red marked area on the graphs to show when the event was ongoing. The comparing baseline graphs are placed on the right side of the figures and framed in blue. All graphs in the same figure have the same minimum and maximum value for the x- and y-axis. 

\begin{figure}[H]
    \begin{subfigure}[b]{0.47\textwidth}
        \centering
        \tcbincludegraphics[size=fbox,width=1.1\hsize,colframe=red]{figures/Mill_Window_Packets_08.01-09.01.png}
    \end{subfigure}
    \begin{subfigure}[b]{0.47\textwidth}
        \centering
        \tcbincludegraphics[size=fbox,width=1.1\hsize, colframe=blue]{figures/Mill_Window_Baseline_Packets_06.03-07.03.png}
    \end{subfigure}
    \begin{subfigure}[b]{0.47\textwidth}
        \centering
        \tcbincludegraphics[size=fbox,width=1.1\hsize,colframe=red]{figures/Mill_Window_Packets_09.01-10.01.png}
    \end{subfigure}
    \begin{subfigure}[b]{0.47\textwidth}
        \centering
        \tcbincludegraphics[size=fbox,width=1.1\hsize,colframe=blue]{figures/Mill_Window_Baseline_Packets_07.03-08.03.png}
    \end{subfigure}
    \begin{subfigure}[b]{0.47\textwidth}
        \centering
        \tcbincludegraphics[size=fbox,width=1.1\hsize,colframe=red]{figures/Mill_Window_Packets_11.01-12.01.png}
    \end{subfigure}
    \begin{subfigure}[b]{0.47\textwidth}
        \centering
        \tcbincludegraphics[size=fbox,width=1.1\hsize,colframe=blue]{figures/Mill_Window_Baseline_Packets_08.03-09.03.png}
    \end{subfigure}
    \begin{subfigure}[b]{0.47\textwidth}
        \centering
        \tcbincludegraphics[size=fbox,width=1.1\hsize,colframe=red]{figures/Mill_Window_Packets_16.01-17.01.png}
    \end{subfigure}
    \begin{subfigure}[b]{0.47\textwidth}
        \centering
        \tcbincludegraphics[size=fbox,width=1.1\hsize,colframe=blue]{figures/Mill_Window_Baseline_Packets_09.03-10.03.png}
    \end{subfigure}
    \begin{subfigure}[b]{0.47\textwidth}
        \centering
        \tcbincludegraphics[size=fbox,width=1.1\hsize,colframe=red]{figures/Mill_Window_Packets_18.01-19.01.png}
    \end{subfigure}
    \begin{subfigure}[b]{0.47\textwidth}
        \centering
        \tcbincludegraphics[size=fbox,width=1.1\hsize,colframe=blue]{figures/Mill_Window_Baseline_Packets_10.03-11.03.png}
    \end{subfigure}
        \begin{subfigure}[b]{0.47\textwidth}
        \centering
        \tcbincludegraphics[size=fbox,width=1.1\hsize,colframe=red]{figures/Mill_Window_Packets_19.01-20.01.png}
    \end{subfigure}
    \begin{subfigure}[b]{0.47\textwidth}
        \centering
        \tcbincludegraphics[size=fbox,width=1.1\hsize,colframe=blue]{figures/Mill_Window_Baseline_Packets_11.03-12.03.png}
    \end{subfigure}
    \begin{subfigure}[b]{0.47\textwidth}
        \centering
        \tcbincludegraphics[size=fbox,width=1.1\hsize,colframe=red]{figures/Mill_Window_Packets_25.01-26.01.png}
    \end{subfigure}
    \hspace{0.6cm}
    \begin{subfigure}[b]{0.47\textwidth}
    \centering
        \tcbincludegraphics[size=fbox,width=1.1\hsize,colframe=blue]{figures/Mill_Window_Baseline_Packets_12.03-13.03.png}
        \end{subfigure}
    \caption{Graphs of traffic flows from the window open events measured in packets with event graphs framed in red and baseline graphs framed in blue for Mill. Event times are marked in red on the event graphs.}
    \label{fig:MillWindowPackets1}
\end{figure}

\begin{figure}[H]
    \begin{subfigure}[b]{0.45\textwidth}
        \centering
        \tcbincludegraphics[size=fbox,width=1.1\hsize,colframe=red]{figures/Mill_Window_Packets_30.01-31.01.png}
    \end{subfigure}
    \begin{subfigure}[b]{0.45\textwidth}
        \centering
        \tcbincludegraphics[size=fbox,width=1.1\hsize, colframe=blue]{figures/Mill_Window_Baseline_Packets_13.03-14.03.png}
    \end{subfigure}
    \begin{subfigure}[b]{0.45\textwidth}
        \centering
        \tcbincludegraphics[size=fbox,width=1.1\hsize,colframe=red]{figures/Mill_Window_Packets_31.01-01.02.png}
    \end{subfigure}
    \begin{subfigure}[b]{0.45\textwidth}
        \centering
        \tcbincludegraphics[size=fbox,width=1.1\hsize,colframe=blue]{figures/Mill_Window_Baseline_Packets_14.03-15.03.png}
    \end{subfigure}
    \begin{subfigure}[b]{0.45\textwidth}
        \centering
        \tcbincludegraphics[size=fbox,width=1.1\hsize,colframe=red]{figures/Mill_Window_Packets_01.02-02.02.png}
    \end{subfigure}
    \caption{Continuing from Figure \ref{fig:MillWindowPackets1}}
    \label{fig:MillWindowPackets2}
\end{figure}

\begin{figure}[H]
    \begin{subfigure}[b]{0.45\textwidth}
        \centering
        \tcbincludegraphics[size=fbox,width=1.1\hsize,colframe=red]{figures/Mill_Window_Bytes_30.01-31.01.png}
    \end{subfigure}
    \begin{subfigure}[b]{0.45\textwidth}
        \centering
        \tcbincludegraphics[size=fbox,width=1.1\hsize, colframe=blue]{figures/Mill_Window_Baseline_Bytes_13.03-14.03.png}
    \end{subfigure}
    \begin{subfigure}[b]{0.45\textwidth}
        \centering
        \tcbincludegraphics[size=fbox,width=1.1\hsize,colframe=red]{figures/Mill_Window_Bytes_31.01-01.02.png}
    \end{subfigure}
    \begin{subfigure}[b]{0.45\textwidth}
        \centering
        \tcbincludegraphics[size=fbox,width=1.1\hsize,colframe=blue]{figures/Mill_Window_Baseline_Bytes_14.03-15.03.png}
    \end{subfigure}
    \begin{subfigure}[b]{0.45\textwidth}
        \centering
        \tcbincludegraphics[size=fbox,width=1.1\hsize,colframe=red]{figures/Mill_Window_Bytes_01.02-02.02.png}
    \end{subfigure}
    \caption{Remaining graphs from Figure \ref{fig:MillWindowBytes2}}
    \label{fig:MillWindowBytes1}
\end{figure}

\begin{figure}[H]
    \begin{subfigure}[b]{0.47\textwidth}
        \centering
        \tcbincludegraphics[size=fbox,width=1.1\hsize,colframe=red]{figures/Mill_Window_Bytes_08.01-09.01.png}
    \end{subfigure}
    \begin{subfigure}[b]{0.47\textwidth}
        \centering
        \tcbincludegraphics[size=fbox,width=1.1\hsize,colframe=blue]{figures/Mill_Window_Baseline_Bytes_06.03-07.03.png}
    \end{subfigure}
    \begin{subfigure}[b]{0.47\textwidth}
        \centering
        \tcbincludegraphics[size=fbox,width=1.1\hsize,colframe=red]{figures/Mill_Window_Bytes_09.01-10.01.png}
    \end{subfigure}
    \begin{subfigure}[b]{0.47\textwidth}
        \centering
        \tcbincludegraphics[size=fbox,width=1.1\hsize,colframe=blue]{figures/Mill_Window_Baseline_Bytes_07.03-08.03.png}
    \end{subfigure}
    \begin{subfigure}[b]{0.47\textwidth}
        \centering
        \tcbincludegraphics[size=fbox,width=1.1\hsize,colframe=red]{figures/Mill_Window_Bytes_11.01-12.01.png}
    \end{subfigure}
    \begin{subfigure}[b]{0.47\textwidth}
        \centering
        \tcbincludegraphics[size=fbox,width=1.1\hsize,colframe=blue]{figures/Mill_Window_Baseline_Bytes_08.03-09.03.png}
    \end{subfigure}
    \begin{subfigure}[b]{0.47\textwidth}
        \centering
        \tcbincludegraphics[size=fbox,width=1.1\hsize,colframe=red]{figures/Mill_Window_Bytes_16.01-17.01.png}
    \end{subfigure}
    \begin{subfigure}[b]{0.47\textwidth}
        \centering
        \tcbincludegraphics[size=fbox,width=1.1\hsize,colframe=blue]{figures/Mill_Window_Baseline_Bytes_09.03-10.03.png}
    \end{subfigure}
    \begin{subfigure}[b]{0.47\textwidth}
        \centering
        \tcbincludegraphics[size=fbox,width=1.1\hsize,colframe=red]{figures/Mill_Window_Bytes_18.01-19.01.png}
    \end{subfigure}
    \begin{subfigure}[b]{0.47\textwidth}
        \centering
        \tcbincludegraphics[size=fbox,width=1.1\hsize,colframe=blue]{figures/Mill_Window_Baseline_Bytes_10.03-11.03.png}
    \end{subfigure}
        \begin{subfigure}[b]{0.47\textwidth}
        \centering
        \tcbincludegraphics[size=fbox,width=1.1\hsize,colframe=red]{figures/Mill_Window_Bytes_19.01-20.01.png}
    \end{subfigure}
    \begin{subfigure}[b]{0.47\textwidth}
        \centering
        \tcbincludegraphics[size=fbox,width=1.1\hsize,colframe=blue]{figures/Mill_Window_Baseline_Bytes_11.03-12.03.png}
    \end{subfigure}
    \begin{subfigure}[b]{0.47\textwidth}
        \centering
        \tcbincludegraphics[size=fbox,width=1.1\hsize,colframe=red]{figures/Mill_Window_Bytes_25.01-26.01.png}
    \end{subfigure}
    \hspace{0.6cm}
    \begin{subfigure}[b]{0.47\textwidth}
    \centering
        \tcbincludegraphics[size=fbox,width=1.1\hsize,colframe=blue]{figures/Mill_Window_Baseline_Bytes_12.03-13.03.png}
        \end{subfigure}
    \caption{Graphs of traffic flows from the window open events measured in bytes with event graphs framed in red and baseline graphs framed in blue for Mill. Event times are marked in red on the event graphs.}  
    \label{fig:MillWindowBytes2}
\end{figure}

Comparing Table \ref{tab:MillWindowCalculations} and \ref{tab:MillBaselineWindowCalculations} shows that the number of packets sent during the events and baseline do not show a significant difference. The packets both varies from around 40,000 to 60,000 when an event is ongoing, in Table \ref{tab:MillWindowCalculations}, to when an event is not triggered in Table \ref{tab:MillBaselineWindowCalculations}. The same is shown in bytes as it varies from 4,000,000 to 7,000,000 for both baseline and events. The biggest packet sent is also similar in the event and baseline capturings where most of the biggest packets are over 1,000 bytes, but both cases have packets that are also under that, such as 456 bytes for events and 456 and 800 bytes for the baseline. 

In Table \ref{tab:MillWindowCalculations} the average value for packets and byte for the baseline is not smaller than the standard deviation for the window open event. The fact that the calculations are very similar is also shown in Figure \ref{fig:MillWindowCalculations}. The result shows that these calculations cannot be used to identify if a user is showering in the environment through the \gls{AQM} Mill. 

The traffic flows with packets in Figures \ref{fig:MillWindowPackets1} and \ref{fig:MillWindowPackets2} do not show a significant difference between the events on the left side, compared to the baseline packets on the right side of the figure. Both cases displays graphs with variations that reaches the approximately the same level of packets. It is not possible to see changes in the traffic from the red marked area under the events compared to the timings when events are not ongoing. The same result is found for bytes in Figure \ref{fig:MillWindowBytes1} and \ref{fig:MillWindowBytes2}. As the results of the graphs from both the baseline and the event looks familiar, there are no traffic patterns which can identify that a window is open in the environment. 

\newpage
\subsection{Nedis}
For the window open event for Nedis, Table \ref{tab:NedisWindowCalculations} presents calculations for the events, while Table \ref{tab:NedisBaselineWindowCalculations} presents the calculations for the comparing baseline files. Both tables includes the number of packets and bytes sent and the biggest packet captured during the event. In Table \ref{tab:NedisComparingBaselineAndWindowCalculations} the average and standard deviation value for the event and comparing baseline are presented. The values are also presented graphically in Figure \ref{fig:NedisWindowCalculations} for number of packets and bytes including average values. 

\begin{table}[H]
\centering
    \caption{Calculations on window open events for Nedis}
\label{tab:NedisWindowCalculations}
    \begin{tabular}{|l|l|l|l|}
        \hline
        \textbf{Dates} & \textbf{Packets} & \textbf{Bytes} & \textbf{Biggest packet} \\ \hline
        08.jan          & 131,961          & 18,088,388     & 424 bytes               \\ \hline
        09.jan          & 121,273          & 15,204,877     & 424 bytes               \\ \hline
        11.jan          & 94,682           & 9,432,611      & 421 bytes               \\ \hline
        16.jan          & 83,344           & 8,875,795      & 424 bytes               \\ \hline
        18.jan          & 94,106           & 10,706,230     & 458 bytes               \\ \hline
        19.jan          & 97,324           & 12,565,538     & 485 bytes               \\ \hline
        25.jan          & 88,413           & 9,470,205      & 421 bytes               \\ \hline
        30.jan          & 59,874           & 6,939,737      & 424 bytes               \\ \hline
        31.jan          & 96,309           & 13,204,218     & 424 bytes               \\ \hline
        01.feb          & 101,872          & 12,988,077     & 485 bytes               \\ \hline
    \end{tabular}
\end{table}

\begin{table}[H]
    \centering
    \caption{Calculations on comparing baseline files for the window open event for Nedis}
    \begin{tabular}{|l|l|l|l|l|l|}
    \hline
        \textbf{Baseline} & \textbf{Packets} & \textbf{Bytes} & \textbf{Biggest packet} \\ \hline
        06.mar & 77,759  & 8,062,010  & 426 bytes \\ \hline
        07.mar & 105,395 & 12,437,964 & 426 bytes \\ \hline
        08.mar & 109,084 & 13,759,303 & 424 bytes \\ \hline
        09.mar & 111,832 & 15,941,453 & 458 bytes \\ \hline
        10.mar & 115,626 & 15,852,737 & 485 bytes \\ \hline
        11.mar & 58,325  & 7,096,260  & 426 bytes \\ \hline
        12.mar & 89,493  & 11,740,679 & 485 bytes \\ \hline
        13.mar & 89,958  & 11,137,820 & 424 bytes \\ \hline
        14.mar & 91,427  & 11,865,346 & 485 bytes \\ \hline
    \end{tabular}
    \label{tab:NedisBaselineWindowCalculations}
\end{table}

\begin{table}[H]
    \centering
    \caption{Traffic comparison between the window open event and baseline for Nedis}
    \begin{tabular}{c|l|l|l|l|}
        \cline{2-5}
        \multicolumn{1}{l|}{}                                              & \textbf{Type} & \textbf{Packets} & \textbf{Bytes} & \textbf{Biggest packet} \\ \hline
        \multicolumn{1}{|c|}{\multirow{2}{*}{\textbf{Average}}}            & Window open         & 96,916             & 11,767,368      & 439 bytes               \\ \cline{2-5} 
        \multicolumn{1}{|c|}{}                                             & Baseline      & 94,322             & 11,988,175      & 449 bytes                \\ \hline
        \multicolumn{1}{|c|}{\multirow{2}{*}{\textbf{Standard deviation}}} & Window open         & 19,686             & 3,334,883       & 27 bytes                 \\ \cline{2-5} 
        \multicolumn{1}{|c|}{}                                             & Baseline      & 18,445             & 3,042,357       & 29 bytes               \\ \hline          
    \end{tabular}
    \label{tab:NedisComparingBaselineAndWindowCalculations}
\end{table}

\begin{figure}[H]
    \centering
    \begin{subfigure}{0.8\textwidth}
        \centering
        \includegraphics[width=1\hsize]{figures/Nedis_Window_Calculations_Bytes.png} 
    \end{subfigure}
    \begin{subfigure}{0.8\textwidth}
        \centering
        \includegraphics[width=1\hsize]{figures/Nedis_Window_Calculations_Packets.png} 
    \end{subfigure}
    \caption{Graphical presentation of event and baseline window open calculations with packets and bytes, including average value extracted from Table \ref{tab:NedisComparingBaselineAndWindowCalculations} for Nedis}
    \label{fig:NedisWindowCalculations}
\end{figure}

The event is also shown graphically in four different figures. Figures \ref{fig:NedisWindowPackets1} and \ref{fig:NedisWindowPackets2} shows traffic flows with packets and Figures \ref{fig:NedisWindowBytes1} and \ref{fig:NedisWindowBytes2} for bytes. All graphs within the same figure have the equal minimum and maximum values for the x- and y-axis. The event graphs are placed on the left side of the figure and marked in red, while the comparing baseline graphs are placed on the right side of the figure and marked in blue. The event graphs also have a red area marked on the graphs for when the event was ongoing. 

\begin{figure}[H]
    \begin{subfigure}[b]{0.47\textwidth}
        \centering
        \tcbincludegraphics[size=fbox,width=1.1\hsize,colframe=red]{figures/Nedis_Window_Packets_08.01-09.01.png}
    \end{subfigure}
    \begin{subfigure}[b]{0.47\textwidth}
        \centering
        \tcbincludegraphics[size=fbox,width=1.1\hsize, colframe=blue]{figures/Nedis_Window_Baseline_Packets_06.03-07.03.png}
    \end{subfigure}
    \begin{subfigure}[b]{0.47\textwidth}
        \centering
        \tcbincludegraphics[size=fbox,width=1.1\hsize,colframe=red]{figures/Nedis_Window_Packets_09.01-10.01.png}
    \end{subfigure}
    \begin{subfigure}[b]{0.47\textwidth}
        \centering
        \tcbincludegraphics[size=fbox,width=1.1\hsize,colframe=blue]{figures/Nedis_Window_Baseline_Packets_07.03-08.03.png}
    \end{subfigure}
    \begin{subfigure}[b]{0.47\textwidth}
        \centering
        \tcbincludegraphics[size=fbox,width=1.1\hsize,colframe=red]{figures/Nedis_Window_Packets_11.01-12.01.png}
    \end{subfigure}
    \begin{subfigure}[b]{0.47\textwidth}
        \centering
        \tcbincludegraphics[size=fbox,width=1.1\hsize,colframe=blue]{figures/Nedis_Window_Baseline_Packets_08.03-09.03.png}
    \end{subfigure}
    \begin{subfigure}[b]{0.47\textwidth}
        \centering
        \tcbincludegraphics[size=fbox,width=1.1\hsize,colframe=red]{figures/Nedis_Window_Packets_16.01-17.01.png}
    \end{subfigure}
    \begin{subfigure}[b]{0.47\textwidth}
        \centering
        \tcbincludegraphics[size=fbox,width=1.1\hsize,colframe=blue]{figures/Nedis_Window_Baseline_Packets_09.03-10.03.png}
    \end{subfigure}
    \begin{subfigure}[b]{0.47\textwidth}
        \centering
        \tcbincludegraphics[size=fbox,width=1.1\hsize,colframe=red]{figures/Nedis_Window_Packets_18.01-19.01.png}
    \end{subfigure}
    \begin{subfigure}[b]{0.47\textwidth}
        \centering
        \tcbincludegraphics[size=fbox,width=1.1\hsize,colframe=blue]{figures/Nedis_Window_Baseline_Packets_10.03-11.03.png}
    \end{subfigure}
        \begin{subfigure}[b]{0.47\textwidth}
        \centering
        \tcbincludegraphics[size=fbox,width=1.1\hsize,colframe=red]{figures/Nedis_Window_Packets_19.01-20.01.png}
    \end{subfigure}
    \begin{subfigure}[b]{0.47\textwidth}
        \centering
        \tcbincludegraphics[size=fbox,width=1.1\hsize,colframe=blue]{figures/Nedis_Window_Baseline_Packets_11.03-12.03.png}
    \end{subfigure}
    \begin{subfigure}[b]{0.47\textwidth}
        \centering
        \tcbincludegraphics[size=fbox,width=1.1\hsize,colframe=red]{figures/Nedis_Window_Packets_25.01-26.01.png}
    \end{subfigure}
    \hspace{0.6cm}
    \begin{subfigure}[b]{0.47\textwidth}
    \centering
        \tcbincludegraphics[size=fbox,width=1.1\hsize,colframe=blue]{figures/Nedis_Window_Baseline_Packets_12.03-13.03.png}
        \end{subfigure}
    \caption{Graphs of traffic flows from the window open events measured in packets with event graphs framed in red and baseline graphs framed in blue for Nedis. Event times are marked in red on the event graphs.}
    \label{fig:NedisWindowPackets1}
\end{figure}

\begin{figure}[H]
    \begin{subfigure}[b]{0.45\textwidth}
        \centering
        \tcbincludegraphics[size=fbox,width=1.1\hsize,colframe=red]{figures/Nedis_Window_Packets_30.01-31.01.png}
    \end{subfigure}
    \begin{subfigure}[b]{0.45\textwidth}
        \centering
        \tcbincludegraphics[size=fbox,width=1.1\hsize, colframe=blue]{figures/Nedis_Window_Baseline_Packets_13.03-14.03.png}
    \end{subfigure}
    \begin{subfigure}[b]{0.45\textwidth}
        \centering
        \tcbincludegraphics[size=fbox,width=1.1\hsize,colframe=red]{figures/Nedis_Window_Packets_31.01-01.02.png}
    \end{subfigure}
    \begin{subfigure}[b]{0.45\textwidth}
        \centering
        \tcbincludegraphics[size=fbox,width=1.1\hsize,colframe=blue]{figures/Nedis_Window_Baseline_Packets_14.03-15.03.png}
    \end{subfigure}
    \begin{subfigure}[b]{0.45\textwidth}
        \centering
        \tcbincludegraphics[size=fbox,width=1.1\hsize,colframe=red]{figures/Nedis_Window_Packets_01.02-02.02.png}
    \end{subfigure}
    \caption{Continuing from Figure \ref{fig:NedisWindowPackets1}}
    \label{fig:NedisWindowPackets2}
\end{figure}

\begin{figure}[H]
    \begin{subfigure}[b]{0.45\textwidth}
        \centering
        \tcbincludegraphics[size=fbox,width=1.1\hsize,colframe=red]{figures/Nedis_Window_Bytes_30.01-31.01.png}
    \end{subfigure}
    \begin{subfigure}[b]{0.45\textwidth}
        \centering
        \tcbincludegraphics[size=fbox,width=1.1\hsize, colframe=blue]{figures/Nedis_Window_Baseline_Bytes_13.03-14.03.png}
    \end{subfigure}
    \begin{subfigure}[b]{0.45\textwidth}
        \centering
        \tcbincludegraphics[size=fbox,width=1.1\hsize,colframe=red]{figures/Nedis_Window_Bytes_31.01-01.02.png}
    \end{subfigure}
    \begin{subfigure}[b]{0.45\textwidth}
        \centering
        \tcbincludegraphics[size=fbox,width=1.1\hsize,colframe=blue]{figures/Nedis_Window_Baseline_Bytes_14.03-15.03.png}
    \end{subfigure}
    \begin{subfigure}[b]{0.45\textwidth}
        \centering
        \tcbincludegraphics[size=fbox,width=1.1\hsize,colframe=red]{figures/Nedis_Window_Bytes_01.02-02.02.png}
    \end{subfigure}
    \caption{Remaining graphs from Figure \ref{fig:NedisWindowBytes2}}
    \label{fig:NedisWindowBytes1}
\end{figure}

\begin{figure}[H]
    \begin{subfigure}[b]{0.47\textwidth}
        \centering
        \tcbincludegraphics[size=fbox,width=1.1\hsize,colframe=red]{figures/Nedis_Window_Bytes_08.01-09.01.png}
    \end{subfigure}
    \begin{subfigure}[b]{0.47\textwidth}
        \centering
        \tcbincludegraphics[size=fbox,width=1.1\hsize,colframe=blue]{figures/Nedis_Window_Baseline_Bytes_06.03-07.03.png}
    \end{subfigure}
    \begin{subfigure}[b]{0.47\textwidth}
        \centering
        \tcbincludegraphics[size=fbox,width=1.1\hsize,colframe=red]{figures/Nedis_Window_Bytes_09.01-10.01.png}
    \end{subfigure}
    \begin{subfigure}[b]{0.47\textwidth}
        \centering
        \tcbincludegraphics[size=fbox,width=1.1\hsize,colframe=blue]{figures/Nedis_Window_Baseline_Bytes_07.03-08.03.png}
    \end{subfigure}
    \begin{subfigure}[b]{0.47\textwidth}
        \centering
        \tcbincludegraphics[size=fbox,width=1.1\hsize,colframe=red]{figures/Nedis_Window_Bytes_11.01-12.01.png}
    \end{subfigure}
    \begin{subfigure}[b]{0.47\textwidth}
        \centering
        \tcbincludegraphics[size=fbox,width=1.1\hsize,colframe=blue]{figures/Nedis_Window_Baseline_Bytes_08.03-09.03.png}
    \end{subfigure}
    \begin{subfigure}[b]{0.47\textwidth}
        \centering
        \tcbincludegraphics[size=fbox,width=1.1\hsize,colframe=red]{figures/Nedis_Window_Bytes_16.01-17.01.png}
    \end{subfigure}
    \begin{subfigure}[b]{0.47\textwidth}
        \centering
        \tcbincludegraphics[size=fbox,width=1.1\hsize,colframe=blue]{figures/Nedis_Window_Baseline_Bytes_09.03-10.03.png}
    \end{subfigure}
    \begin{subfigure}[b]{0.47\textwidth}
        \centering
        \tcbincludegraphics[size=fbox,width=1.1\hsize,colframe=red]{figures/Nedis_Window_Bytes_18.01-19.01.png}
    \end{subfigure}
    \begin{subfigure}[b]{0.47\textwidth}
        \centering
        \tcbincludegraphics[size=fbox,width=1.1\hsize,colframe=blue]{figures/Nedis_Window_Baseline_Bytes_10.03-11.03.png}
    \end{subfigure}
        \begin{subfigure}[b]{0.47\textwidth}
        \centering
        \tcbincludegraphics[size=fbox,width=1.1\hsize,colframe=red]{figures/Nedis_Window_Bytes_19.01-20.01.png}
    \end{subfigure}
    \begin{subfigure}[b]{0.47\textwidth}
        \centering
        \tcbincludegraphics[size=fbox,width=1.1\hsize,colframe=blue]{figures/Nedis_Window_Baseline_Bytes_11.03-12.03.png}
    \end{subfigure}
    \begin{subfigure}[b]{0.47\textwidth}
        \centering
        \tcbincludegraphics[size=fbox,width=1.1\hsize,colframe=red]{figures/Nedis_Window_Bytes_25.01-26.01.png}
    \end{subfigure}
    \hspace{0.6cm}
    \begin{subfigure}[b]{0.47\textwidth}
    \centering
        \tcbincludegraphics[size=fbox,width=1.1\hsize,colframe=blue]{figures/Nedis_Window_Baseline_Bytes_12.03-13.03.png}
        \end{subfigure}
    \caption{Graphs of traffic flows from the window open events measured in packets with event graphs framed in red and baseline graphs framed in blue for Nedis. Event times are marked in red on the event graphs.}  
    \label{fig:NedisWindowBytes2}
\end{figure}

Comparing Table \ref{tab:NedisWindowCalculations} for events and \ref{tab:NedisBaselineWindowCalculations} for baseline calculations shows that both cases varies a lot in number of packets and bytes during the traffic capture. For both events and baseline, the number of packets varies from around 50,000 to over 100,000 packets, which is a big difference. The same applies to bytes where both varies from around 7,000,000 to over 15,000,000 bytes. There are no significant differences in the biggest packet during the capture, where all days within both cases are between 400 and 500 bytes. 

Even though the number of packets and bytes varies much for each event and baseline day, they vary almost equally as the average value shown in Table \ref{tab:NedisComparingBaselineAndWindowCalculations} are almost the same. The average value for both packets and bytes for the baseline is not outside of the standard deviation for the window open events and cannot be used to distinguish if showering is ongoing or not in the environment. The graphs in Figure \ref{fig:NedisWindowCalculations} also demonstrates how much the packets vary, but still the average values are very close to each other.

For the packet traffic flows presented in Figures \ref{fig:NedisWindowPackets1} and \ref{fig:NedisWindowPackets2} the event graphs do have some more spikes overall compared to the baseline graphs. However, 4 out of 9 of the baseline days also have spikes that are similar to the event days where there are only one spike included. The bytes traffic flows presented in Figures \ref{fig:NedisWindowBytes1} and \ref{fig:NedisWindowBytes2} shows well the variations that were visible in the calculations presented. Both on the left side with events and on the right side of the figure with the baseline traffic there are variations easily visible. Therefore, there are no significant differences in events and baseline traffic as the variations are shown in both cases. This results in that it is not possible to identify if a window is open in the environment compared to standard traffic which is not triggered by the same event. 


\newpage
\section{Test Case 4: Weekends}
This chapter presents the results and analysis for Test Case 4: Weekends. 
\subsection{General}
Test Case 4: Weekends is tested over the course of 14 different weekends. 7 weekends when the environment was occupied, meaning that the user was at home, and 7 when the environment were not occupied, meaning that user was not home during this time. The different dates are described in Table \ref{tab:WeekendDates}. 
\begin{table}[H]
    \centering
    \caption{Dates for Test Case 4: Weekends}
    \begin{adjustbox}{width=0.6\textwidth} 
        \begin{tabular}{l|l|}
            \cline{2-2} & \textbf{Dates}\\ \hline
            \multicolumn{1}{|l|}{\textbf{Weekends at home}} & \begin{tabular}[c]{@{}l@{}}13.01.2023-15.01.2023\\ 27.01.2023-29.01.2023\\ 03.02.2023-05.02.2023\\ 17.02.2023-19.02.2023\\ 10.03.2023-12.03.2023\\ 28.03.2023-30.03.2023\\ 31.03.2023-01.04.2023\end{tabular} \\ \hline
            \multicolumn{1}{|l|}{\textbf{Weekends gone}} & \begin{tabular}[c]{@{}l@{}}23.12.2022-25.12.2022\\ 30.12.2022-01.01.2023\\ 20.01.2023-22.01.2023\\ 10.02.2023-12.02.2023\\ 24.02.2023-26.02.2023\\ 03.03.2023-05.03.2023\\ 17.03.2023-19.03.2023\end{tabular} \\ \hline
        \end{tabular}
    \end{adjustbox}
    \label{tab:WeekendDates}
\end{table}

The times for weekends at home and gone were from 16:00 at Friday to 22:00 at Sunday. In a weekend were the home was occupied, it was variably occupied. Some weekends a lot of time were spent in the home and other weekends it was just occupied during evenings or daytime, but every occupied weekend were spent sleeping at night in the home. One weekend, 28.03.2023-30.03.2023, were tested from Tuesday to Thursday, but are treated as Tuesday=Friday and Thursday=Sunday. 

This results in the following filter used on the pcaps to created the files: 

\begin{itemize}
    \item frame.time >= "Month Date, Year 16:00:00" \&\& frame.time <= "Month Date, Year 22:00:00"
\end{itemize}

\newpage
\subsection{Netatmo}
For the weekend test for Netatmo calculations are listed in Table \ref{tab:NetatmoHomeWeekends}, for weekends at home, and Table \ref{tab:NetatmoGoneWeekends}, for weekends gone. The different values are total number of packets and bytes during the packet capturing and the biggest packet sent in bytes, biggest src, and the biggest packet size in bytes received, biggest dst. Table \ref{tab:NetatmoWeekends} compares the average and \gls{SD} for all weekends. 

\begin{table}[H]
    \centering
    \caption{Calculations for weekends at home for Netatmo}
    \begin{tabular}{|l|l|l|l|l|l|}
    \hline
        \textbf{Date} & \textbf{Packets} & \textbf{Bytes}  & \textbf{Biggest src} & \textbf{Biggest dst} \\ \hline
        13.01-15.01   & 31,640           & 4,285,664       & 407 bytes            & 418 bytes            \\ \hline
        27.01-29.01   & 25,465           & 3,436,632       & 407 bytes            & 154 bytes            \\ \hline
        03.02-05.02   & 24,887           & 3,379,806       & 444 bytes            & 396 bytes            \\ \hline
        17.02-19.02   & 23,654           & 3,202,102       & 1,130 bytes          & 154 bytes            \\ \hline
        10.03-12.03   & 24,881           & 3,372,230       & 407 bytes            & 136 bytes            \\ \hline
        28.03-30.03   & 27,555           & 3,743,121       & 407 bytes            & 154 bytes            \\ \hline
        31.03-02.04   & 28,445           & 3,852,003       & 407 bytes            & 418 bytes            \\ \hline
    \end{tabular}
    \label{tab:NetatmoHomeWeekends}
\end{table}

\begin{table}[H]
    \centering
    \caption{Calculations for weekends gone for Netatmo}
    \begin{tabular}{|l|l|l|l|l|l|}
    \hline
        \textbf{Date} & \textbf{Packets} & \textbf{Bytes} & \textbf{Biggest src} & \textbf{Biggest dst} \\ \hline
        23.12-25.12   & 22,367           & 2,987,324      & 134 bytes            & 136 bytes            \\ \hline
        30.12-01.01   & 22,553           & 3,012,868      & 134 bytes            & 136 bytes            \\ \hline
        20.01-22.01   & 24,631           & 3,288,842      & 134 bytes            & 154 bytes            \\ \hline
        10.02-12.02   & 20,320           & 2,715,486      & 134 bytes            & 136 bytes            \\ \hline
        24.02-26.02   & 21,332           & 2,849,186      & 134 bytes            & 136 bytes            \\ \hline
        03.03-05.03   & 19,023           & 2,538,937      & 134 bytes            & 136 bytes            \\ \hline
        17.03-19.03   & 23,905           & 3,191,924      & 134 bytes            & 136 bytes            \\ \hline
    \end{tabular}
    \label{tab:NetatmoGoneWeekends}
\end{table}

\begin{table}[H]
    \centering
    \caption{Traffic comparison of weekend values from Table \ref{tab:NetatmoHomeWeekends} and \ref{tab:NetatmoGoneWeekends}}
    \begin{tabular}{ll|l|l|l|l|l|}
        \cline{3-6}
        &      & \textbf{Packets} & \textbf{Bytes} & \textbf{Biggest src} & \textbf{Biggest dst} \\ \hline
    \multicolumn{1}{|l|}{\multirow{2}{*}{\textbf{Average}}} & Home & 26,647          & 3,610,223    & 516 bytes                 & 261 bytes                 \\ \cline{2-6} 
    \multicolumn{1}{|l|}{}                              & Gone & 22,019          & 2,940,652    & 134 bytes                    & 139 bytes                 \\ \hline
    \multicolumn{1}{|l|}{\multirow{2}{*}{\textbf{\gls{SD}}}} & Home & 2,756            & 373,892      & 271 bytes                 & 140 bytes                 \\ \cline{2-6} 
    \multicolumn{1}{|l|}{}                             & Gone & 1,963            & 262,109      & 0 bytes                   & 7 bytes                   \\ \hline
    \end{tabular}
    \label{tab:NetatmoWeekends}
\end{table}

Graphs of packets are presented in Figure \ref{fig:NetatmoWeekendPackets} and bytes in Figure \ref{fig:NetatmoWeekendBytes}. In these figures, the weekends at home have a green frame and placed on the left side, while the weekends gone have an orange frame and placed on the right side of the figure. All graphs in the same figure have the same maximum value on the y-axis and have the same timings on the x-axis to be easily comparable. 

\begin{figure} [H]
    \begin{subfigure}[b]{0.47\textwidth}
    \centering
        \tcbincludegraphics[size=fbox,width=1.1\hsize,colframe=green]{figures/Netatmo_Weekend_Packets_13.01-15.01.png}
    \end{subfigure}
    \begin{subfigure}[b]{0.47\textwidth}
    \centering
        \tcbincludegraphics[size=fbox,width=1.1\hsize,colframe=orange]{figures/Netatmo_Weekend_Packets_23.12-25.12.png}
    \end{subfigure}
    \begin{subfigure}[b]{0.47\textwidth}
        \tcbincludegraphics[size=fbox,width=1.1\hsize,colframe=green]{figures/Netatmo_Weekend_Packets_27.01-29.01.png}
    \end{subfigure}
    \begin{subfigure}[b]{0.47\textwidth}
        \tcbincludegraphics[size=fbox,width=1.1\hsize,colframe=orange]{figures/Netatmo_Weekend_Packets_30.12-01.01.png}
    \end{subfigure}
    \begin{subfigure}[b]{0.47\textwidth}
        \tcbincludegraphics[size=fbox,width=1.1\hsize,colframe=green]{figures/Netatmo_Weekend_Packets_03.02-05.02.png}
    \end{subfigure}    
    \begin{subfigure}[b]{0.47\textwidth}
        \tcbincludegraphics[size=fbox,width=1.1\hsize,colframe=orange]{figures/Netatmo_Weekend_Packets_20.01-22.01.png}
    \end{subfigure}
    \begin{subfigure}[b]{0.47\textwidth}
        \tcbincludegraphics[size=fbox,width=1.1\hsize,colframe=green]{figures/Netatmo_Weekend_Packets_17.02-19.02.png}
    \end{subfigure}    
    \begin{subfigure}[b]{0.47\textwidth}
        \tcbincludegraphics[size=fbox,width=1.1\hsize,colframe=orange]{figures/Netatmo_Weekend_Packets_10.02-12.02.png}
    \end{subfigure}
    \begin{subfigure}[b]{0.47\textwidth}
        \tcbincludegraphics[size=fbox,width=1.1\hsize,colframe=green]{figures/Netatmo_Weekend_Packets_10.03-12.03.png}
    \end{subfigure}    
    \begin{subfigure}[b]{0.47\textwidth}
        \tcbincludegraphics[size=fbox,width=1.1\hsize,colframe=orange]{figures/Netatmo_Weekend_Packets_24.02-26.02.png}
    \end{subfigure}
    \begin{subfigure}[b]{0.47\textwidth}
        \tcbincludegraphics[size=fbox,width=1.1\hsize,colframe=green]{figures/Netatmo_Weekend_Packets_28.03-30.03.png}
    \end{subfigure}
    \begin{subfigure}[b]{0.47\textwidth}
        \tcbincludegraphics[size=fbox,width=1.1\hsize,colframe=orange]{figures/Netatmo_Weekend_Packets_03.03-05.03.png}
    \end{subfigure}
    \begin{subfigure}[b]{0.47\textwidth}
        \tcbincludegraphics[size=fbox,width=1.1\hsize,colframe=green]{figures/Netatmo_Weekend_Packets_31.03-02.04.png}
    \end{subfigure}
    \hspace{0.6cm}
    \begin{subfigure}[b]{0.47\textwidth}
        \tcbincludegraphics[size=fbox,width=1.1\hsize,colframe=orange]{figures/Netatmo_Weekend_Packets_17.03-19.03.png}
    \end{subfigure}
    \caption{Traffic patterns for weekends at home, marked in green, and gone, marked in orange, measured in total amount of packets sent and received for Netatmo during the weekends}
    \label{fig:NetatmoWeekendPackets}
\end{figure}

\begin{figure}[H]
    \begin{subfigure}[b]{0.47\textwidth}
    \centering
        \tcbincludegraphics[size=fbox,width=1.1\hsize,colframe=green]{figures/Netatmo_Weekend_Bytes_13.01-15.01.png}
    \end{subfigure}
    \begin{subfigure}[b]{0.47\textwidth}
    \centering
        \tcbincludegraphics[size=fbox,width=1.1\hsize,colframe=orange]{figures/Netatmo_Weekend_Bytes_23.12-25.12.png}
    \end{subfigure}
    \begin{subfigure}[b]{0.47\textwidth}
        \tcbincludegraphics[size=fbox,width=1.1\hsize,colframe=green]{figures/Netatmo_Weekend_Bytes_27.01-29.01.png}
    \end{subfigure}
    \begin{subfigure}[b]{0.47\textwidth}
        \tcbincludegraphics[size=fbox,width=1.1\hsize,colframe=orange]{figures/Netatmo_Weekend_Bytes_30.12-01.01.png}
    \end{subfigure}
    \begin{subfigure}[b]{0.47\textwidth}
        \tcbincludegraphics[size=fbox,width=1.1\hsize,colframe=green]{figures/Netatmo_Weekend_Bytes_03.02-05.02.png}
    \end{subfigure}
    \begin{subfigure}[b]{0.47\textwidth}
        \tcbincludegraphics[size=fbox,width=1.1\hsize,colframe=orange]{figures/Netatmo_Weekend_Bytes_20.01-22.01.png}
    \end{subfigure}
    \begin{subfigure}[b]{0.47\textwidth}
        \tcbincludegraphics[size=fbox,width=1.1\hsize,colframe=green]{figures/Netatmo_Weekend_Bytes_17.02-19.02.png}
    \end{subfigure}
    \begin{subfigure}[b]{0.47\textwidth}
        \tcbincludegraphics[size=fbox,width=1.1\hsize,colframe=orange]{figures/Netatmo_Weekend_Bytes_10.02-12.02.png}
    \end{subfigure}
    \begin{subfigure}[b]{0.47\textwidth}
        \tcbincludegraphics[size=fbox,width=1.1\hsize,colframe=green]{figures/Netatmo_Weekend_Bytes_10.03-12.03.png}
    \end{subfigure}
    \begin{subfigure}[b]{0.47\textwidth}
        \tcbincludegraphics[size=fbox,width=1.1\hsize,colframe=orange]{figures/Netatmo_Weekend_Bytes_24.02-26.02.png}
    \end{subfigure}
    \begin{subfigure}[b]{0.47\textwidth}
        \tcbincludegraphics[size=fbox,width=1.1\hsize,colframe=green]{figures/Netatmo_Weekend_Bytes_28.03-30.03.png}
    \end{subfigure}
    \begin{subfigure}[b]{0.47\textwidth}
        \tcbincludegraphics[size=fbox,width=1.1\hsize,colframe=orange]{figures/Netatmo_Weekend_Bytes_03.03-05.03.png}
    \end{subfigure}
   \begin{subfigure}[b]{0.47\textwidth}
        \tcbincludegraphics[size=fbox,width=1.1\hsize,colframe=green]{figures/Netatmo_Weekend_Bytes_31.03-02.04.png}
    \end{subfigure}
    \hspace{0.6cm}
    \begin{subfigure}[b]{0.47\textwidth}
        \tcbincludegraphics[size=fbox,width=1.1\hsize,colframe=orange]{figures/Netatmo_Weekend_Bytes_17.03-19.03.png}
    \end{subfigure}
    \caption{Traffic patterns for weekends at home, marked in green, and gone, marked in orange, measured in total amount of bytes sent and received for Netatmo during the weekends}
    \label{fig:NetatmoWeekendBytes}
\end{figure}

The calculations for weekends show a significant difference from when the home is occupied to when its not. Comparing the total packets between Table \ref{tab:NetatmoHomeWeekends} and \ref{tab:NetatmoGoneWeekends} does not show a big difference with 26,647 packets at home to 22,019 packets when gone in average. However, in bytes the difference is more significant. The difference of the average value for weekends at home compared to gone is nearly 500,000 bytes, as shown in Table \ref{tab:NetatmoWeekends}. It is also shown in the table that the average value for packets and bytes for the weekends gone, are smaller than the standard deviation than for the weekends at home. Therefore, we can conclude that there are differences in the number of packets and bytes sent when a user is home or not and can be used to infer private information.

Another difference is in the biggest packet sent in the "biggest src" column. For the weekends gone, in Table \ref{tab:NetatmoGoneWeekends}, the biggest packet sent is only 134 bytes, while for the weekends at home in Table \ref{tab:NetatmoHomeWeekends}, the biggest packet sent is always much larger than 134 bytes with a variation from 407 bytes to 1,130 bytes. Since the biggest packet overall for gone weekend are 154 bytes, this will be used as a reference to look at differences to packets sent while at home. 

The graphs in Figure \ref{fig:NetatmoWeekendPackets} with packets shows small differences in home and gone activity. The home graphs have more spikes, but there are also some graphs for gone that have small spikes and look similar to the home graphs. However, for the graphs measured in bytes in Figure \ref{fig:NetatmoWeekendBytes} it is a clear difference to when the home is occupied or not. All the graphs for weekends at home, marked in green, shows a very different traffic pattern than the graphs for the weekends gone, marked in orange. This corresponds to the findings from the calculations in Table \ref{tab:NetatmoHomeWeekends} and \ref{tab:NetatmoGoneWeekends}.

Since it is possible to see a difference in biggest packet during weekends at home and gone, it is also interesting to see if these bigger packets only occur a few times during the weekend or if it is sending larger packets during the whole weekend. Therefore, a new filter have been applied to the pcaps to see how many packets during a weekend at home is actually bigger than the traffic sent on a weekend gone. A filter filtering on only packets larger then 154 bytes have been applied to the pcaps, and have the following format:

\begin{itemize}
    \item frame.len > 154
\end{itemize}

Table \ref{tab:NetatmoBigPackets} shows the results from the filtering. The percentage of packets that are over 154 bytes in Table \ref{tab:NetatmoBigPackets} are very small. A graphical view is therefore presented in Figure \ref{fig:NetatmoBigPackets} to see if the amount is enough to see differences from the weekends gone which does not have any packets over 154 bytes. The graphs in Figure \ref{fig:NetatmoBigPackets} shows that it is possible to see if a home is occupied by looking at the packet sizes that are sent. 

\begin{table}[H]
    \centering
    \caption{Amount of packets bigger than 154 bytes sent on weekends at home for Netatmo}
    \begin{tabular}{|l|l|l|}
        \hline
        \textbf{Dates}   & \textbf{Packets} & \textbf{Percentage} \\ \hline
        13.01-15.01      & 251              & 0.8\%               \\ \hline
        27.01-29.01      & 194              & 0.8\%               \\ \hline
        03.02-05.02      & 271              & 1.1\%               \\ \hline
        17.02-19.02      & 171              & 0.7\%               \\ \hline
        10.03-12.03      & 224              & 0.9\%               \\ \hline
        28.03-30.03      & 446              & 1.6\%               \\ \hline
        31.03-02.04      & 275              & 1\%                 \\ \hline
        \textbf{Average} & \textbf{262}     & \textbf{0.99\%}     \\ \hline
        \textbf{\gls{SD}} & \textbf{90}      & \textbf{0.30\%}     \\ \hline
    \end{tabular}
    \label{tab:NetatmoBigPackets}
\end{table}

\begin{figure}[H]
    \begin{subfigure}[b]{0.47\textwidth}
    \centering
        \includegraphics[width=1.1\hsize]{figures/Netatmo_Weekend_BigBytes_13.01-15.01.png}
    \end{subfigure}
    \begin{subfigure}[b]{0.47\textwidth}
        \includegraphics[width=1.1\hsize]{figures/Netatmo_Weekend_BigBytes_27.01-29.01.png}
    \end{subfigure}
    \begin{subfigure}[b]{0.47\textwidth}
        \includegraphics[width=1.1\hsize]{figures/Netatmo_Weekend_BigBytes_03.02-05.02.png}
    \end{subfigure}
    \begin{subfigure}[b]{0.47\textwidth}
        \includegraphics[width=1.1\hsize]{figures/Netatmo_Weekend_BigBytes_17.02-19.02.png}
    \end{subfigure}
    \begin{subfigure}[b]{0.47\textwidth}
        \includegraphics[width=1.1\hsize]{figures/Netatmo_Weekend_BigBytes_10.03-12.03.png}
    \end{subfigure}
    \begin{subfigure}[b]{0.47\textwidth}
        \includegraphics[width=1.1\hsize]{figures/Netatmo_Weekend_BigBytes_28.03-30.03.png}
    \end{subfigure}
    \begin{subfigure}[b]{0.47\textwidth}
        \includegraphics[width=1.1\hsize]{figures/Netatmo_Weekend_BigBytes_31.03-02.04.png}
    \end{subfigure}
    \caption{Traffic patterns for weekends at home with filtered with packets over 154 bytes for Netatmo}
    \label{fig:NetatmoBigPackets}
\end{figure}

The last evaluation on Netatmo weekend events is to compare the graphs for bytes in Figure \ref{fig:NetatmoWeekendBytes} against values from the application of the device. This is presented in Figure \ref{fig:NetatmoWeekendBytesandApp1} and \ref{fig:NetatmoWeekendBytesandApp2}. The graphs are extracted from the app, Healthy Home Coach, and the sensor values are in \(CO_2\). The y-axis, which are measured in ppm are not equal for each of the graphs from the app, as this is not possible to change manually in the app. The x-axis is given in time. 

The Figures in \ref{fig:NetatmoWeekendBytesandApp1} and \ref{fig:NetatmoWeekendBytesandApp2} shows the same pattern as for the bytes from the traffic captures on the left side of the figure. When the home is not occupied, the \(CO_2\) value does not change much which the corresponding traffic flow in the byte-graphs shows, and when the home is occupied, the \(CO_2\) levels varies a lot and the corresponding traffic flow also varies a lot. It is therefore shown in this sub chapter that it is possible to infer whether a user is home or not by looking at the traffic pattern both graphically and with calculations. 

\begin{figure}[H]
    \begin{subfigure}[b]{0.47\textwidth}
    \centering
        \tcbincludegraphics[size=fbox,width=1.1\hsize,height=0.15\vsize,colframe=green]{figures/Netatmo_Weekend_Bytes_13.01-15.01.png}
    \end{subfigure}
    \begin{subfigure}[b]{0.47\textwidth}
    \centering
        \tcbincludegraphics[size=fbox,width=1.1\hsize,height=0.15\vsize,colframe=green]{figures/Netatmo_Weekend_App_13.01-15.01.png}
    \end{subfigure}
    \begin{subfigure}[b]{0.47\textwidth}
    \centering
        \tcbincludegraphics[size=fbox,width=1.1\hsize,height=0.15\vsize,colframe=orange]{figures/Netatmo_Weekend_Bytes_23.12-25.12.png}
    \end{subfigure}
    \begin{subfigure}[b]{0.47\textwidth}
    \centering
        \tcbincludegraphics[size=fbox,width=1.1\hsize,height=0.15\vsize,colframe=orange]{figures/Netatmo_Weekend_App_23.12-25.12.png}
    \end{subfigure}
    \begin{subfigure}[b]{0.47\textwidth}
        \tcbincludegraphics[size=fbox,width=1.1\hsize,height=0.15\vsize,colframe=green]{figures/Netatmo_Weekend_Bytes_27.01-29.01.png}
    \end{subfigure}
    \begin{subfigure}[b]{0.47\textwidth}
        \tcbincludegraphics[size=fbox,width=1.1\hsize,height=0.15\vsize,colframe=green]{figures/Netatmo_Weekend_App_27.01-29.01.png}
    \end{subfigure}
    \begin{subfigure}[b]{0.47\textwidth}
        \tcbincludegraphics[size=fbox,width=1.1\hsize,height=0.15\vsize,colframe=orange]{figures/Netatmo_Weekend_Bytes_30.12-01.01.png}
    \end{subfigure}
    \begin{subfigure}[b]{0.47\textwidth}
        \tcbincludegraphics[size=fbox,width=1.1\hsize,height=0.15\vsize,colframe=orange]{figures/Netatmo_Weekend_App_30.12-01.01.png}
    \end{subfigure}    
    \begin{subfigure}[b]{0.47\textwidth}
        \tcbincludegraphics[size=fbox,width=1.1\hsize,height=0.15\vsize,colframe=green]{figures/Netatmo_Weekend_Bytes_03.02-05.02.png}
    \end{subfigure}
    \begin{subfigure}[b]{0.47\textwidth}
        \tcbincludegraphics[size=fbox,width=1.1\hsize,height=0.15\vsize,colframe=green]{figures/Netatmo_Weekend_App_03.02-05.02.png}
    \end{subfigure}
    \begin{subfigure}[b]{0.47\textwidth}
        \tcbincludegraphics[size=fbox,width=1.1\hsize,height=0.15\vsize,colframe=orange]{figures/Netatmo_Weekend_Bytes_20.01-22.01.png}
    \end{subfigure}
    \begin{subfigure}[b]{0.47\textwidth}
        \tcbincludegraphics[size=fbox,width=1.1\hsize,height=0.15\vsize,colframe=orange]{figures/Netatmo_Weekend_App_20.01-22.01.png}
    \end{subfigure}    
    \begin{subfigure}[b]{0.47\textwidth}
        \tcbincludegraphics[size=fbox,width=1.1\hsize,height=0.15\vsize,colframe=green]{figures/Netatmo_Weekend_Bytes_17.02-19.02.png}
    \end{subfigure}
    \hspace{0.6cm}
    \begin{subfigure}[b]{0.47\textwidth}
        \tcbincludegraphics[size=fbox,width=1.1\hsize,height=0.15\vsize,colframe=green]{figures/Netatmo_Weekend_App_17.02-19.02.png}
    \end{subfigure}
    \caption{Traffic flows in bytes for weekends at home, framed in green, and gone, framed in orange, compared to CO2 values from the corresponding dates in the application, Healthy Home Coach}
    \label{fig:NetatmoWeekendBytesandApp1}
\end{figure} 
\begin{figure}[H]
    \begin{subfigure}[b]{0.47\textwidth}
        \tcbincludegraphics[size=fbox,width=1.1\hsize,height=0.15\vsize,colframe=orange]{figures/Netatmo_Weekend_Bytes_10.02-12.02.png}
    \end{subfigure}
    \begin{subfigure}[b]{0.47\textwidth}
        \tcbincludegraphics[size=fbox,width=1.1\hsize,height=0.15\vsize,colframe=orange]{figures/Netatmo_Weekend_App_10.02-12.02.png}
    \end{subfigure}
    \begin{subfigure}[b]{0.47\textwidth}
        \tcbincludegraphics[size=fbox,width=1.1\hsize,height=0.15\vsize,colframe=green]{figures/Netatmo_Weekend_Bytes_10.03-12.03.png}
    \end{subfigure}
    \begin{subfigure}[b]{0.47\textwidth}
        \tcbincludegraphics[size=fbox,width=1.1\hsize,height=0.15\vsize,colframe=green]{figures/Netatmo_Weekend_App_10.03-12.03.png}
    \end{subfigure}
    \begin{subfigure}[b]{0.47\textwidth}
        \tcbincludegraphics[size=fbox,width=1.1\hsize,height=0.15\vsize,colframe=orange]{figures/Netatmo_Weekend_Bytes_24.02-26.02.png}
    \end{subfigure}
    \begin{subfigure}[b]{0.47\textwidth}
        \tcbincludegraphics[size=fbox,width=1.1\hsize,height=0.15\vsize,colframe=orange]{figures/Netatmo_Weekend_App_24.02-26.02.png}
    \end{subfigure}
    \begin{subfigure}[b]{0.47\textwidth}
        \tcbincludegraphics[size=fbox,width=1.1\hsize,height=0.15\vsize,colframe=green]{figures/Netatmo_Weekend_Bytes_28.03-30.03.png}
    \end{subfigure}
    \begin{subfigure}[b]{0.47\textwidth}
        \tcbincludegraphics[size=fbox,width=1.1\hsize,height=0.15\vsize,colframe=green]{figures/Netatmo_Weekend_App_28.03-30.03.png}
    \end{subfigure}
    \begin{subfigure}[b]{0.47\textwidth}
        \tcbincludegraphics[size=fbox,width=1.1\hsize,height=0.15\vsize,colframe=orange]{figures/Netatmo_Weekend_Bytes_03.03-05.03.png}
    \end{subfigure}
    \begin{subfigure}[b]{0.47\textwidth}
        \tcbincludegraphics[size=fbox,width=1.1\hsize,height=0.15\vsize,colframe=orange]{figures/Netatmo_Weekend_App_03.03-05.03.png}
    \end{subfigure}
   \begin{subfigure}[b]{0.47\textwidth}
        \tcbincludegraphics[size=fbox,width=1.1\hsize,height=0.15\vsize,colframe=green]{figures/Netatmo_Weekend_Bytes_31.03-02.04.png}
    \end{subfigure}
    \begin{subfigure}[b]{0.47\textwidth}
        \tcbincludegraphics[size=fbox,width=1.1\hsize,height=0.15\vsize,colframe=green]{figures/Netatmo_Weekend_App_31.03-02.04.png}
    \end{subfigure}
    \begin{subfigure}[b]{0.47\textwidth}
        \tcbincludegraphics[size=fbox,width=1.1\hsize,height=0.15\vsize,colframe=orange]{figures/Netatmo_Weekend_Bytes_17.03-19.03.png}
    \end{subfigure}
    \hspace{0.6cm}
    \begin{subfigure}[b]{0.47\textwidth}
        \tcbincludegraphics[size=fbox,width=1.1\hsize,height=0.15\vsize,colframe=orange]{figures/Netatmo_Weekend_App_17.03-19.03.png}
    \end{subfigure}
    \caption{Remaining graphs from Figure \ref{fig:NetatmoWeekendBytesandApp1}}
    \label{fig:NetatmoWeekendBytesandApp2}
\end{figure}

\subsection{Mill}
For weekend testing with the device, Mill, Table \ref{tab:MillHomeWeekends} and \ref{tab:MillGoneWeekends} presents the calculations for the weekends at home and gone. The values presented are total number of packets and bytes sent and received and the biggest packet sent, biggest src, and received, biggest dst. Average and standard deviation values from the tables are combined and compared in Table \ref{tab:MillWeekends}. 

\begin{table}[H]
    \centering
    \caption{Calculations for weekends at home for Mill}
    \begin{tabular}{|l|l|l|l|l|}
        \hline
        \textbf{Date}    & \textbf{Packets} & \textbf{Bytes} & \textbf{Biggest src} & \textbf{Biggest dst} \\ \hline
        13.01-15.01      & 285,543            & 33,269,620           & 456 bytes            & 1,593 bytes          \\ \hline
        27.01-29.01      & 240,434            & 29,179,764           & 456 bytes            & 1,593 bytes          \\ \hline
        03.02-05.02      & 223,095            & 28,543,539           & 456 bytes            & 1,593 bytes          \\ \hline
        17.02-19.02      & 237,964            & 25,637,948           & 456 bytes            & 1,583 bytes          \\ \hline
        10.03-12.03      & 295,395            & 31,161,679           & 456 bytes            & 1,593 bytes          \\ \hline
        28.03-30.03      & 342,001            & 39,951,155           & 456 bytes            & 421 bytes            \\ \hline
        31.03-02.04      & 275,456            & 30,155,191           & 456 bytes            & 421 bytes            \\ \hline
    \end{tabular}
    \label{tab:MillHomeWeekends}
\end{table}

\begin{table}[H]
    \centering
    \caption{Calculations for weekends gone for Mill}
    \begin{tabular}{|l|l|l|l|l|}
        \hline
        \textbf{Date}    & \textbf{Packets} & \textbf{Bytes} & \textbf{Biggest src} & \textbf{Biggest dst} \\ \hline
        23.12-25.12      & 259,516            & 34,066,162           & 456 bytes            & 421 bytes            \\ \hline
        30.12-01.01      & 255,487            & 32,809,172           & 456 bytes            & 421 bytes            \\ \hline
        20.01-22.01      & 256,455            & 29,417,721           & 456 bytes            & 421 bytes            \\ \hline
        10.02-12.02      & 270,718            & 30,482,741           & 456 bytes            & 421 bytes            \\ \hline
        24.02-26.02      & 218,536            & 25,525,312           & 456 bytes            & 421 bytes            \\ \hline
        03.03-05.03      & 236,655            & 23,064,715           & 456 bytes            & 421 bytes            \\ \hline
        17.03-19.03      & 278,146            & 28,612,487           & 456 bytes            & 421 bytes            \\ \hline
    \end{tabular}
    \label{tab:MillGoneWeekends}
\end{table}

\begin{table}[H]
    \centering
    \caption{Traffic comparison of weekend values from Table \ref{tab:MillHomeWeekends} and \ref{tab:MillGoneWeekends}}
    \begin{tabular}{ll|l|l|l|l|}
        \cline{3-6}
        \textbf{}                                           & \textbf{} & \textbf{Packets} & \textbf{Bytes} & \textbf{Biggest src} & \textbf{Biggest dst} \\ \hline
        \multicolumn{1}{|l|}{\multirow{2}{*}{\textbf{Average}}} & Home      & 271,413          & 31,128,414      & 456 bytes            & 1,257 bytes          \\ \cline{2-6} 
        \multicolumn{1}{|l|}{}                              & Gone      & 253,645          & 29,139,759      & 456 bytes            & 421 bytes            \\ \hline
        \multicolumn{1}{|l|}{\multirow{2}{*}{\textbf{\gls{SD}}}}  & Home      & 41,205           & 4,546,021      & 0 bytes              & 571 bytes            \\ \cline{2-6} 
        \multicolumn{1}{|l|}{}                              & Gone      & 20,224           & 3,870,041       & 0 bytes              & 0 bytes              \\ \hline
    \end{tabular}
    \label{tab:MillWeekends}
\end{table}

Graphs of traffic flows are presented in Figure \ref{fig:MillWeekendBytes} for bytes and in Figure \ref{fig:MillWeekendPackets} for packets. The graphs marked in green on the figures are weekends at home and are placed on the left side of the figures, while the graphs marked in orange on the figures are weekends gone and placed on the right side of the figures. The graphs in the same figure all have the same maximum values for the y-axis to be easily comparable.

\begin{figure} [H]
    \begin{subfigure}[b]{0.47\textwidth}
    \centering
        \tcbincludegraphics[size=fbox,width=1.1\hsize,colframe=green]{figures/Mill_Weekend_Packets_13.01-15.01.png}    
    \end{subfigure}
    \begin{subfigure}[b]{0.47\textwidth}
    \centering
        \tcbincludegraphics[size=fbox,width=1.1\hsize,colframe=orange]{figures/Mill_Weekend_Packets_23.12-25.12.png}
    \end{subfigure}
    \begin{subfigure}[b]{0.47\textwidth}
        \tcbincludegraphics[size=fbox,width=1.1\hsize,colframe=green]{figures/Mill_Weekend_Packets_27.01-29.01.png}
    \end{subfigure}
    \begin{subfigure}[b]{0.47\textwidth}
        \tcbincludegraphics[size=fbox,width=1.1\hsize,colframe=orange]{figures/Mill_Weekend_Packets_30.12-01.01.png}
    \end{subfigure}
    \begin{subfigure}[b]{0.47\textwidth}
        \tcbincludegraphics[size=fbox,width=1.1\hsize,colframe=green]{figures/Mill_Weekend_Packets_03.02-05.02.png}
    \end{subfigure}    
    \begin{subfigure}[b]{0.47\textwidth}
        \tcbincludegraphics[size=fbox,width=1.1\hsize,colframe=orange]{figures/Mill_Weekend_Packets_20.01-22.01.png}
    \end{subfigure}
    \begin{subfigure}[b]{0.47\textwidth}
        \tcbincludegraphics[size=fbox,width=1.1\hsize,colframe=green]{figures/Mill_Weekend_Packets_17.02-19.02.png}
    \end{subfigure}    
    \begin{subfigure}[b]{0.47\textwidth}
        \tcbincludegraphics[size=fbox,width=1.1\hsize,colframe=orange]{figures/Mill_Weekend_Packets_10.02-12.02.png}
    \end{subfigure}
    \begin{subfigure}[b]{0.47\textwidth}
        \tcbincludegraphics[size=fbox,width=1.1\hsize,colframe=green]{figures/Mill_Weekend_Packets_10.03-12.03.png}
    \end{subfigure}    
    \begin{subfigure}[b]{0.47\textwidth}
        \tcbincludegraphics[size=fbox,width=1.1\hsize,colframe=orange]{figures/Mill_Weekend_Packets_24.02-26.02.png}
    \end{subfigure}
    \begin{subfigure}[b]{0.47\textwidth}
        \tcbincludegraphics[size=fbox,width=1.1\hsize,colframe=green]{figures/Mill_Weekend_Packets_28.03-30.03.png}
    \end{subfigure}
    \begin{subfigure}[b]{0.47\textwidth}
        \tcbincludegraphics[size=fbox,width=1.1\hsize,colframe=orange]{figures/Mill_Weekend_Packets_03.03-05.03.png}
    \end{subfigure}
    \begin{subfigure}[b]{0.47\textwidth}
        \tcbincludegraphics[size=fbox,width=1.1\hsize,colframe=green]{figures/Mill_Weekend_Packets_31.03-02.04.png}
    \end{subfigure}
    \hspace{0.6cm}
    \begin{subfigure}[b]{0.47\textwidth}
        \tcbincludegraphics[size=fbox,width=1.1\hsize,colframe=orange]{figures/Mill_Weekend_Packets_17.03-19.03.png}
    \end{subfigure}
    \caption{Traffic patterns for weekends at home, marked in green, and gone, marked in orange, measured in total amount of packets sent and received for Mill during the weekends}
    \label{fig:MillWeekendPackets}
\end{figure}

\begin{figure}[H]
    \begin{subfigure}[b]{0.47\textwidth}
    \centering
        \tcbincludegraphics[size=fbox,width=1.1\hsize,colframe=green]{figures/Mill_Weekend_Bytes_13.01-15.01.png}
    \end{subfigure}
    \begin{subfigure}[b]{0.47\textwidth}
    \centering
        \tcbincludegraphics[size=fbox,width=1.1\hsize,colframe=orange]{figures/Mill_Weekend_Bytes_23.12-25.12.png}
    \end{subfigure}
    \begin{subfigure}[b]{0.47\textwidth}
        \tcbincludegraphics[size=fbox,width=1.1\hsize,colframe=green]{figures/Mill_Weekend_Bytes_27.01-29.01.png}
    \end{subfigure}
    \begin{subfigure}[b]{0.47\textwidth}
        \tcbincludegraphics[size=fbox,width=1.1\hsize,colframe=orange]{figures/Mill_Weekend_Bytes_30.12-01.01.png}
    \end{subfigure}
    \begin{subfigure}[b]{0.47\textwidth}
        \tcbincludegraphics[size=fbox,width=1.1\hsize,colframe=green]{figures/Mill_Weekend_Bytes_03.02-05.02.png}
    \end{subfigure}
    \begin{subfigure}[b]{0.47\textwidth}
        \tcbincludegraphics[size=fbox,width=1.1\hsize,colframe=orange]{figures/Mill_Weekend_Bytes_20.01-22.01.png}
    \end{subfigure}
    \begin{subfigure}[b]{0.47\textwidth}
        \tcbincludegraphics[size=fbox,width=1.1\hsize,colframe=green]{figures/Mill_Weekend_Bytes_17.02-19.02.png}
    \end{subfigure}
    \begin{subfigure}[b]{0.47\textwidth}
        \tcbincludegraphics[size=fbox,width=1.1\hsize,colframe=orange]{figures/Mill_Weekend_Bytes_10.02-12.02.png}
    \end{subfigure}
    \begin{subfigure}[b]{0.47\textwidth}
        \tcbincludegraphics[size=fbox,width=1.1\hsize,colframe=green]{figures/Mill_Weekend_Bytes_10.03-12.03.png}
    \end{subfigure}
    \begin{subfigure}[b]{0.47\textwidth}
        \tcbincludegraphics[size=fbox,width=1.1\hsize,colframe=orange]{figures/Mill_Weekend_Bytes_24.02-26.02.png}
    \end{subfigure}
    \begin{subfigure}[b]{0.47\textwidth}
        \tcbincludegraphics[size=fbox,width=1.1\hsize,colframe=green]{figures/Mill_Weekend_Bytes_28.03-30.03.png}
    \end{subfigure}
    \begin{subfigure}[b]{0.47\textwidth}
        \tcbincludegraphics[size=fbox,width=1.1\hsize,colframe=orange]{figures/Mill_Weekend_Bytes_03.03-05.03.png}
    \end{subfigure}
   \begin{subfigure}[b]{0.47\textwidth}
        \tcbincludegraphics[size=fbox,width=1.1\hsize,colframe=green]{figures/Mill_Weekend_Bytes_31.03-02.04.png}
    \end{subfigure}
    \hspace{0.6cm}
    \begin{subfigure}[b]{0.47\textwidth}
        \tcbincludegraphics[size=fbox,width=1.1\hsize,colframe=orange]{figures/Mill_Weekend_Bytes_17.03-19.03.png}
    \end{subfigure}
    \caption{Traffic patterns for weekends at home, marked in green, and gone, marked in orange, measured in total amount of bytes sent and received for Mill during the weekends}
    \label{fig:MillWeekendBytes}
\end{figure}

Looking at the values in Table \ref{tab:MillHomeWeekends} and \ref{tab:MillGoneWeekends}, the total packets and bytes remains the same whether or not it's a weekend at home or gone. In regards of the largest packets, all the weekends gone have the same size; 456 bytes for packets sent from the device and 421 bytes for packets sent to the device. For the first 5 weekends at home it looks like there's a pattern change that the device receives much bigger packets than during a weekend gone with a size of 1,583 and 1,593 bytes, but as the last two weekends at home shows their biggest packet is actually the same as for a weekend gone. 

The values in Table \ref{tab:MillWeekends} shows that the average value for packets and bytes for the weekends gone is not smaller than the standard deviation for the weekends at home. Therefore, these calculations cannot be used to infer whether a user is home or gone. 

The graphs in Figure \ref{fig:MillWeekendPackets} and \ref{fig:MillWeekendBytes} shows that the traffic pattern for weekend at home or gone look the same. There are variations in the both weekend at home or gone, but constant pattern which distinguishes weekends at home from gone. This means that is it not possible to distinguish whether a user is home or not by looking at the traffic patterns for Mill.  

\newpage
\subsection{Nedis}
For Weekend testing on the device Nedis, tables for calculations and figures for graphs are presented to analyze the traffic patterns. In Table \ref{tab:NedisHomeWeekends} and \ref{tab:NedisGoneWeekends} the calculations for weekends at home and gone are presented. The values included are total amount of packets, total number of bytes, the biggest packet sent, biggest src, and the biggest packet received, biggest dst. Table \ref{tab:NedisWeekends} compares the average and \gls{SD} values for all the weekends at home and weekends gone.

\begin{table}[H]
    \centering
    \caption{Calculations for weekends at home for Nedis}
    \begin{tabular}{|l|l|l|l|l|l|}
        \hline
        \textbf{Date}    & \textbf{Packets} & \textbf{Bytes} & \textbf{Biggest src} & \textbf{Biggest dst} \\ \hline
        13.01-15.01      & 565,380            & 66,062,682           & 485 bytes            & 522 bytes            \\ \hline
        27.01-29.01      & 546,269            & 58,677,501           & 424 bytes            & 522 bytes            \\ \hline
        03.02-05.02      & 501,936            & 62,219,835           & 485 bytes            &    522 bytes            \\ \hline
        17.02-19.02      & 501,747            & 64,037,197           & 485 bytes            & 522 bytes            \\ \hline
        10.03-12.03      & 482,205            & 60,333,999           & 485 bytes            & 522 bytes            \\ \hline
        28.03-30.03      & 578,033            & 65,159,251           & 424 bytes            & 426 bytes            \\ \hline
        31.03-02.04      & 621,200            & 81,810,849           & 485 bytes            & 458 bytes            \\ \hline
    \end{tabular}
    \label{tab:NedisHomeWeekends}
\end{table}

\begin{table}[H]
    \centering
    \caption{Calculations for weekends gone for Nedis}
    \begin{tabular}{|l|l|l|l|l|l|}
        \hline
        \textbf{Date} & \textbf{Packets} & \textbf{Bytes} & \textbf{Biggest src} & \textbf{Biggest dst} \\ \hline
        13.01-15.01   & 678,979            & 98,642,277           & 485 bytes            & 522 bytes            \\ \hline
        27.01-29.01   & 655,417            & 91,850,956           & 485 bytes            & 650 bytes            \\ \hline
        03.02-05.02   & 518,316            & 66,069,105           & 485 bytes            & 522 bytes            \\ \hline
        17.02-19.02   & 426,157            & 53,137,691           & 424 bytes            & 522 bytes            \\ \hline
        10.03-12.03   & 357,715            & 42,390,770           & 424 bytes            & 522 bytes            \\ \hline
        28.03-30.03   & 507,506            & 65,163,914           & 485 bytes            & 522 bytes            \\ \hline
        31.03-02.04   & 634,018            & 87,175,396           & 554 bytes            & 522 bytes            \\ \hline
    \end{tabular}
    \label{tab:NedisGoneWeekends}
\end{table}

\begin{table}[H]
    \centering
    \caption{Traffic comparison of weekend values from Table \ref{tab:NedisGoneWeekends} and \ref{tab:NedisHomeWeekends}}
    \begin{tabular}{ll|l|l|l|l|}
        \cline{3-6}
        \textbf{}                                           & \textbf{} & \textbf{Packets} & \textbf{Bytes} & \textbf{Biggest src} & \textbf{Biggest dst} \\ \hline
        \multicolumn{1}{|l|}{\multirow{2}{*}{\textbf{Average}}} & Home      & 542,396          & 65,471,616     & 470 bytes            & 522 bytes            \\ \cline{2-6} 
        \multicolumn{1}{|l|}{}                              & Gone      & 539,730          & 72,061,444     & 477 bytes            & 540 bytes            \\ \hline
        \multicolumn{1}{|l|}{\multirow{2}{*}{\textbf{\gls{SD}}}}  & Home      & 49,893           & 7,665,988      & 30 bytes             & 40 bytes             \\ \cline{2-6} 
        \multicolumn{1}{|l|}{}                              & Gone      & 121,922          & 21,010,071     & 44 bytes             & 48 bytes             \\ \hline
    \end{tabular}
    \label{tab:NedisWeekends}
\end{table}

 Figure \ref{fig:NedisWeekendPackets} presents the graphs for weekend traffic measured in packets and Figure \ref{fig:NedisWeekendBytes} presents the graphs for weekend traffic measured in bytes. Both figures have the weekends at home placed on the left side of the figure framed in green and the weekends gone placed on the right side of the figure framed in orange.  

\begin{figure} [H]
    \begin{subfigure}[b]{0.47\textwidth}
    \centering
        \tcbincludegraphics[size=fbox,width=1.1\hsize,colframe=green]{figures/Nedis_Weekend_Packets_13.01-15.01.png}  
    \end{subfigure}
    \begin{subfigure}[b]{0.47\textwidth}
    \centering
        \tcbincludegraphics[size=fbox,width=1.1\hsize,colframe=orange]{figures/Nedis_Weekend_Packets_23.12-25.12.png}
    \end{subfigure}
    \begin{subfigure}[b]{0.47\textwidth}
        \tcbincludegraphics[size=fbox,width=1.1\hsize,colframe=green]{figures/Nedis_Weekend_Packets_27.01-29.01.png}
    \end{subfigure}
    \begin{subfigure}[b]{0.47\textwidth}
        \tcbincludegraphics[size=fbox,width=1.1\hsize,colframe=orange]{figures/Nedis_Weekend_Packets_30.12-01.01.png}
    \end{subfigure}
    \begin{subfigure}[b]{0.47\textwidth}
        \tcbincludegraphics[size=fbox,width=1.1\hsize,colframe=green]{figures/Nedis_Weekend_Packets_03.02-05.02.png}
    \end{subfigure}    
    \begin{subfigure}[b]{0.47\textwidth}
        \tcbincludegraphics[size=fbox,width=1.1\hsize,colframe=orange]{figures/Nedis_Weekend_Packets_20.01-22.01.png}
    \end{subfigure}
    \begin{subfigure}[b]{0.47\textwidth}
        \tcbincludegraphics[size=fbox,width=1.1\hsize,colframe=green]{figures/Nedis_Weekend_Packets_17.02-19.02.png}
    \end{subfigure}    
    \begin{subfigure}[b]{0.47\textwidth}
        \tcbincludegraphics[size=fbox,width=1.1\hsize,colframe=orange]{figures/Nedis_Weekend_Packets_10.02-12.02.png}
    \end{subfigure}
    \begin{subfigure}[b]{0.47\textwidth}
        \tcbincludegraphics[size=fbox,width=1.1\hsize,colframe=green]{figures/Nedis_Weekend_Packets_10.03-12.03.png}
    \end{subfigure}    
    \begin{subfigure}[b]{0.47\textwidth}
        \tcbincludegraphics[size=fbox,width=1.1\hsize,colframe=orange]{figures/Nedis_Weekend_Packets_24.02-26.02.png}
    \end{subfigure}
    \begin{subfigure}[b]{0.47\textwidth}
        \tcbincludegraphics[size=fbox,width=1.1\hsize,colframe=green]{figures/Nedis_Weekend_Packets_28.03-30.03.png}
    \end{subfigure}
    \begin{subfigure}[b]{0.47\textwidth}
        \tcbincludegraphics[size=fbox,width=1.1\hsize,colframe=orange]{figures/Nedis_Weekend_Packets_03.03-05.03.png}
    \end{subfigure}
    \begin{subfigure}[b]{0.47\textwidth}
        \tcbincludegraphics[size=fbox,width=1.1\hsize,colframe=green]{figures/Nedis_Weekend_Packets_31.03-02.04.png}
    \end{subfigure}
    \hspace{0.6cm}
    \begin{subfigure}[b]{0.47\textwidth}
        \tcbincludegraphics[size=fbox,width=1.1\hsize,colframe=orange]{figures/Nedis_Weekend_Packets_17.03-19.03.png}
    \end{subfigure}
  \caption{Traffic patterns for weekends at home, marked in green, and gone, marked in orange, measured in total amount of packets sent and received for Nedis during the weekends}
    \label{fig:NedisWeekendPackets}
\end{figure}

\begin{figure}[H]
    \begin{subfigure}[b]{0.47\textwidth}
    \centering
        \tcbincludegraphics[size=fbox,width=1.1\hsize,colframe=green]{figures/Nedis_Weekend_Bytes_13.01-15.01.png}
    \end{subfigure}
    \begin{subfigure}[b]{0.47\textwidth}
    \centering
        \tcbincludegraphics[size=fbox,width=1.1\hsize,colframe=orange]{figures/Nedis_Weekend_Bytes_23.12-25.12.png}
    \end{subfigure}
    \begin{subfigure}[b]{0.47\textwidth}
        \tcbincludegraphics[size=fbox,width=1.1\hsize,colframe=green]{figures/Nedis_Weekend_Bytes_27.01-29.01.png}
    \end{subfigure}
    \begin{subfigure}[b]{0.47\textwidth}
        \tcbincludegraphics[size=fbox,width=1.1\hsize,colframe=orange]{figures/Nedis_Weekend_Bytes_30.12-01.01.png}
    \end{subfigure}
    \begin{subfigure}[b]{0.47\textwidth}
        \tcbincludegraphics[size=fbox,width=1.1\hsize,colframe=green]{figures/Nedis_Weekend_Bytes_03.02-05.02.png}
    \end{subfigure}
    \begin{subfigure}[b]{0.47\textwidth}
        \tcbincludegraphics[size=fbox,width=1.1\hsize,colframe=orange]{figures/Nedis_Weekend_Bytes_20.01-22.01.png}
    \end{subfigure}
    \begin{subfigure}[b]{0.47\textwidth}
        \tcbincludegraphics[size=fbox,width=1.1\hsize,colframe=green]{figures/Nedis_Weekend_Bytes_17.02-19.02.png}
    \end{subfigure}
    \begin{subfigure}[b]{0.47\textwidth}
        \tcbincludegraphics[size=fbox,width=1.1\hsize,colframe=orange]{figures/Nedis_Weekend_Bytes_10.02-12.02.png}
    \end{subfigure}
    \begin{subfigure}[b]{0.47\textwidth}
        \tcbincludegraphics[size=fbox,width=1.1\hsize,colframe=green]{figures/Nedis_Weekend_Bytes_10.03-12.03.png}
    \end{subfigure}
    \begin{subfigure}[b]{0.47\textwidth}
        \tcbincludegraphics[size=fbox,width=1.1\hsize,colframe=orange]{figures/Nedis_Weekend_Bytes_24.02-26.02.png}
    \end{subfigure}
    \begin{subfigure}[b]{0.47\textwidth}
        \tcbincludegraphics[size=fbox,width=1.1\hsize,colframe=green]{figures/Nedis_Weekend_Bytes_28.03-30.03.png}
    \end{subfigure}
    \begin{subfigure}[b]{0.47\textwidth}
        \tcbincludegraphics[size=fbox,width=1.1\hsize,colframe=orange]{figures/Nedis_Weekend_Bytes_03.03-05.03.png}
    \end{subfigure}
   \begin{subfigure}[b]{0.47\textwidth}
        \tcbincludegraphics[size=fbox,width=1.1\hsize,colframe=green]{figures/Nedis_Weekend_Bytes_31.03-02.04.png}
    \end{subfigure}
    \hspace{0.6cm}
    \begin{subfigure}[b]{0.47\textwidth}
        \tcbincludegraphics[size=fbox,width=1.1\hsize,colframe=orange]{figures/Nedis_Weekend_Bytes_17.03-19.03.png}
    \end{subfigure}
  \caption{Traffic patterns for weekends at home, marked in green, and gone, marked in orange, measured in total amount of bytes sent and received for Nedis during the weekends}
    \label{fig:NedisWeekendBytes}
\end{figure}

The results in Table \ref{tab:NedisHomeWeekends} and \ref{tab:NedisGoneWeekends} shows that the amount of packets does not change whether its a weekend at home or gone. Both tables shows that there are sent and received around 500,000 to 600,000 packets regardless if the user is home or not. The amount of bytes varies a lot in both tables from 58,677,501 bytes to 81,810,849 bytes for the weekends at home and from 42,390,770 bytes to 98,642,277 bytes for the weekends gone. Looking at the biggest packet received and sent shows very similar values for both weekends at home and gone. Table \ref{tab:NedisWeekends} also gives the same results. Neither average values for packets, bytes or biggest packets received or sent differs significantly between a weekend at home or gone. The standard deviation for weekends gone are some higher than for weekends at home which means that the values differ more from each other, but since the average is similar and the values differ both higher and lower than the average, it is still not possible to see any significant differences here.

The values in Table \ref{tab:NedisWeekends} shows that the average value for weekends gone are within the range of the standard deviation for weekends at home for both packets and bytes. This results that it is not possible to infer whether a user is home or gone during a weekend using these calculations from the \gls{AQM} Nedis. 

The graphs in Figures \ref{fig:NedisWeekendPackets} and \ref{fig:NedisWeekendBytes} also shows that there are no significant differences in the traffic pattern when a user is home or gone for the weekend. The spikes at around 3 am, which also were explained in the baseline capture, occurs at both nights at home and gone. The results therefore shows that it is not possible to discovery if a user is home or not by looking at the traffic patterns from Nedis. 