\chapter*{Sammendrag}
Den økende bruken av IoT enheter i hjem skaper sikkerhetsutfordringer med tanke på om disse enhetene avslører privat informasjon om brukere og hjemmemiljøet. Luftkvalitetsmålere er en type IoT
enheter som alltid er påskrudd for å måle den luftkvaliteten innendørs. Selv om luftkvalitetsmålere har vært inkludert i flere sikkerhetstester, er det mangler i forskningen på å sammenligne ulike enheter med hverandre and å utføre tester som er spesifkt designed for å se om det er forskjeller i hvor mye privat informasjon enhetene avslører. Denne masteroppgaven utfører et passivt netverksavlytnings angrep mot tre ulike luftkvalitetsmålere for å undersøke om mønstre på nettverkstrafikken endrer seg når en test utføres og om dette kan brukes for å hente ut privat informasjon for å fange opp brukeraktivitet. Alle enhetene kommuniserer over Wi-Fi og er laget av ulike selskaper. Oppgaven presenterer først standard trafikk fra enhetene for å se på mønstre på nettverkstrafikken når det ikke blir utført spesfikke tester i miljøet. Deretter, ble fire ulike tester utført i miljøet til enhetene for å se om det var mulig å hente ut informasjon om miljøet ved å se på tilsvarende nettverksmønstre. De fire testene som ble utført er lage mat, dusje, ha vindu åpent på natten og helg hjemme eller borte. Resultatene viste at kun en av luftkvalitetsmålerne avslører privat informasjon fra en av testene utført. Fra trafikken på denne enheten er det mulig å se om en bruker er hjemme eller ikke ved å se på bytes sent og mottatt og forskjeller i pakkestørrelse. Denne informasjonen kan bli misbrukt av en ondsinnet aktør ved å vite om brukeren er hjemme kun ved å se på trafikken sendt over Wi-Fi, til og fra enheten. Testene viser også at ulike luftkvalitetsmålere som kommuniserer med samme protokoll, i denne sammenheng Wi-Fi, har ulike mønstre på nettverkstrafikken og avslører privat informasjon ulikt. Dette motiverer også til å fortsette testingen på luftkvalitetsmålere, siden disse ikke har vært like populære å teste som andre IoT enheter. 