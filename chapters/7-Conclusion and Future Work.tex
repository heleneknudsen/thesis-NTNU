\chapter{Conclusions and Future Work}
This thesis has investigated three different air quality monitors to see what kind of private information can be inferred while carrying out a passive network eavesdropping attack. The three devices were Netatmo Smart Indoor Air Quality Monitor, Mill Sense and Nedis SmartLife. To test which kind of private information can be inferred from the devices, four different test cases were designed. Three test cases which targets routine behaviour; cooking, showering and window open during night, and one test case over a longer period of time; home or gone during a weekend. 

The initial baseline capturings for each device showed that the devices communicates quite differently in number of packets and bytes sent and received and packet sizes. Scripts for generating graphs of the traffic flows in a presentable way were created and used to present the results. The three routine test were compared with corresponding times for baseline capturings. For Mill Sense and Nedis SmartLife, the baseline traffic differs from the same standard traffic that is captured before an event is triggered. This shows how the network patterns of these devices can be affected by factors such as season or location. For Netatmo Smart Indoor Air Quality Monitor, the baseline traffic is comparable to the event traffic. 

The routine tests were carried out over 10 different days to have enough capturings to look for a specific traffic pattern. During these tests, only showering and window open during night triggered the sensor thresholds as expected and sent notifications to the application. For cooking, not every execution did this. When looking at the traffic patterns for these test cases and devices, there are no significant change in traffic pattern from before an event is triggered to when the event is ongoing. This applies to the three test cases cooking, showering and window open for the three devices Netatmo Smart Indoor Air Quality Monitor, Mill Sense and Nedis SmartLife.  

The weekend tests were carried out over 14 different weekends with 7 of them at home and 7 gone. For the weekend test, Mill Sense and Nedis SmartLife did not expose differences in traffic patterns from when a user was home or gone. However, for Netatmo Smart Indoor Air Quality Monitor the results shows a clear difference. When a user is home, the device sends significantly more bytes and bigger packets compared to when the user is gone. This information can be misused by an adversary to know if a user is home during a weekend or a holiday and perform malicious actions such as burglary. 

This research shows that different air quality monitors communicate with different network patterns on \gls{Wi-Fi}. It is therefore important to understand that when choosing an air quality monitor to install in a home environment or use for further testing in other researches will impact the results of private information possible to expose. Further, the results showed that Netatmo Smart Indoor Air Quality Monitor is the only air quality monitor which exposed private information whether a user is at home or gone during a weekend. The scripts created to generate graphs of traffic flows and methods used can be used to further test other air quality monitors also communicating on other communication protocols.
\\\\
\textbf{Future Work}
\\
This research is limited to investigating air quality monitors which communicates over \gls{Wi-Fi}, therefore future researches could look into differences between communication protocols and include air quality monitors that communicates over protocols such as Bluetooth, ZigBee or Z-wave. Since Netatmo Smart Indoor Air Quality Monitor reveals private information during the weekend-testing, building further on these findings could be to look into timings for when the traffic pattern changes. This research only showed during a weekend or longer gone, but testing for days or hours could also give more information as to how easy it is to see when a user is home or not. Another aspect is testing the devices in an environment with other \gls{IoT} devices to see if there are differences in test cases or if it reveals more or less private information than other \gls{IoT} devices. This thesis does not look into security measurements, but as one of the tests discovers private information on the target environment, future work should look into how implement security measurements to this case. 
