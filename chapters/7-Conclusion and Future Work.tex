\chapter{Conclusion and Future Work}



\section{Future Work}
This research only looks into air quality monitors which communicates over Wi-Fi, therefore future researches could look into differences between communication protocols and include air quality monitors that communicates over protocols such as Bluetooth, Zigbee or Z-wave. Since Netatmo Smart Indoor Air Quality Monitor reveals private information during the weekend-testing, building further on these findings could be to look into timings for when the traffic pattern changes. This research only showed during a weekend or longer gone, but testing for days or hours could also give more information as to how easy it is to see when a user is home or not. Another aspect is testing the devices in an environment with other IoT devices to see if there are differences in test cases or if it reveals more or less private information than other IoT devices. This thesis does not look into security measurements, but as one of the tests discovers private information on the target environment, future work should look into how implement security measurements to this case. 
