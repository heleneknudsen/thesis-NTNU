\chapter{Related Work}
This chapter presents other research within the research scope. It is divided into two sections, security and privacy issues of air quality monitors and misusing of private information using data from IoT devices. The main emphasis is on research done on security and privacy on air quality monitors, but it is also relevant to look at security and privacy issues on IoT devices as a larger scope of research is published on this topic and IoT devices holds many of the same functionality and vulnerabilities.\\\\
\section{Security and Privacy Issues of Air Quality Monitors}
Considering security in air quality monitors, the standard and functionality varies \cite{AQMHowFarFunctionality}. Some vendors of air quality monitors use secure communication channels and authentication to access data, while others have not authentication to collect data from their devices. However, as technology evolves to support more efficient and greater computing power for the air quality monitors, it becomes easier to develop and implement stronger security. One the other side, air quality monitors are beginning to become more popular as they are increasing in functionality on which sensors they have implemented, and therefore the focus on security are being down-prioritized \cite{SecurityAndDataIntInAQM}.
\\\\ 
A research conducted on several IoT devices included two types of air quality monitors, Awair Air Monitor and Netatmo Weather Station \cite{IoTSecurityandPrivacyImpl}. The research made experiments on the devices confidentiality, integrity, authentication, access control and reflection attacks. The results for confidentiality both the devices passed the test, however for the integrity and authentication test both air quality monitors failed tests for DNSSEC and DNS Spoofing. Netatmo Weather station passed its applicable tests for access control, but the Awair Air Monitor had open ports for TCP and UDP and where vulnerable for ICMP and UDP DDoS attacks. The Awair Air Monitor where also vulnerable for ICMP Reflection attacks \cite{IoTSecurityandPrivacyImpl}.
\\\\
\cite{SecurityAndDataIntInAQM} studies a low-cost air quality monitoring system. To investigate security and data integrity, its architecture and communication is analyzed using a HTTPS proxy tool called, mitmproxy. Their results shows that the particular system uses unencrypted communication and MAC addresses to identify the sensors. They conduct two different attacks, man-in-the-middle and populating false data to the communication link with the device's MAC address. This vulnerability is a big security challenge as it is possible for an attacker to falsify any a.com device with spoofing their MAC address \cite{SecurityAndDataIntInAQM}.
\\\\
A literature review conducted in \cite{PrivacyOnGeneralIoT} of several IoT privacy related articles shows that IoT devices has several privacy issues that needs to be addressed. Even though this research is not specifically on air quality monitors, it is applicable to general IoT devices and therefore relevant to this chapter. The research highlights complexity of networking as a challenge to security for IoT devices. Since the devices, including air quality monitors, communicate over different protocols, one standard secure channel for all devices is not realistic. Another security issue here is misconfiguration. The users can buy the devices of-shelf and installs them themselves and can lead to a weakness in the system. Resource constraints in IoT devices can effect the possibility and means to incorporate security features in the devices \cite{PrivacyOnGeneralIoT}.
\\\\
A study, conducted in 2009, on six different households in America shows results that the users did not care about the privacy of the data from the air quality monitors because they did not consider it private information \cite{inAirPrivacy}. The study in itself states that it is likely that the participants did not think about the full range of private information that can be extracted from a set of data from an air quality monitor. Even though this study is old, the users aspect of privacy is of interest.
\\\\
The fact that indoor air quality monitors does monitor users environment risks that private information about users living, working or habits can be leaked \cite{IAQMonitorCommunicationReview}. The research done in \cite{IAQMonitorCommunicationReview} reviews several aspects of security in communication protocols for indoor air quality monitors. The change in Co2 in users environment can reveal information about when a user is sleeping or working and can be misused by a malicious attacker. The research investigated 91 papers and only 2 of them looked into the challenges and possible solutions to data privacy-related issues.  
\\\\
In \cite{PrivacyAndFrameworkDecentAQM} a decentralized framework for wireless communication protocols for air quality monitors are proposed. The framework uses encryption and onion routing technology to send data. The onions that transmits the packets does not access the encryption keys that encrypts the body of the onion, as the keys are only included in the last layer of the onion message. The framework can therefore protect against malicious attackers eavesdropping private information from the air quality monitors. The sensors nodes communicates with sink nodes that initiates the communication, which means that a change in the environment for the user will not result in data being transmitted immediately. Other techniques such as padding, data-link-layer encryption, timing intervals and randomized paths are also used to prevent eavesdropping. 

\section{Misusing of Private Information Using Data from IoT Devices}
In \cite{PassiveInferenceIoT} a research on one IoT hub communicating with up to 16 different IoT devices where investigated and results showed that it is possible to infer up to 90\% of user behaviour with a passive attack. The traffic is encrypted, but it was possible to infer the users action by comparing the users action with the observed encrypted application data sized received by the gateway. 
\\\\
\cite{IoTPrivacyAndMisuse} aim to classify different privacy threats and challenges for IoT devices. The different categories they suggest are identification, localization and tracking, profiling, privacy-violating interaction and presentation, life-cycle transitions, inventory attack and linkage. The threat of identification refers to collecting data about an individual that can be used to identify that person, such as an image, name or address. Further on, location and tracking is stated as an important feature of IoT systems and has become more specific and can therefore also track users indoor activity. Collecting location data can also track the users behaviour and compare indoor and outdoor routines to forecast where a users is located. Since IoT devices integrates as a part of our everyday life, profiling for directed and personal preferences can be gathered for misuse. Privacy-violating interaction and presentation are referred to as a threat more applicable to the future since this type of technology is not too common. However, identifying that presenting private information in a real-world environment is a privacy threat is important for the development of these services. For life-cycle threats, the research \cite{IoTPrivacyAndMisuse} highlights that IoT devices does not have established standards for total memory wipes or physical destruction leaving possible private information stored history of private data. Inventory attacks combines information about IoT devices and their characteristics and can be used by burglars to target potential victims. Lastly, the research states linkage between interconnected systems that shares data in a way that was not disclosed to the sources when they where isolated. 
\\\\
N. Apthorpe et al in \cite{VulEncIoTTraffic} proposes a three step strategy an attacker can carry out to passively network eavesdrop traffic from IoT devices to infer private information. The attacker first have to identify the different IoT devices in a smart home and recreate the smart home in its own environment. Then by doing normal user cases and looking at the variations from general traffic, it is possible to infer users private behaviour. However, the less functionality the devices has, the easier it is to infer the user behaviour as a change in traffic can correspond to the functionality being used \cite{VulEncIoTTraffic}. 
\\\\
A research testing several smart home devices in \cite{SpyingonSmartHomes}, reveals that it is possible to infer user behaviour based on traffic rates. Traffic analysis from a IoT camera reveals when a user is actively looking at the camera stream and a sleep monitor were inferred to show when a user is sleeping. The research is based on traffic analysis of only traffic rate and headers. All the devices tested were encrypted \cite{SpyingonSmartHomes}.
\\\\
