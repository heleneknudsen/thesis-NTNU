\chapter*{Related Work}

\section*{Security and Privacy Issues of air quality monitors}
\addcontentsline{toc}{section}{Security and Privacy Issues of air quality monitors}
A research conducted on several IoT devices included two types of Air Quality monitors, Awair Air Monitor and Netatmo Weather Station. \cite{IoTSecurityandPrivacyImpl} The research made experiments on the devices confidentiality, integrity, authentication, access control and reflection attacks. The results for confidentiality both the devices passed the test, however for the integrity and authentication test both air quality monitors failed tests for DNSSEC and DNS Spoofing. Netatmo Weather station passed its applicable tests for access control, but the Awair Air Monitor had open ports for TCP and UDP and where vulnerable for ICMP and UDP DDoS attacks. The Awair Air Monitor where also vulnerable for ICMP Reflection attacks. \cite{IoTSecurityandPrivacyImpl}

\section*{Privacy inference on air quality monitors}
\addcontentsline{toc}{section}{Privacy inference on air quality monitors}

\section*{Misusing of private information from IoT devices}
\addcontentsline{toc}{section}{Misusing of private information from IoT devices}